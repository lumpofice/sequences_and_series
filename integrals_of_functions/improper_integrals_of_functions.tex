\section{Improper Integrals}

\begin{definition}
If
\begin{align*}
    \int_{a}^{t} f(x) dx
\end{align*}
exists for all $t \geq a$, then 
\begin{align*}
    \lim_{t \longrightarrow \infty} \int_{a}^{t} f(x) dx = \int_{a}^{\infty} f(x) dx
\end{align*}
If
\begin{align*}
    \int_{t}^{a} f(x) dx
\end{align*}
exists for all $t \leq a$, then 
\begin{align*}
    \lim_{t \longrightarrow -\infty} \int_{t}^{a} f(x) dx = \int_{-\infty}^{a} f(x) dx
\end{align*}
\end{definition}

\begin{theorem}
For functions that follow the structure
\begin{align*}
    f(x) = \dfrac{1}{x^{p}} \hspace{20pt} x \in [1, \infty)
\end{align*}
we have the following
\begin{align*}
    \int_{1}^{\infty} \dfrac{1}{x^{p}} dx
\end{align*}
is convergent for $p > 1$ and divergent for $p \leq 1$
\end{theorem}

\begin{example}
Take $f(x) = \dfrac{1}{x}$ from $[1, \infty)$. We set up the integral as follows
\begin{align*}
    \int_{1}^{\infty} \dfrac{1}{x} dx
\end{align*}
and then replace the $\infty$ in the upper bound with a variable, say $b$, and push that variable to $\infty$ as follows
\begin{align*}
    \int_{1}^{\infty} \dfrac{1}{x} dx = \lim_{b \longrightarrow \infty} \int_{1}^{b} \dfrac{1}{x} dx
\end{align*}
Now, we can evaluate using the Fundamental Theorem of Calculus, per usual
\begin{align*}
    \lim_{b \longrightarrow \infty} \int_{1}^{b} \dfrac{1}{x} dx &= \lim_{b \longrightarrow \infty} (\ln x) \Big|_{1}^{b}\\[2ex]
    &= \lim_{b \longrightarrow \infty} \ln b - \ln 1 
\end{align*}
As you can see, when $b \longrightarrow \infty$, the integral pushes to infinity. 
\end{example}

\begin{definition}
If $f$ is continuous on $[a, b)$ and is discontinuous at $b$, then 
\begin{align*}
    \lim_{t \longrightarrow b^{-}} \int_{a}^{t} f(x) dx = \int_{a}^{b} f(x) dx
\end{align*}
If $f$ is continuous on $(a, b]$ and is discontinuous at $a$, then 
\begin{align*}
    \lim_{t \longrightarrow a^{+}} \int_{t}^{b} f(x) dx = \int_{a}^{b} f(x) dx
\end{align*}
\end{definition}

\begin{example}
Let us take a shot at finding the integral
\begin{align*}
    \int_{0}^{3} \dfrac{dx}{x - 1}
\end{align*}
Because the integrand is discontinuous at $x = 1$, we must break apart the integral as follows
\begin{align*}
    \int_{0}^{3} \dfrac{dx}{x - 1} = \int_{0}^{1} \dfrac{dx}{x-1} + \int_{1}^{3} \dfrac{dx}{x-1} 
\end{align*}
Let us evaluate the first integral in the sum on the right hand side
\begin{align*}
    \lim_{t \longrightarrow 1^{-}} \int_{0}^{t} \dfrac{dx}{x-1} &= \lim_{t \longrightarrow 1^{-}} \ln \lvert x - 1 \rvert \Big|_{0}^{t}\\[2ex]
    &= \lim_{t \longrightarrow 1^{-}} (\ln \lvert t - 1 \rvert - \ln \lvert -1 \rvert)\\[2ex]
    &= \ln \Big\lvert \lim_{t \longrightarrow 1^{-}} (t - 1) \Big\rvert\\[2ex]
    &= \ln \Big(1 - \lim_{t \longrightarrow 1^{-}} t \Big)
\end{align*}
If is clear that this portion of the integral diverges. Thus, the entire integral diverges. 
\end{example}

\begin{exercise}
Determine whether the integral is convergent or divergent
\begin{align*}
    \int_{-\infty}^{-1} \dfrac{1}{\sqrt{2-x}} dx
\end{align*}
\end{exercise}

\begin{exercise}
Determine whether the integral is convergent or divergent
\begin{align*}
    \int_{-\infty}^{-1} e^{-2x} dx
\end{align*}
\end{exercise}

\begin{exercise}
Determine whether the integral is convergent or divergent
\begin{align*}
    \int_{-\infty}^{\infty} xe^{-x^{2}} dx
\end{align*}
\end{exercise}

\begin{exercise}
Determine whether the integral is convergent or divergent
\begin{align*}
    \int_{-\infty}^{\infty} \cos(\pi x) dx
\end{align*}
\end{exercise}

\begin{exercise}
Determine whether the integral is convergent or divergent
\begin{align*}
    \int_{-\infty}^{6} xe^{x/3} dx
\end{align*}
\end{exercise}

\begin{exercise}
Determine whether the integral is convergent or divergent
\begin{align*}
    \int_{0}^{\infty} \dfrac{x \arctan(x)}{(1+x^{2})^{2}}dx
\end{align*}
\end{exercise}

\begin{exercise}
Determine whether the integral is convergent or divergent
\begin{align*}
    \int_{e}^{\infty} \dfrac{1}{x (\ln(x))^{3}}dx
\end{align*}
\end{exercise}

\begin{exercise}
Determine whether the following is convergent or divergent
\begin{align*}
    \lim_{c \longrightarrow 1^{+}} \int_{1}^{\infty} \dfrac{1}{x \ln(cx)} dx
\end{align*}
\end{exercise}