\section{Improper Integrals}

\begin{theorem}
For functions that follow the structure
\begin{align*}
    f(x) = \dfrac{1}{x^{p}} \hspace{20pt} x \in [1, \infty)
\end{align*}
we have the following
\begin{align*}
    \int_{1}^{\infty} \dfrac{1}{x^{p}} dx
\end{align*}
is convergent for $p > 1$ and divergent for $p \leq 1$
\end{theorem}

\begin{example}
Take $f(x) = \dfrac{1}{x}$ from $[1, \infty)$. We set up the integral as follows
\begin{align*}
    \int_{1}^{\infty} \dfrac{1}{x} dx
\end{align*}
and then replace the $\infty$ in the upper bound with a variable, say $b$, and push that variable to $\infty$ as follows
\begin{align*}
    \int_{1}^{\infty} \dfrac{1}{x} dx = \lim_{b \longrightarrow \infty} \int_{1}^{b} \dfrac{1}{x} dx
\end{align*}
Now, we can evaluate using the Fundamental Theorem of Calculus, per usual
\begin{align*}
    \lim_{b \longrightarrow \infty} \int_{1}^{b} \dfrac{1}{x} dx &= \lim_{b \longrightarrow \infty} (\ln x) \Big|_{1}^{b}\\[2ex]
    &= \lim_{b \longrightarrow \infty} \ln b - \ln 1 
\end{align*}
As you can see, when $b \longrightarrow \infty$, the integral pushes to infinity. 
\end{example}