\section{Introduction to Sequences of Numbers}\label{introduction_to_sequences_of_numbers}

\begin{definition}
A sequence of real numbers is a mapping
\begin{align*}
    f: \mathbb{N} \longrightarrow \mathbb{R}: f(n) \mapsto a_{n}
\end{align*}
We typically write the sequence in the following concise way:
\begin{align*}
    \{f(n)\}_{n=1}^{\infty} \hspace{20pt} \text{or} \hspace{20pt} \{a_{n}\}_{n=1}^{\infty}
\end{align*}
where $a_{n}$ is the real number function value at index $n$.
\end{definition}

\begin{example}
\begin{align*}
    &\Big\{\dfrac{1}{n}\Big\}_{n=1}^{\infty}\\[2ex]
    = \hspace{4pt} &\Big\{\dfrac{1}{1}, \hspace{4pt} \dfrac{1}{2}, \hspace{4pt} \dfrac{1}{3}, \hspace{4pt} \cdots \Big\}
\end{align*}
Here, $f(n) = \dfrac{1}{n}$ is the function definition from the natural numbers $\mathbb{N}$ to the real numbers $\mathbb{R}$.
\end{example}

\begin{exercise}
Write the $5^{\text{th}}$ term for the following sequence of real numbers:
\begin{align*}
    \Big\{\dfrac{n}{n+(1/n)}\Big\}_{n=1}^{\infty}
\end{align*}
\end{exercise}

\begin{exercise}
Write the first five terms for the following sequence of real numbers:
\begin{align*}
    \Big\{\dfrac{(-1)^{n} \cdot n}{(4^{n}/3)}\Big\}_{n=1}^{\infty}
\end{align*}
\end{exercise}

Sometimes, the domain is extended or attenuated a bit to accommodate the function definition.

\begin{exercise}
Write the first five terms for the following sequence of real numbers:
\begin{align*}
    \{\sqrt{n-b} \hspace{2pt}\}_{n=c}^{\infty} \hspace{4pt} \text{,} \hspace{10pt} \text{where $b > 1$ and $c = \lceil b \rceil$}
\end{align*}
\end{exercise}

\begin{exercise}
Write the first five terms for the following sequence of real numbers:
\begin{align*}
    \Big\{\cos{\Big(\dfrac{n\pi}{4} + n \pi \Big)}\Big\}_{n=0}^{\infty}
\end{align*}
\end{exercise}

A tricky part of this concept is using a sequence's function values to determine the general formula.

\begin{example}
Let's take the following sequence of terms:
\begin{align*}
    \Big\{\dfrac{5}{2}, \hspace{4pt} -\dfrac{6}{4}, \hspace{4pt} \dfrac{7}{8}, \hspace{4pt} -\dfrac{8}{16}, \hspace{4pt} \dfrac{9}{32}, \hspace{4pt} \cdots \Big\}
\end{align*}
Our job is to discover a pattern. Well, it seems every other term is negative. We know this can be established with the factor $(-1)^{n}$, but given that there are other factors we've yet to analyze thoroughly, it's not certain that this is a factor in our general expression for the sequence. It also seems the numerator begins at $5$ and increments by one unit as the sequence progresses. Given that the sequence begins at $n=5$, will our first term in the sequence be positive? We see $(-1)^{5}$ is negative. So, we fix this by adding or subtracting a $1$ from the power $n$. While it is not always the case that either adding or subtracting a $1$ will work, in this example, either approach will. So, we add a $1$, giving us $(-1)^{n+1}$ as a factor in our general sequence. Finally, the denominator seems to be some power of $2$ in each term of the sequence, and it seems those powers begin at $1$, which is equal to $5-4$. Checking the power in our second term, we have a power of $2$, which is equal to $6-4$. It seems our powers progress through the expression $n-4$, where $n$ is the index of the sequence. Thus, we have the following general expression for our sequence:
\begin{align*}
    \Big\{\dfrac{(-1)^{n+1}n}{2^{n-4}}\Big\}_{n=5}^{\infty}
\end{align*}
\end{example}

\begin{exercise}
Find the general expression of the sequence following the pattern:
\begin{align*}
    \Big\{9, \hspace{4pt} -\dfrac{8}{3}, \hspace{4pt} \dfrac{9}{9}, \hspace{4pt} -\dfrac{8}{27}, \hspace{4pt} \cdots \Big\}
\end{align*}
\end{exercise}

\newpage
\section{Limits of Sequences of Numbers}\label{limits_of_sequences_of_numbers}

\begin{definition}
A sequence $\{f(n)\}_{n=1}^{\infty}$ has a limit $L$, which we denote as 
\begin{align*}
    \lim_{n \longrightarrow \infty} f(n) = L\\[2ex]
    \text{if for all} \hspace{4pt} \epsilon \hspace{2pt} > \hspace{2pt} 0 \hspace{4pt} \text{there exists a natural number} \hspace{4pt} &N \hspace{4pt} \text{such that for all} \hspace{4pt} n \geq N \hspace{4pt} \text{we have}\\[2ex]
    \lvert f(n) - L \rvert < \epsilon
\end{align*}
Any sequence with a limit can be referred to as convergent.
\label{definition_limit_sequence_numbers}
\end{definition}

\begin{example}
$\lim_{n \longrightarrow \infty} \dfrac{1}{n} = 0$
\begin{proof}
Take $\epsilon = \dfrac{1}{k}$, where $k \in \mathbb{N}$ is arbitrary. Then by Definition \ref{definition_limit_sequence_numbers} we have
\begin{align*}
    &\Big\lvert \dfrac{1}{k+n} - 0 \Big\rvert\\[2ex]
    &= \Big\lvert \dfrac{1}{k+n} \Big\rvert\\[2ex]
    &< \dfrac{1}{k} = \epsilon, \hspace{4pt} \text{for all} \hspace{4pt} n \in \mathbb{N}
\end{align*}
\end{proof}
\label{limit_one_over_n}
\end{example}

\begin{recall}
For all $a, b \in \mathbb{R}$, we have
\begin{align*}
    &i) \hspace{4pt} \lvert a + b \rvert \hspace{2pt} \leq \hspace{2pt} \lvert a \rvert + \lvert b \rvert \\[2ex]
    &ii) \hspace{4pt} \lvert a - b \rvert \hspace{2pt} \leq \hspace{2pt} \lvert a - c \rvert + \lvert c - b \rvert \\[2ex]
    &iii) \hspace{4pt} \Big\lvert \lvert a \rvert - \lvert b \rvert \Big\rvert \hspace{2pt} \leq \hspace{2pt} \lvert a - b \rvert 
\end{align*}
These are forms of a relationship known as the Triangle Inequality.
\label{triangle_inequality}
\begin{proof}
    \begin{align*}
        &i) \hspace{4pt} (a + b)^{2} = a^{2} + 2ab + b^{2} \hspace{2pt} \leq \hspace{2pt} \lvert a \rvert ^{2} + 2 \lvert a \rvert \lvert b \rvert + \lvert b \rvert ^{2} = (\lvert a \rvert + \lvert b \rvert)^{2} \\[1ex]
        &\text{this gives us} \\[1ex]
        &\lvert a + b \rvert = \sqrt{(a + b)^{2}} \hspace{2pt} \leq \hspace{2pt} \sqrt{(\lvert a \rvert + \lvert b \rvert)^{2}} = \Big\lvert \lvert a \rvert + \lvert b \rvert \Big\rvert = \lvert a \rvert + \lvert b \rvert \\[6ex]
        &ii) \hspace{4pt} \lvert a - b \rvert = \lvert a - c + c - b \rvert \hspace{2pt} \leq \hspace{2pt} \lvert a - c \rvert + \lvert c - b \rvert \\[6ex]
        &iii_{j}) \hspace{4pt} \lvert a \rvert = \lvert a - b + b \rvert \hspace{2pt} \leq \hspace{2pt} \lvert a - b \rvert + \lvert b \rvert \hspace{10pt} \Longleftrightarrow \hspace{10pt} \lvert a \rvert - \lvert b \rvert \hspace{2pt} \leq \hspace{2pt} \lvert a - b \rvert \\[1ex]
        &iii_{jj}) \hspace{4pt} \lvert b \rvert = \lvert b - a + a \rvert \hspace{2pt} \leq \hspace{2pt} \lvert b - a \rvert + \lvert a \rvert \hspace{10pt} \Longleftrightarrow \hspace{10pt} \lvert b \rvert - \lvert a \rvert \hspace{2pt} \leq \hspace{2pt} \lvert b - a \rvert = \lvert a - b \rvert \\[1ex]
        &\text{By $iii_{j})$ and $iii_{jj}$, we have} \\[1ex]
        &iii) \hspace{4pt} \Big\lvert \lvert a \rvert - \lvert b \rvert \Big\rvert \hspace{2pt} \leq \hspace{2pt} \lvert a - b \rvert
    \end{align*}
\end{proof}
\end{recall}

\begin{theorem}
    If $\{f(n)\}_{n = 1}^{\infty}$ is convergent, then $\{f(n)\}_{n = 1}^{\infty}$ is bounded.
    \begin{proof}
        \text{Let } $f$ \text{ be the limit of sequence } $\{f(n)\}_{n = 1}^{\infty}$. \text{ Then for all } $\epsilon \hspace{2pt} > \hspace{2pt} 0$ \text{ there exists an } $N \in \mathbb{N}$ \text{ such that when } $n \hspace{2pt} \geq \hspace{2pt} N$ \text{ we have}
        \begin{align*}
            \lvert f(n) - f \rvert \hspace{2pt} < \hspace{2pt} \epsilon
        \end{align*}
        \text{By the Triangle Inequality, we have }
        \begin{align*}
            &\Big\lvert \lvert f(n) \rvert - \lvert f \rvert \Big\rvert \hspace{2pt} \leq \hspace{2pt} \lvert f(n) - f \rvert \hspace{2pt} < \hspace{2pt} \epsilon \\[1ex]
            &\Longrightarrow \hspace{10pt} \lvert f(n) \rvert - \lvert f \rvert \hspace{2pt} < \hspace{2pt} \epsilon \\[1ex]
            &\Longleftrightarrow \hspace{10pt} \lvert f(n) \rvert \hspace{2pt} < \hspace{2pt} \lvert f \rvert + \epsilon \hspace{10pt} \forall n \hspace{2pt} \geq \hspace{2pt} N
        \end{align*}
        \text{There exists a finite number of terms in the sequence with indices that do not meet the condition } $n \hspace{2pt} \geq \hspace{2pt} N$. \text{ So, we can set the maximum of this sequence as}
        \begin{align*}
            M = \{\lvert f(1) \rvert \hspace{2pt}, \hspace{2pt} ... \hspace{2pt}, \hspace{2pt} \lvert f(N) \rvert \hspace{2pt}, \hspace{2pt} \lvert f \rvert + \epsilon\}
        \end{align*}
        \text{and show boundedness through the following: Let } $\epsilon > 0,$ \text{ then}
        \begin{align*}
            \exists N \in \mathbb{N} \hspace{10pt} \text{such that} \hspace{10pt} \forall n \hspace{2pt} \geq \hspace{2pt} N, \hspace{4pt}  \lvert f(n) \rvert \hspace{2pt} < \hspace{2pt} \lvert f \rvert \hspace{2pt} + \hspace{2pt} \epsilon \hspace{2pt} \leq \hspace{2pt} M
        \end{align*}
        \text{Thus, for all } $n \in \mathbb{N} \hspace{10pt}$ \text{we have} $\hspace{10pt} \lvert f(n) \rvert < M$.
    \end{proof}
\end{theorem}

\begin{theorem}
If $\{f(n)\}_{n=1}^{\infty}$ and $\{g(n)\}_{n=1}^{\infty}$ are convergent sequences, and if $c \in \mathbb{R}$, then
\begin{align*}
    &i) \hspace{4pt} \lim_{n \longrightarrow \infty} (f(n) + g(n)) = \lim_{n \longrightarrow \infty} f(n) + \lim_{n \longrightarrow \infty} g(n) \\[2ex]
    &ii) \hspace{4pt} \lim_{n \longrightarrow \infty} (f(n) \cdot g(n)) = \lim_{n \longrightarrow \infty} f(n) \cdot \lim_{n \longrightarrow \infty} g(n)\\[2ex]
    &iii) \hspace{4pt} \lim_{n \longrightarrow \infty} cf(n) = c\lim_{n \longrightarrow \infty} f(n)\\[2ex]
    &iv) \hspace{4pt} \lim_{n \longrightarrow \infty}(f(n) - g(n)) = \lim_{n \longrightarrow \infty} f(n) - \lim_{n \longrightarrow \infty} g(n)\\[2ex]
    &v) \hspace{4pt} \lim_{n \longrightarrow \infty}\dfrac{f(n)}{g(n)} = \dfrac{\lim_{n \longrightarrow \infty} f(n)}{\lim_{n \longrightarrow \infty} g(n)}, \hspace{4pt} \text{ when } \hspace{4pt} \lim_{n \longrightarrow \infty} g(n) \neq 0\\[2ex]
    &vi) \hspace{4pt} \lim_{n \longrightarrow \infty} (f(n))^{k} = (\lim_{n \longrightarrow \infty} f(n))^{k}, \hspace{4pt} \forall k \in \mathbb{N}\\[2ex]
    &vii) \hspace{4pt} \lim_{n \longrightarrow \infty} c = c
\end{align*}
\label{properties_limit_sequence_numbers}
\newpage
\begin{proof} 
    \text{We prove each part, using previous results in the proof of subsequent results.}\\[8ex]
    \text{Since } $\{f(n)\}_{n = 1}^{\infty}$ \text{ and } $\{g(n)\}_{n = 1}^{\infty}$ \text{ are convergent, we let their limits be } $f$ \text{ and } $g$, \text{ respectively.}\\
    i) \text{Take } $\epsilon \hspace{2pt} > \hspace{2pt} 0$; \text{ then there exists } $K_{1} \in \mathbb{N}$ \text{ such that } $\forall n \hspace{2pt} \geq \hspace{2pt} K_{1}$, \text{ and there exists } $K_{2} \in \mathbb{N}$ \text{such that } $\forall n \hspace{2pt} \geq \hspace{2pt} K_{2}$, \text{ and we have}
    \begin{align*}
        \lvert f(n) - f \rvert \hspace{2pt} < \hspace{2pt} \epsilon \hspace{20pt} \text{and} \hspace{20pt} \lvert g(n) - g \rvert \hspace{2pt} < \hspace{2pt} \epsilon. 
    \end{align*}
    \text{Thus, for all } $n \hspace{2pt} \geq \hspace{2pt} K = \max\{K_{1}, K_{2}\}$, \text{ we have} 
    \begin{align*}
        \lvert (f(n) - g(n)) - (f-g) \rvert = \lvert (f(n) - f) + (g - g(n)) \rvert \hspace{2pt} \leq \hspace{2pt} \lvert f(n) - f \rvert + \lvert g(n) - g \rvert \hspace{2pt} < \hspace{2pt} 2\epsilon \\[4ex]
    \end{align*}
    ii) \text{Take } $\epsilon \hspace{2pt} > \hspace{2pt} 0$; \text{ then there exists } $K_{1} \in \mathbb{N}$ \text{ such that } $\forall n \hspace{2pt} \geq \hspace{2pt} K_{1}$, \text{and there exists } $K_{2} \in \mathbb{N}$ \text{such that } $\forall n \hspace{2pt} \geq \hspace{2pt} K_{2}$, \text{ and we have}
    \begin{align*}
        \lvert f(n) - f \rvert \hspace{2pt} < \hspace{2pt} \epsilon \hspace{20pt} \text{and} \hspace{20pt} \lvert g(n) - g \rvert \hspace{2pt} < \hspace{2pt} \epsilon. 
    \end{align*}
    \text{Thus, for all } $n \hspace{2pt} \geq \hspace{2pt} K = \max\{K_{1}, K_{2}\}$, \text{ we have} 
    \begin{align*}
        \lvert f(n)g(n) - fg \rvert = \lvert f(n)g(n) - fg(n) + fg(n) - fg) \rvert \hspace{2pt} \leq \hspace{2pt} \lvert f(n) - f \rvert \hspace{2pt} \lvert g(n) \rvert + \lvert f \rvert \hspace{2pt} \lvert g(n) - g \rvert
    \end{align*}
    \text{The sequence } $\{g(n)\}_{n = 1}^{\infty}$ \text{ being convergent means that it is bounded. So, }
    \begin{align*}
        \exists M \in \mathbb{N} \text{ such that } \forall n \in \mathbb{N}, \hspace{4pt} \lvert g(n) \rvert \leq M 
    \end{align*}
    \text{As well, the sequence } $\{f(n)\}_{n = 1}^{\infty}$ \text{ being convergent means that it is bounded. So, }
    \begin{align*}
        \exists M \in \mathbb{N} \text{ such that } \forall n \in \mathbb{N}, \hspace{4pt} \lvert f(n) \rvert \hspace{2pt} \leq \hspace{2pt} M \hspace{10pt} \Longrightarrow \hspace{10pt} \lvert f \rvert \hspace{2pt} \leq \hspace{2pt} M  
    \end{align*}
    \text{This gives us}
    \begin{align*}
        \lvert f(n) - f \rvert \hspace{2pt} \lvert g(n) \rvert + \lvert f \rvert \hspace{2pt} \lvert g(n) - g \rvert \hspace{2pt} \leq \hspace{2pt} \lvert f(n) - f \rvert \hspace{2pt} M + M \hspace{2pt} \lvert g(n) - g \rvert \hspace{2pt} < \hspace{2pt} 2\epsilon \\[4ex]
    \end{align*}
    iii) \text{Note that } $\lvert c \rvert \hspace{2pt} \leq \hspace{2pt} M,$ \text{ for some } $M \in \mathbb{N}$. \text{Take } $\epsilon \hspace{2pt} > \hspace{2pt} 0$. \text{Then there exists } $K \in \mathbb{N}$ \text{ such that }
    \begin{align*}
        \lvert f(n) - f \rvert \hspace{2pt} < \hspace{2pt} \epsilon \hspace{10pt} \forall n \hspace{2pt} \geq \hspace{2pt} K 
    \end{align*}
    \text{And we have for all } $n \hspace{2pt} \geq \hspace{2pt} K$
    \begin{align*}
        \lvert cf(n) - cf \rvert \hspace{2pt} = \hspace{2pt} \lvert c \rvert \hspace{2pt} \lvert f(n) - f \rvert \hspace{2pt} \leq \hspace{2pt} M \hspace{2pt} \lvert f(n) - f \rvert \hspace{2pt} < \hspace{2pt} M \epsilon \\[4ex] 
    \end{align*}
    \text{By iii) } $\{f(n)\}_{n = 1}^{\infty}$ \text{ and } $\{(-1)g(n)\}_{n = 1}^{\infty}$ \text{ are convergent.} \\
    \text{Let their limits be } $f$ \text{ and } $-g$, \text{ respectively.} \\
    iv) \text{Take } $\epsilon \hspace{2pt} > \hspace{2pt} 0$; \text{ then there exists } $K_{1} \in \mathbb{N}$ \text{such that } $\forall n \hspace{2pt} \geq \hspace{2pt} K_{1}$, \text{and there exists } $K_{2} \in \mathbb{N}$ \text{such that } $\forall n \hspace{2pt} \geq \hspace{2pt} K_{2}$, \text{ and we have}
    \begin{align*}
        \lvert f(n) - f \rvert \hspace{2pt} < \hspace{2pt} \epsilon \hspace{20pt} \text{and} \hspace{20pt} \lvert (-g(n)) - (-g) \rvert \hspace{2pt} < \hspace{2pt} \epsilon. 
    \end{align*}
    \text{Thus, for all } $n \hspace{2pt} \geq \hspace{2pt} K = \max\{K_{1}, K_{2}\}$ \text{ we have} 
    \begin{align*}
        &\lvert (f(n) - (-g(n))) - (f - (-g)) \rvert = \lvert (f(n) - f) + (-g + g(n)) \rvert \hspace{2pt} \leq \hspace{2pt} \lvert f(n) - f \rvert + \lvert (-g(n)) - (-g) \rvert \hspace{2pt} < \hspace{2pt} 2\epsilon \\[4ex]
    \end{align*}
    v) \text{Considering } $\{g(n)\}_{n = 1}^{\infty},$ \text{ we have}
    \begin{align*}
        \exists K \in \mathbb{N} \hspace{10pt} \text{ such that } \hspace{10pt} \forall n \hspace{2pt} \geq \hspace{2pt} K \hspace{10pt} \lvert g(n) - g \rvert \hspace{2pt} \leq \hspace{2pt} \dfrac{1}{2} \lvert g \rvert
    \end{align*}
    \text{By the Triangle Inequality, we have}
    \begin{align*}
        -\dfrac{1}{2} \lvert g \rvert \hspace{2pt} \leq \hspace{2pt} -\lvert g(n) - g \rvert \hspace{2pt} \leq \hspace{2pt} \lvert g(n) \rvert - \lvert g \rvert \hspace{2pt} \leq \hspace{2pt} \lvert g(n) - g \rvert \hspace{2pt} \leq \hspace{2pt} \dfrac{1}{2} \lvert g \rvert 
    \end{align*}
    \text{Which gives us}
    \begin{align*}
        &\dfrac{1}{2} \lvert g \rvert \hspace{2pt} \leq \hspace{2pt} \lvert g(n) \rvert \hspace{10pt} \forall n \hspace{2pt} \geq \hspace{2pt} K \\[1ex]
        &\Longleftrightarrow \hspace{10pt} \dfrac{1}{\lvert g(n) \rvert} \hspace{2pt} < \hspace{2pt} \dfrac{2}{\lvert g \rvert}
    \end{align*}
    \text{We have }
    \begin{align*}
        \Big\lvert \dfrac{1}{g(n)} - \dfrac{1}{g} \Big\rvert \hspace{2pt} = \hspace{2pt} \Big\lvert \dfrac{g}{g \cdot g(n)} - \dfrac{g(n)}{g \cdot g(n)} \Big\rvert \hspace{2pt} = \hspace{2pt} \Big\lvert \dfrac{1}{g \cdot g(n)} \Big\rvert \hspace{2pt} \lvert g(n) - g \rvert \hspace{2pt} < \hspace{2pt} \dfrac{2}{\lvert g \cdot g \rvert} \hspace{2pt} \lvert g(n) - g \rvert
    \end{align*}
    \text{Since } $\lim_{n \longrightarrow \infty} g(n) \hspace{2pt} = \hspace{2pt} g$ \text{ we can take } $\epsilon \hspace{2pt} > \hspace{2pt} 0,$ \text{ and find } $\exists M \in \mathbb{N}$ \text{ such that}
    \begin{align*}
        \forall n \hspace{2pt} \geq \hspace{2pt} M \hspace{10pt} \lvert g(n) - g \rvert \hspace{2pt} < \hspace{2pt} \epsilon
    \end{align*}
    \text{Thus, we have}
    \begin{align*}
        \Big\lvert \dfrac{1}{g(n)} - \dfrac{1}{g} \Big\rvert \hspace{2pt} < \hspace{2pt} \dfrac{2}{\lvert g \cdot g \rvert} \hspace{2pt} \lvert g(n) - g \rvert \hspace{2pt} < \hspace{2pt} \dfrac{2}{\lvert g \cdot g \rvert} \hspace{2pt} \epsilon
    \end{align*}
    \text{So, we have our limit for } $\Big\{\dfrac{1}{g(n)}\Big\}_{n = 1}^{\infty}$
    \begin{align*}
        \lim_{n \longrightarrow \infty} \dfrac{1}{g(n)} \hspace{2pt} = \hspace{2pt} \dfrac{1}{g}
    \end{align*}
    \text{Now, by iii), setting } $\varphi \hspace{2pt} = \hspace{2pt} \lim_{n \longrightarrow \infty} \varphi (n) \hspace{2pt} = \hspace{2pt} \lim_{n \longrightarrow \infty} \dfrac{1}{g(n)} \hspace{2pt} = \hspace{2pt} \dfrac{1}{g},$ \text{ we have}
    \begin{align*}
        \lim_{n \longrightarrow \infty} \dfrac{f(n)}{g(n)} \hspace{2pt} = \hspace{2pt} \lim_{n \longrightarrow \infty} f(n) \varphi (n) \hspace{2pt} = \hspace{2pt} f \cdot \varphi \hspace{2pt} = \hspace{2pt} \dfrac{f}{g} \\[4ex] 
    \end{align*}
    vi) \text{We can use induction to prove this portion of the theorem. Take } $\epsilon \hspace{2pt} > \hspace{2pt} 0$ \text{ and consider } $\{f(n)\}_{n = 1}^{\infty}.$ \text{Then we have}
    \begin{align*}
        \exists K \in \mathbb{N} \hspace{10pt} \text{ such that } \hspace{10pt} \forall n \hspace{2pt} \geq \hspace{2pt} K \hspace{10pt} \lvert f(n) - f \rvert \hspace{2pt} < \hspace{2pt} \epsilon
    \end{align*}
    \text{ For the case } $n \hspace{2pt} = \hspace{2pt} 1,$ \text{ we have } $\{(f(n))^{2}\}_{n = 1}^{\infty},$ \text{ and } $\forall n \hspace{2pt} \geq \hspace{2pt} K,$ \text{ and by the boundedness of } $\{f(n)\}_{n = 1}^{\infty}$ 
    \begin{align*}
        &\lvert (f(n))^{2} - (f)^{2} \rvert \hspace{2pt} = \hspace{2pt} \lvert f(n) \cdot f(n) - f(n) \cdot f + f(n) \cdot f - f \cdot f \rvert \\[1ex]
        &\leq \hspace{2pt} \lvert f(n) \cdot f(n) - f(n) \cdot f \rvert + \lvert f(n) \cdot f - f \cdot f \rvert \\[1ex]
        &\leq \hspace{2pt} \lvert f(n) \rvert \hspace{2pt} \lvert f(n) - f \rvert + \lvert f(n) - f \rvert \hspace{2pt} \lvert f \rvert \hspace{2pt} < \hspace{2pt} 2M \epsilon \hspace{10pt} \text{where} \hspace{10pt} \mathbb{N} \ni M \hspace{2pt} = \hspace{2pt} \max_{n \in \mathbb{N}}\{(f(n))\}
    \end{align*}
    \text{This gives us}
    \begin{align*}
        \lim_{n \longrightarrow \infty} (f(n))^{2} \hspace{2pt} = \hspace{2pt} f^{2} \hspace{2pt} = \hspace{2pt} (\lim_{n \longrightarrow \infty} f(n))^{2}
    \end{align*}
    \text{We assume the statement holds for an arbitrary } $n \hspace{2pt} = \hspace{2pt} k - 1.$ \text{ Meaning, the statement holds for } $\{(f(n))^{k - 1}\}_{n = 1}^{\infty}.$ \text{ So, for the given } $\epsilon \hspace{2pt} > \hspace{2pt} 0$ \text{ we have that there exists } $K \in \mathbb{N}$ \text{ such that for all } $n \hspace{2pt} \geq \hspace{2pt} K$
    \begin{align*}
        \lim_{n \longrightarrow \infty} (f(n))^{k - 1} \hspace{2pt} = \hspace{2pt} f^{k - 1} \hspace{2pt} = \hspace{2pt} (\lim_{n \longrightarrow \infty} f(n))^{k - 1}
    \end{align*}
    \text{Let } $N \hspace{2pt} = \hspace{2pt} \max_{n \longrightarrow \infty}\{(f(n))^{k - 1}\}.$ \text{ Then we have, for } $n \hspace{2pt} = \hspace{2pt} k,$ \text{ for } $\{(f(n))^{k}\}_{n = 1}^{\infty}, \hspace{2pt} \forall n \hspace{2pt} \geq \hspace{2pt} K,$ \text{and by the boundedness of } $\{f(n)\}_{n = 1}^{\infty}$
    \begin{align*}
        &\lvert (f(n))^{k} - (f)^{k} \rvert \hspace{2pt} = \hspace{2pt} \lvert f(n) \cdot (f(n))^{k - 1} - f(n) \cdot f^{k - 1} + f(n) \cdot f^{k - 1} - f \cdot f^{k - 1} \rvert \\[1ex]
        &\leq \hspace{2pt} \lvert f(n) \cdot (f(n))^{k - 1} - f(n) \cdot f^{k - 1} \rvert + \lvert f(n) \cdot f^{k - 1} - f \cdot f^{k - 1} \rvert \\[1ex]
        &\leq \hspace{2pt} \lvert f(n) \rvert \hspace{2pt} \lvert (f(n))^{k - 1} - f^{k - 1} \rvert + \lvert f(n) - f \rvert \hspace{2pt} \lvert f^{k - 1} \rvert \hspace{2pt} < \hspace{2pt} 2H \epsilon \hspace{10pt} \text{where} \hspace{10pt} \mathbb{N} \ni H \hspace{2pt} = \hspace{2pt} \max\{M, N\}
    \end{align*}
    \text{Finally, we have that for any } $k \in \mathbb{N}$
    \begin{align*}
        \lim_{n \longrightarrow \infty} (f(n))^{k} \hspace{2pt} = \hspace{2pt} f^{k} \hspace{2pt} = \hspace{2pt} (\lim_{n \longrightarrow \infty} f(n))^{k} \\[4ex]
    \end{align*}
    vii) \text{Let } $\epsilon \hspace{2pt} > \hspace{2pt} 0$. \text{ Then, as } $n \longrightarrow \infty,$ \text{ we have}
    \begin{align*}
        0 \hspace{2pt} = \hspace{2pt} \lvert c - c \rvert \hspace{2pt} < \hspace{2pt} \epsilon 
    \end{align*}
    \text{So, } $\lim_{n \longrightarrow \infty} c \hspace{2pt} = \hspace{2pt} c$
\end{proof}
\end{theorem}

Now that we have confidence with these results, we can use them to find limits of other sequences.

\begin{example}
$\lim_{n \longrightarrow \infty} \dfrac{n}{n+1} = 1$
\begin{proof}
We have a trick we can employ to find the above limit. By Example \ref{limit_one_over_n} and Theorem \ref{properties_limit_sequence_numbers},
\begin{align*}
    &\dfrac{n}{n+1} = 1 \cdot \dfrac{n}{n+1}
    = \dfrac{1/n}{1/n} \cdot \dfrac{n}{n+1}
    = \dfrac{(1/n) \cdot n}{(1/n) \cdot (n+1)}
    = \dfrac{1}{1 + (1/n)}\\[2ex]
    &\Longrightarrow \hspace{10pt} \lim_{n \longrightarrow \infty} \dfrac{n}{n+1}
    = \lim_{n \longrightarrow \infty} \dfrac{1}{1+(1/n)}
    = \dfrac{\lim_{n \longrightarrow \infty}1}{\lim_{n \longrightarrow \infty}1 + \lim_{n \longrightarrow \infty}(1/n)}
    = \dfrac{1}{1 + 0} = 1
\end{align*}
\end{proof}
\end{example}

\begin{exercise}
Find the limit of 
\begin{align*}
    \Big\{\dfrac{n^{30}}{n^{30}+1}\Big\}_{n=1}^{\infty}
\end{align*}
\end{exercise}

\begin{exercise}
Find the limit of 
\begin{align*}
    \Big\{\dfrac{3+(1/5)n^{2}}{5n+(1/3)n^{2}}\Big\}_{n=1}^{\infty}
\end{align*}
\end{exercise}

\begin{definition}
Take a sequence $\{f(n)\}_{n=1}^{\infty}$ 
\begin{align*}
    &i) \hspace{4pt} \text{If} \hspace{4pt} \lim_{n \longrightarrow \infty} f(n) = \infty, \hspace{4pt} \text{then}\\[1ex]
    &\text{for all} \hspace{4pt} N \in \mathbb{N}, \hspace{4pt} \text{there exists an} \hspace{4pt} M \in \mathbb{N} \hspace{4pt} \text{such that for all} \hspace{4pt} n > M \hspace{4pt} \text{we have} \hspace{4pt} f(n) > N. \\[6ex]
    &ii) \hspace{4pt} \text{If} \hspace{4pt} \lim_{n \longrightarrow \infty} f(n) = -\infty, \hspace{4pt} \text{then}\\[1ex]
    &\text{for all} \hspace{4pt} N \in \mathbb{N}, \hspace{4pt} \text{there exists an} \hspace{4pt} M \in \mathbb{N} \hspace{4pt} \text{such that for all} \hspace{4pt} n > M \hspace{4pt} \text{we have} \hspace{4pt} f(n) < -N.
\end{align*}
Any sequence with an infinite limit is said to be divergent.
\label{definition_infinite_limit_sequence}
\end{definition}

\begin{example}
Take $\{n\}_{n=1}^{\infty}$. Clearly, as $n \longrightarrow \infty$, the sequence of function values pushes to infinity.
\end{example}

\begin{exercise}
Show the sequence $\Big\{\dfrac{5n^{3}}{3n+1}\Big\}_{n=1}^{\infty}$ has an infinite limit. 
\end{exercise}

\begin{exercise}\label{exercise_4_ch_2}
Show the sequence $\Big\{\dfrac{n!}{2^{n}}\Big\}_{n=1}^{\infty}$ has an infinite limit.\\[1ex]
For this exercise, we merely need to show 
\begin{align*}
    n! \hspace{2pt} \geq \hspace{2pt} 2^{n} \hspace{4pt}, \hspace{10pt} \forall n \hspace{2pt} \geq \hspace{2pt} 4
\end{align*}
To show this, we use the Induction method. 
\begin{align*}
    &i) \hspace{4pt} \text{For} \hspace{4pt} n = 4 \hspace{2pt}, \hspace{10pt} 4! = 24 \hspace{2pt} \geq \hspace{2pt} 16 = 2^{4}\\[2ex]
    &ii) \hspace{4pt} \text{We assume for some} \hspace{4pt} n \in \mathbb{N}, \hspace{10pt} n! \hspace{2pt} \geq \hspace{2pt} 2^{n}\\[2ex]
    &iii) \hspace{4pt} (n+1)! = (n+1)n! \hspace{2pt} \geq \hspace{2pt} (n+1) \cdot 2^{n} \hspace{2pt} \geq \hspace{2pt} 2 \cdot 2^{n} = 2^{n+1}
\end{align*}
Thus, $\hspace{2pt} n! \hspace{2pt} \geq \hspace{2pt} 2^{n} \hspace{2pt}$ for all $\hspace{2pt} n \hspace{2pt} \geq \hspace{2pt} 4. \hspace{2pt}$ So, as $\hspace{2pt} n \longrightarrow \infty, \hspace{4pt} \dfrac{n!}{2^{n}} \hspace{2pt}$ diverges.  
\end{exercise}



\begin{theorem}
A sequence of numbers can have at most one limit.
\begin{proof}
    Let $\{f(n)\}_{n=1}^{\infty}$ be a sequence, $L = \lim_{n \longrightarrow \infty} \{f(n)\}_{n=1}^{\infty}$, and $L' = \lim_{n \longrightarrow \infty} \{f(n)\}_{n=1}^{\infty}$ .\\
    We have that for all $\epsilon > 0$, there exist $N_{1}, \hspace{4pt} N_{2} > 0$ such that
    \begin{align*}
        &n > N_{1} \hspace{20pt} \Longrightarrow \hspace{20pt} \lvert f(n) - L \rvert < \epsilon \\[2ex]
        &n > N_{2} \hspace{20pt} \Longrightarrow \hspace{20pt} \lvert f(n) - L' \rvert < \epsilon
    \end{align*}
    We can choose $M = \max\{N_1, \hspace{4pt} N_{2}\}$. Then
    \begin{align*}
        &n > M \hspace{20pt} \Longrightarrow \hspace{20pt} \lvert L - L' \rvert \leq \lvert f(n) - L \rvert + \lvert f(n) - L' \rvert < 2\epsilon
    \end{align*}
    So, $L = L^{'}$.
\end{proof}
\label{limit_uniqueness}
\end{theorem}

\begin{note}
Any sequence that does not converge to a single, finite value is considered divergent.
\end{note}

\begin{exercise}
Find the general expression of the divergent sequence following the pattern:
\begin{align*}
    \{-1, \hspace{4pt} 1, \hspace{4pt} -1, \hspace{4pt} 1, \hspace{4pt} -1, \hspace{4pt} 1, \hspace{4pt} \cdots\}
\end{align*}
\end{exercise}

\begin{exercise}
Find the general expression of the divergent sequence following the pattern:
\begin{align*}
    \{2, \hspace{4pt} 7, \hspace{4pt} 12, \hspace{4pt} 17, \hspace{4pt} \cdots\}
\end{align*}
\end{exercise}

\begin{exercise}\label{exercise_6_ch_2}
Determine whether the sequence defined by the following is convergent or divergent:
\begin{align*}
    f(1) = 1 \hspace{20pt} f(n) = 4 - f(n-1), \hspace{20pt} n \geq 1
\end{align*}
\end{exercise}

For Exercise \ref{exercise_6_ch_2}, you could use the method of Induction shown in Exercise \ref{exercise_4_ch_2}

\begin{exercise}
Determine whether the sequence defined by the following is convergent or divergent:
\begin{align*}
    f(1) = 3 \hspace{20pt} f(n) = 6 - f(n-1), \hspace{20pt} n \geq 1
\end{align*}
\end{exercise}



\begin{theorem}
Let $\{f(n)\}_{n=1}^{\infty}, \hspace{4pt} \{g(n)\}_{n=1}^{\infty}$ and $\{h(n)\}_{n=1}^{\infty}$ be sequences.
\begin{align*}
    \text{If} &\hspace{8pt} f(n) \hspace{2pt} \leq \hspace{2pt} h(n) \hspace{2pt} \leq \hspace{2pt} g(n) \hspace{4pt} \text{for all} \hspace{4pt} n \in \mathbb{N} \\[1ex]
    \text{and if} \hspace{8pt} &\lim_{n \longrightarrow \infty} f(n) = \lim_{n \longrightarrow \infty} g(n) = L \\[1ex]
    \text{then} \hspace{8pt} &\lim_{n \longrightarrow \infty} h(n) = L 
\end{align*}
\label{squeeze_theorem}
\begin{proof}
    \text{Let } $\epsilon \hspace{2pt} > \hspace{2pt} 0.$ \text{ Then } $\exists K_{1} \in \mathbb{N}$ \text{ such that}
    \begin{align*}
        \forall n \hspace{2pt} \geq \hspace{2pt} K_{1}, \hspace{10pt} \lvert f(n) - L \rvert \hspace{2pt} < \hspace{2pt} \epsilon
    \end{align*}
    \text{ and } $\exists K_{2} \in \mathbb{N}$ \text{ such that}
    \begin{align*}
        \forall n \hspace{2pt} \geq \hspace{2pt} K_{2}, \hspace{10pt} \lvert g(n) - L \rvert \hspace{2pt} < \hspace{2pt} \epsilon 
    \end{align*}
    \text{So, we take } $k = \max\{K_{1}, K_{2}\}.$ \text{ Then } $\forall n \hspace{2pt} \geq \hspace{2pt} K,$ \text{ we have}
    \begin{align*}
        f(n) \hspace{2pt} \leq \hspace{2pt} h(n) \hspace{2pt} \leq \hspace{2pt} g(n) \hspace{20pt} \Longleftrightarrow \hspace{20pt} f(n) - L \hspace{2pt} \leq \hspace{2pt} h(n) - L \hspace{2pt} \leq \hspace{2pt} g(n) - L
    \end{align*}
    \text{which gives us}
    \begin{align*}
        -\epsilon \hspace{2pt} < \hspace{2pt} f(n) - L \hspace{2pt} \leq \hspace{2pt} h(n) - L \hspace{2pt} \leq \hspace{2pt} g(n) - L \hspace{2pt} < \hspace{2pt} \epsilon
    \end{align*}
    \text{Thus, } $\forall n \hspace{2pt} \geq \hspace{2pt} K,$ \text{ we have our result}
    \begin{align*}
        -\epsilon \hspace{2pt} < \hspace{2pt} h(n) - L \hspace{2pt} < \hspace{2pt} \epsilon \hspace{20pt} \Longleftrightarrow \hspace{20pt} \lvert h(n) - L \rvert \hspace{2pt} < \hspace{2pt} \epsilon
    \end{align*}
\end{proof}
\end{theorem}

\begin{exercise}
Show the following, using Theorem \ref{squeeze_theorem}
\begin{align*}
    \lim_{n \longrightarrow \infty} \dfrac{(-1)^{n-1}n}{n^{2}+1} = 0
\end{align*}
\begin{proof}
    \begin{align*}
        &-\dfrac{1}{n} \leq \dfrac{(-1)^{n-1} \hspace{2pt} n}{n^{2} + 1} \leq \dfrac{n}{n^{2} + 1} \leq \dfrac{n}{n^{2}} = \dfrac{1}{n}\\[2ex]
        &\lim_{n \longrightarrow \infty} \hspace{2pt} \pm \hspace{2pt} \dfrac{1}{n} = 0\\[2ex]
        &\text{So,} \hspace{4pt} 0 = \lim_{n \longrightarrow \infty} -\dfrac{1}{n} \leq \lim_{n \longrightarrow \infty} \dfrac{(-1)^{n-1} \hspace{2pt} n}{n^{2} + 1} \leq \lim_{n \longrightarrow \infty} \dfrac{1}{n} = 0
    \end{align*}
\end{proof}
\end{exercise}

\begin{theorem} 
Let $\{f(n)\}_{n = 1}^{\infty}$ be a sequence of numbers.
\begin{align*}
    \text{If} \hspace{4pt} \lim_{n \longrightarrow \infty} \lvert f(n) \rvert = L \hspace{20pt} \text{then} \hspace{20pt} \lim_{n \longrightarrow \infty} f(n) = L
\end{align*}
\begin{proof}
    From Definition \ref{definition_limit_sequence_numbers}, we have
    \begin{align*}
        &\lim_{n \longrightarrow \infty}\lvert f(n) \rvert = L \hspace{20pt} \Longleftrightarrow \hspace{20pt} \text{For} \hspace{4pt} \forall \epsilon>0, \hspace{4pt} \exists K \in \mathbb{N} \hspace{4pt} \text{such that} \\[2ex] 
        &\text{when} \hspace{4pt} n \geq K, \hspace{4pt} \text{we have} \hspace{4pt} \Big\lvert \lvert f(n) \rvert - L \Big\rvert < \epsilon \\[2ex]
        &\text{Well, for all such} \hspace{4pt} n, \hspace{4pt} \text{since} \hspace{4pt} f(n) \leq \lvert f(n) \rvert, \hspace{4pt} \text{we have} \hspace{4pt} \lvert f(n) - L \rvert \leq \Big\lvert \lvert f(n) \rvert - L \Big\rvert < \epsilon \\[2ex]
        &\text{So,} \hspace{4pt} \lim_{n \longrightarrow \infty} f(n) = L
    \end{align*}
\end{proof}
\end{theorem}



\begin{theorem}
The sequence $\{r^{n}\}_{n=1}^{\infty}$ is convergent for $r \in [-1, 1]$ such that
\begin{align*}
    \lim_{n \longrightarrow \infty} r^{n} = \begin{cases}
    0, \hspace{4pt} \text{if} \hspace{4pt} -1 < r < 1,\\[2ex]
    1, \hspace{4pt} \text{if} \hspace{4pt} r = 1
    \end{cases}
\end{align*}
and is divergent, otherwise.
\label{geometric_term_sequence}
\begin{proof}
    If $\lvert r \rvert = 1$, is the result obvious? If $\lvert r \rvert < 1$, then $\lvert r \rvert \in \Big[\dfrac{k-1}{m}, \dfrac{k}{m}\Big)$, where $k < m$ and $k, m \in \mathbb{N}$. Choose $\epsilon = \dfrac{1}{k}$. Then when $N = k^{2}$ and $n > N$, we have
    \begin{align*}
        \Big|\hspace{4pt} \Big(\dfrac{k}{m}\Big)^{n} - 0 \hspace{4pt} \Big| \leq \dfrac{k^{n}}{(k+1)^{n}} \leq \dfrac{k^{n}}{k^{n} + n \cdot k^{n-1}} \leq \dfrac{k^{n}}{n \cdot k^{n-1}} = \dfrac{k}{n} < \dfrac{1}{k} = \epsilon
    \end{align*}
\end{proof}
\end{theorem}

\begin{exercise}
Find the limit of the following sequence:
\begin{align*}
    \Big\{\dfrac{3^{n+2}}{5^{n}}\Big\}_{n=1}^{\infty}
\end{align*}
\end{exercise}

\begin{exercise}
Find the limit of the following sequence:
\begin{align*}
    \Big\{\dfrac{1}{2^{n}}\Big\}_{n=1}^{\infty}
\end{align*}
\end{exercise}

\begin{exercise}
If $\$1000$ is invested at $6\%$ interest, compounded annually, then after $n$ years the investment is worth
\begin{align*}
    f(n) = 1000(1.06)^{n} \hspace{4pt} \text{dollars}
\end{align*}
Find the first five terms of the sequence $\{f(n)\}_{n=1}^{\infty}$ and determine whether or not it is convergent.
\end{exercise}

\begin{exercise}
Use Theorem \ref{squeeze_theorem} to find the limit of the following sequence:
\begin{align*}
    \Big\{\dfrac{\cos^{2}(n)}{2^{n}}\Big\}_{n=1}^{\infty}
\end{align*}
\end{exercise}

\begin{exercise}
Find the limit of the following sequence:
\begin{align*}
    \Big\{\dfrac{(-3)^{n}}{n!}\Big\}_{n=1}^{\infty}
\end{align*}
\end{exercise}

\begin{exercise}
Find the limit of the following sequence:
\begin{align*}
    \Big\{\Big(\sqrt[\leftroot{2}\uproot{2}n]{n} - 1\Big)^{n}\Big\}_{n=1}^{\infty}
\end{align*}
\end{exercise}

\newpage
\section{Monotone and Bounded Sequences of Numbers}

\begin{definition}
We say $\{f(n)\}_{n=1}^{\infty}$ is increasing if $f(n-1) < f(n)$ for all $n \geq 1$.\\[1ex]
We say $\{f(n)\}_{n=1}^{\infty}$ is decreasing if $f(n-1) > f(n)$ for all $n \geq 1$. When a function is either increasing or decreasing, it is referred to as monotonic. A function that is neither increasing nor decreasing is referred to as non-monotonic.
\end{definition}

\begin{example}
The following sequence is decreasing:
\begin{align*}
    &\Big\{\dfrac{1}{n}\Big\}_{n=1}^{\infty}\\[2ex]
    \text{since} \hspace{4pt} \dfrac{1}{n+1} &< \dfrac{1}{n} \hspace{4pt} \text{for all} \hspace{4pt} n
\end{align*}
\end{example}

\begin{example}
When it's not so obvious, we use direct comparisons, showing first $n=1$ compared with $n=2$. Then we assume it is true for some arbitrary $n=k$ compared with $n=k-1$, and we use this result to show it's true for $n=k+1$. By showing this holds for arbitrary natural number $k+1$, we can safely state this hold for all natural numbers. This process is called induction, and we will show it through the sequence $\Big\{\dfrac{1}{n} + 1\Big\}_{n=1}^{\infty}$
\begin{align*}
    &n=1: \hspace{20pt} f(1) = \dfrac{1}{1} + 1\\[2ex]
    &n=2: \hspace{20pt} f(2) = \dfrac{1}{2} + 1 < \dfrac{1}{1} + 1 = f(1)\\[2ex]
    \text{Assume} \hspace{4pt} &n=k: f(k) = \dfrac{1}{k} + 1 < f(k-1) = \dfrac{1}{k-1} + 1\\[2ex]
    \text{Now, we know}& \hspace{4pt} \dfrac{1}{k+1} < \dfrac{1}{k} \hspace{20pt} \Longleftrightarrow \hspace{20pt} \dfrac{1}{k+1} + 1 < \dfrac{1}{k} + 1\\[2ex]
    \text{Thus, we have}& \hspace{4pt} f(k+1) = \dfrac{1}{k+1} + 1 < \dfrac{1}{k} + 1 = f(k)
\end{align*}
and the sequence is monotonically decreasing.
\end{example}

\begin{exercise}
Through induction, show the following sequence is monotonic decreasing:
\begin{align*}
    \Big\{\dfrac{3}{n+5}\Big\}_{n=1}^{\infty}
\end{align*}
\end{exercise}

\begin{exercise}
Give an example of an increasing sequence. Through induction, show that it is increasing.
\end{exercise}

\begin{definition}
A sequence $\{f(n)\}_{n=1}^{\infty}$ is bounded above if
\begin{align*}
    \text{there exists an} \hspace{4pt} M \in \mathbb{Z} \hspace{4pt} \text{such that} \hspace{4pt} f(n) \leq M \hspace{4pt} \text{for all} \hspace{4pt} n
\end{align*}
A sequence $\{f(n)\}_{n=1}^{\infty}$ is bounded below if
\begin{align*}
    \text{there exists an} \hspace{4pt} M \hspace{4pt} \text{such that} \hspace{4pt} M \in \mathbb{Z} \hspace{4pt} \text{and} \hspace{4pt} f(n) \geq M \hspace{4pt} \text{for all} \hspace{4pt} n
\end{align*}
A sequence $\{f(n)\}_{n=1}^{\infty}$ is bounded if it is both bounded above and bounded below.
\end{definition}

\begin{example}
It is clear the sequence $\Big\{\dfrac{1}{n}\Big\}_{n=1}^{\infty}$ is bounded above by $1$. If we write the first few terms of the sequence
\begin{align*}
    \Big\{\dfrac{1}{1}, \hspace{4pt} \dfrac{1}{2}, \hspace{4pt} \dfrac{1}{3}, \hspace{4pt} \dfrac{1}{4}, \hspace{4pt} \dfrac{1}{5}, \hspace{4pt} \dfrac{1}{6}, \hspace{4pt} \dfrac{1}{7}, \hspace{4pt} \cdots \Big\}
\end{align*}
it appears that $0$ will be a lower bound. We can test this using induction.
\begin{align*}
    &\text{For} \hspace{4pt} n=1: \hspace{20pt} 0 < f(1)\\[2ex]
    &\text{For} \hspace{4pt} n=k: \hspace{20pt} 0 < f(k) \hspace{4pt} \text{is assumed true}\\[2ex]
    &\text{Since the sequence is decreasing, we have} \hspace{4pt} f(k+1) < f(k)\\[2ex]
    &\text{and we have} \hspace{4pt} f(k) = \dfrac{1}{k} > f(k+1) = \dfrac{1}{k+1} > 0 
\end{align*}
Thus, the sequence is bounded below by $0$. Being bounded both above and below, the sequence is bounded.
\end{example}

\begin{exercise}
Through induction, show the sequence $\Big\{\dfrac{1}{n} + 1\Big\}_{n=1}^{\infty}$ is bounded.
\end{exercise}

\begin{theorem}
Every bounded monotonic sequence is convergent.
\label{monotone_bounded_convergent}
\end{theorem}

\begin{example}
We already know, for the sequence $\Big\{\dfrac{1}{n}\Big\}_{n=1}^{\infty}$
\begin{align*}
    \lim_{n \longrightarrow \infty} \dfrac{1}{n} = 0.
\end{align*}
\end{example}

\begin{exercise}
Find the limit of the sequence
\begin{align*}
    \Big\{\dfrac{1}{n} + 1\Big\}_{n=1}^{\infty}
\end{align*}
\end{exercise}

\begin{exercise}
Suppose you know that $\{f(n)\}_{n=1}^{\infty}$ is a decreasing sequence and all its terms lie between $[5, 8]$. Explain why the sequence has a limit. What can you say about the value of the limit?
\end{exercise}

\begin{exercise}
Determine whether the sequence is increasing, decreasing, or not monotonic. Show, using induction. Is the sequence bounded? Show, using induction.
\begin{flalign*}
\text{a.} \hspace{20pt} &\{(-2)^{n+1}\}_{n=1}^{\infty}&\\[2ex]
\text{b.} \hspace{20pt} &\Big\{\dfrac{1}{2n+3}\Big\}_{n=1}^{\infty}&\\[2ex]
\text{c.} \hspace{20pt} &\Big\{\dfrac{2n-3}{3n+4}\Big\}_{n=1}^{\infty}&\\[2ex]
\text{d.} \hspace{20pt} &\{n(-1)^{n}\}_{n=1}^{\infty}&\\[2ex]
\text{e.} \hspace{20pt} &\Big\{\dfrac{n}{n^{2}+1}\Big\}_{n=1}^{\infty}&\\[2ex]
\text{f.} \hspace{20pt} &\Big\{n + \dfrac{1}{n}\Big\}_{n=1}^{\infty}&\\[2ex]
\end{flalign*}
\end{exercise}

\begin{theorem}
For any $p > 0$, we have
\begin{align*}
    \lim_{n \longrightarrow \infty} n^{-p} = 0
\end{align*}
\begin{proof}
\begin{align*}
    1 > n^{-p} = \dfrac{1}{n^{p}} > \dfrac{1}{(n+k)^{p}} > 0 \hspace{4pt} \text{for all} \hspace{4pt} k \in \mathbb{N}
\end{align*}
Choose $\epsilon = \dfrac{1}{(m+k)^p}$. Then for all $n \geq m+k$, we have
\begin{align*}
    \Big\lvert \dfrac{1}{n^{p}} - 0 \Big\rvert = \dfrac{1}{n^{p}} < \dfrac{1}{(m + k)^{p}} = \epsilon
\end{align*}
\end{proof}
\label{n_denom_limit}
\end{theorem}

\begin{exercise}
Use Theorem \ref{squeeze_theorem} and Theorem \ref{n_denom_limit} to find the limit of the following sequence:
\begin{align*}
    \Big\{\dfrac{\sin(2n)}{1+\sqrt{n}}\Big\}_{n=1}^{\infty}
\end{align*}
\end{exercise}

\begin{exercise}
A sequence $\{f(n)\}_{n=1}^{\infty}$ is defined by 
\begin{align*}
    f(1) = 1, \hspace{20pt} f(n) = \dfrac{1}{1+f(n-1)}, \hspace{20pt} n \geq 1
\end{align*}
Does the sequence have a limit? If so, find it.
\end{exercise}

\begin{exercise}
A sequence $\{f(n)\}_{n=1}^{\infty}$ is given by 
\begin{align*}
    f(1) = \sqrt{2}, \hspace{20pt} f(n) = \sqrt{2 + f(n-1)}, \hspace{20pt} n \geq 1
\end{align*}
By induction, show the sequence is increasing. Also, show the sequence is bounded above by $3$. Does the sequence have a limit (yes/no)? Support your answer with relevant definitions or theorems.
\end{exercise}

\begin{exercise}
Prove that if $\lim_{n \longrightarrow \infty} f(n) = 0$ and if $\{g(n)\}_{n=1}^{\infty}$ is bounded, then 
\begin{align*}
    \lim_{n \longrightarrow \infty} f(n)g(n) = 0
\end{align*}
\end{exercise}

\newpage
\section{Sequences: Miscellaneous} 

\begin{theorem}
Let $\{f(n)\}_{n=1}^{\infty}$ be a sequence of non-negative numbers that converges to $L$. Then we have 
\begin{align*}
    \lim_{n \longrightarrow \infty} \sqrt{f(n)} = \sqrt{L}
\end{align*}
\label{limit_passes_under_square_root}
\end{theorem}