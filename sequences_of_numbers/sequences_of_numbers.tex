\section{Introduction to Sequences of Numbers}\label{introduction_to_sequences_of_numbers}

\begin{definition}
A sequence of real numbers is a mapping
\begin{align*}
    f: \mathbb{N} \longrightarrow \mathbb{R}: f(n) \mapsto a_{n}
\end{align*}
We typically write the sequence in the following concise way:
\begin{align*}
    \{f(n)\}_{n=1}^{\infty} \hspace{20pt} \text{or} \hspace{20pt} \{a_{n}\}_{n=1}^{\infty}
\end{align*}
where $a_{n}$ is the real number function value at index $n$.
\end{definition}

\begin{example}
\begin{align*}
    &\Big\{\dfrac{1}{n}\Big\}_{n=1}^{\infty}\\[2ex]
    = \hspace{4pt} &\Big\{\dfrac{1}{1}, \hspace{4pt} \dfrac{1}{2}, \hspace{4pt} \dfrac{1}{3}, \hspace{4pt} \cdots \Big\}
\end{align*}
Here, $f(n) = \dfrac{1}{n}$ is the function definition from the natural numbers $\mathbb{N}$ to the real numbers $\mathbb{R}$.
\end{example}

\begin{exercise}
Write the $5^{\text{th}}$ term for the following sequence of real numbers:
\begin{align*}
    \Big\{\dfrac{n}{n+1}\Big\}_{n=1}^{\infty}
\end{align*}
\end{exercise}

\begin{exercise}
Write the first five terms for the following sequence of real numbers:
\begin{align*}
    \Big\{\dfrac{(-1)^{n}(n+1)}{3^{n}}\Big\}_{n=1}^{\infty}
\end{align*}
\end{exercise}

Sometimes, the domain is extended or attenuated a bit to accommodate the function definition.

\begin{exercise}
Write the first five terms for the following sequence of real numbers:
\begin{align*}
    \{\sqrt{n-10}\}_{n=10}^{\infty}
\end{align*}
\end{exercise}

\begin{exercise}
Write the first five terms for the following sequence of real numbers:
\begin{align*}
    \Big\{\cos{\Big(\dfrac{n\pi}{4}\Big)}\Big\}_{n=0}^{\infty}
\end{align*}
\end{exercise}

A tricky part of this concept is using a sequence's function values to determine the general formula.

\begin{example}
Let's take the following sequence of terms:
\begin{align*}
    \Big\{\dfrac{3}{5}, \hspace{4pt} -\dfrac{4}{25}, \hspace{4pt} \dfrac{5}{125}, \hspace{4pt} -\dfrac{6}{625}, \hspace{4pt} \dfrac{7}{3125}, \hspace{4pt} \cdots \Big\}
\end{align*}
Our job is to discover a pattern. Well, it seems every other term is negative. We know this can be established with the factor $(-1)^{n}$, but given that there are other factors we've yet to analyze thoroughly, it's not certain that this is a factor in our general expression for the sequence. It also seems the numerator begins at $3$ and increments by one unit as the sequence progresses. Given that the sequence begins at $n=3$, will our first term in the sequence be positive? We see $(-1)^{3}$ is negative. So, we fix this by adding or subtracting a $1$ from the power $n$. While it is not always the case that either adding or subtracting a $1$ will work, in this example, either approach will. So, we add a $1$, giving us $(-1)^{n+1}$ as a factor in our general sequence. Finally, the denominator seems to be some power of $5$ in each term of the sequence, and it seems those powers begin at $1$, which is equal to $3-2$. Checking the power in our second term, we have a power of $2$, which is equal to $4-2$. It seems our powers progress through the expression $n-2$, where $n$ is the index of the sequence. Thus, we have the following general expression for our sequence:
\begin{align*}
    \Big\{\dfrac{(-1)^{n+1}n}{5^{n-2}}\Big\}_{n=3}^{\infty}
\end{align*}
\end{example}

\begin{exercise}
Find the general expression of the sequence following the pattern:
\begin{align*}
    \Big\{1, \hspace{4pt} -\dfrac{2}{3}, \hspace{4pt} \dfrac{4}{9}, \hspace{4pt} -\dfrac{8}{27}, \hspace{4pt} \cdots \Big\}
\end{align*}
\end{exercise}

\newpage
\section{Limits of Sequences of Numbers}\label{limits_of_sequences_of_numbers}

\begin{definition}
A sequence $\{f(n)\}_{n=1}^{\infty}$ has a limit $L$, which we denote as 
\begin{align*}
    \lim_{n \longrightarrow \infty} f(n) = L\\[2ex]
    \text{if for all} \hspace{4pt} \epsilon > 0 \hspace{4pt} \text{there exists a natural number} \hspace{4pt} &N \hspace{4pt} \text{such that for all} \hspace{4pt} n \geq N \hspace{4pt} \text{we have}\\[2ex]
    \lvert f(n) - L \rvert < \epsilon
\end{align*}
Any sequence with a limit can be referred to as convergent.
\label{definition_limit_sequence_numbers}
\end{definition}

\begin{example}
$\lim_{n \longrightarrow \infty} \dfrac{1}{n} = 0$
\begin{proof}
Take $\epsilon = \dfrac{1}{k}$, where $k \in \mathbb{N}$ is arbitrary. Then by Definition \ref{definition_limit_sequence_numbers} we have
\begin{align*}
    &\Big\lvert \dfrac{1}{k+n} - 0 \Big\rvert\\[2ex]
    &= \Big\lvert \dfrac{1}{k+n} \Big\rvert\\[2ex]
    &< \dfrac{1}{k} = \epsilon, \hspace{4pt} \text{for all} \hspace{4pt} n \in \mathbb{N}
\end{align*}
\end{proof}
\label{limit_one_over_n}
\end{example}

\begin{theorem}
If $\{f(n)\}_{n=1}^{\infty}$ and $\{g(n)\}_{n=1}^{\infty}$ are convergent sequences, and if $c \in \mathbb{R}$, then
\begin{align*}
    &\lim_{n \longrightarrow \infty} (f(n) + g(n)) = \lim_{n \longrightarrow \infty} f(n) + \lim_{n \longrightarrow \infty} g(n) \\[2ex]
    &\lim_{n \longrightarrow \infty}(f(n) - g(n)) = \lim_{n \longrightarrow \infty} f(n) - \lim_{n \longrightarrow \infty} g(n)\\[2ex]
    &\lim_{n \longrightarrow \infty} cf(n) = c\lim_{n \longrightarrow \infty} f(n)\\[2ex]
    &\lim_{n \longrightarrow \infty} (f(n) \cdot g(n)) = \lim_{n \longrightarrow \infty} f(n) \cdot \lim_{n \longrightarrow \infty} g(n)\\[2ex]
    &\lim_{n \longrightarrow \infty}\dfrac{f(n)}{g(n)} = \dfrac{\lim_{n \longrightarrow \infty} f(n)}{\lim_{n \longrightarrow \infty} g(n)}, \hspace{4pt} \text{when} \hspace{4pt} \lim_{n \longrightarrow \infty} g(n) \neq 0\\[2ex]
    &\lim_{n \longrightarrow \infty} (f(n))^{p} = (\lim_{n \longrightarrow \infty} f(n))^{p}, \hspace{4pt} \text{where} \hspace{4pt} p > 0 \hspace{4pt} \text{and} \hspace{4pt} f(n) > 0\\[2ex]
    &\lim_{n \longrightarrow \infty} c = c
\end{align*}
\label{properties_limit_sequence_numbers}
\end{theorem}

Now that we have these result, we can use them to find limits of other sequences.

\begin{example}
$\lim_{n \longrightarrow \infty} \dfrac{n}{n+1} = 1$
\begin{proof}
We have a trick we can employ to find the above limit. By Example \ref{limit_one_over_n} and Theorem \ref{properties_limit_sequence_numbers},
\begin{align*}
    \dfrac{n}{n+1} &= 1 \cdot \dfrac{n}{n+1}
    = \dfrac{1/n}{1/n} \cdot \dfrac{n}{n+1}
    = \dfrac{(1/n) \cdot n}{(1/n) \cdot (n+1)}
    = \dfrac{1}{1 + (1/n)}\\[2ex]
    \Longrightarrow &\lim_{n \longrightarrow \infty} \dfrac{n}{n+1}
    = \lim_{n \longrightarrow \infty} \dfrac{1}{1+(1/n)}
    = \dfrac{\lim_{n \longrightarrow \infty}1}{\lim_{n \longrightarrow \infty}1 + \lim_{n \longrightarrow \infty}(1/n)}
    = \dfrac{1}{1 + 0} = 1
\end{align*}
\end{proof}
\end{example}

\begin{exercise}
Find the limit of 
\begin{align*}
    \Big\{\dfrac{n^{3}}{n^{3}+1}\Big\}_{n=1}^{\infty}
\end{align*}
\end{exercise}

\begin{exercise}
Find the limit of 
\begin{align*}
    \Big\{\dfrac{3+5n^{2}}{n+n^{2}}\Big\}_{n=1}^{\infty}
\end{align*}
\end{exercise}

\begin{definition}
Take a sequence $\{f(n)\}_{n=1}^{\infty}$ 
\begin{align*}
    \text{If} \hspace{4pt} \lim_{n \longrightarrow \infty} f(n) &= \infty, \hspace{4pt} \text{then}\\[2ex]
    \text{for all} \hspace{4pt} N \in \mathbb{N}, \hspace{4pt} \text{there exists an} \hspace{4pt} M \in \mathbb{N} \hspace{4pt} &\text{such that for all} \hspace{4pt} n > M \hspace{4pt} \text{we have} \hspace{4pt} f(n) > N.
\end{align*}
\begin{align*}
    \text{If} \hspace{4pt} \lim_{n \longrightarrow \infty} f(n) &= -\infty, \hspace{4pt} \text{then}\\[2ex]
    \text{for all} \hspace{4pt} N \in \mathbb{N}, \hspace{4pt} \text{there exists an} \hspace{4pt} M \in \mathbb{N} \hspace{4pt} &\text{such that for all} \hspace{4pt} n > M \hspace{4pt} \text{we have} \hspace{4pt} f(n) < -N.
\end{align*}
Any sequence with an infinite limit is said to be divergent.
\label{definition_infinite_limit_sequence}
\end{definition}

\begin{example}
Take $\{n\}_{n=1}^{\infty}$. Clearly, as $n \longrightarrow \infty$, the sequence of function values push to infinity.
\end{example}

\begin{exercise}
Show the sequence $\Big\{\dfrac{n^{3}}{n+1}\Big\}_{n=1}^{\infty}$ has an infinite limit. 
\end{exercise}

\begin{exercise}
Show the sequence $\Big\{\dfrac{n!}{2^{n}}\Big\}_{n=1}^{\infty}$ has an infinite limit. 
\end{exercise}

\begin{theorem}
A sequence of numbers can have at most one limit.
\label{limit_uniqueness}
\end{theorem}

\begin{note}
Any sequence that does not converge to a single, finite value is considered divergent.
\end{note}

\begin{exercise}
Find the general expression of the divergent sequence following the pattern:
\begin{align*}
    \{5, \hspace{4pt} 1, \hspace{4pt} 5, \hspace{4pt} 1, \hspace{4pt} 5, \hspace{4pt} 1, \hspace{4pt} \cdots\}
\end{align*}
\end{exercise}

\begin{exercise}
Find the general expression of the divergent sequence following the pattern:
\begin{align*}
    \{2, \hspace{4pt} 7, \hspace{4pt} 12, \hspace{4pt} 17, \hspace{4pt} \cdots\}
\end{align*}
\end{exercise}

\begin{exercise}
Determine whether the sequence defined by the following is convergent or divergent:
\begin{align*}
    f(1) = 1 \hspace{20pt} f(n) = 4 - f(n-1), \hspace{20pt} n \geq 1
\end{align*}
\end{exercise}

\begin{exercise}
Determine whether the sequence defined by the following is convergent or divergent:
\begin{align*}
    f(1) = 2 \hspace{20pt} f(n) = 4 - f(n-1), \hspace{20pt} n \geq 1
\end{align*}
\end{exercise}

\begin{recall}
For all $a, b \in \mathbb{R}$, we have
\begin{align*}
    \Big\lvert \lvert a \rvert - \lvert b \rvert \Big\rvert &\leq \lvert a - b \rvert \\[2ex]
    \lvert a + b \rvert &\leq \lvert a \rvert + \lvert b \rvert
\end{align*}
\label{triangle_inequality}
\end{recall}

\begin{theorem}
Let $\{f(n)\}_{n=1}^{\infty}, \hspace{4pt} \{g(n)\}_{n=1}^{\infty}$ and $\{h(n)\}_{n=1}^{\infty}$ be sequences. If
\begin{align*}
    \text{If} \hspace{4pt} f(n)\leq h(n)\leq g(n) \hspace{4pt} \text{for all} \hspace{4pt} n \in \mathbb{N} \hspace{20pt}
    & \text{and if}  \hspace{20pt} \lim_{n \longrightarrow \infty} f(n) = \lim_{n \longrightarrow \infty} g(n) = L \\[2ex]
    & \text{then} \hspace{4pt} \lim_{n \longrightarrow \infty} h(n) = L 
\end{align*}
\label{squeeze_theorem}
\end{theorem}

\begin{exercise}
Show the following, using Theorem \ref{squeeze_theorem}
\begin{align*}
    \lim_{n \longrightarrow \infty} \dfrac{(-1)^{n-1}n}{n^{2}+1} = 0
\end{align*}
\end{exercise}

\begin{exercise}
Prove the following theorem two different ways:
\begin{itemize}
    \item [1.] Using the triangle inequality from Recall \ref{triangle_inequality}
    \item [2.] Using Theorem \ref{squeeze_theorem}
\end{itemize}
\begin{align*}
    \text{If} \hspace{4pt} \lim_{n \longrightarrow \infty} \lvert f(n) \rvert = L \hspace{20pt} \text{then} \hspace{20pt} \lim_{n \longrightarrow \infty} f(n) = L
\end{align*}
\end{exercise}

\begin{exercise}
Use Theorem \ref{squeeze_theorem} to find the limit of the following sequence:
\begin{align*}
    \Big\{\dfrac{\sin(n)}{n}\Big\}_{n=1}^{\infty}
\end{align*}
\end{exercise}

\begin{theorem}
The sequence $\{r^{n}\}_{n=1}^{\infty}$ is convergent for $r \in [-1, 1]$ such that
\begin{align*}
    \lim_{n \longrightarrow \infty} r^{n} = \begin{cases}
    0, \hspace{4pt} \text{if} \hspace{4pt} -1 < r < 1,\\[2ex]
    1, \hspace{4pt} \text{if} \hspace{4pt} r = 1
    \end{cases}
\end{align*}
and is divergent, otherwise.
\label{geometric_term_sequence}
\end{theorem}

\begin{exercise}
Find the limit of the following sequence:
\begin{align*}
    \Big\{\dfrac{3^{n+2}}{5^{n}}\Big\}_{n=1}^{\infty}
\end{align*}
\end{exercise}

\begin{exercise}
Find the limit of the following sequence:
\begin{align*}
    \Big\{\dfrac{1}{2^{n}}\Big\}_{n=1}^{\infty}
\end{align*}
\end{exercise}

\begin{exercise}
If $\$1000$ is invested at $6\%$ interest, compounded annually, then after $n$ years the investment is worth
\begin{align*}
    f(n) = 1000(1.06)^{n} \hspace{4pt} \text{dollars}
\end{align*}
Find the first five terms of the sequence $\{f(n)\}_{n=1}^{\infty}$ and determine whether or not it is convergent.
\end{exercise}

\begin{exercise}
Use Theorem \ref{squeeze_theorem} to find the limit of the following sequence:
\begin{align*}
    \Big\{\dfrac{\cos^{2}(n)}{2^{n}}\Big\}_{n=1}^{\infty}
\end{align*}
\end{exercise}

\begin{exercise}
Find the limit of the following sequence:
\begin{align*}
    \Big\{\dfrac{(-3)^{n}}{n!}\Big\}_{n=1}^{\infty}
\end{align*}
\end{exercise}

\begin{exercise}
Find the limit of the following sequence:
\begin{align*}
    \Big\{\Big(\sqrt[\leftroot{2}\uproot{2}n]{n} - 1\Big)^{n}\Big\}_{n}
\end{align*}
\end{exercise}

\newpage
\section{Monotone and Bounded Sequences of Numbers}

\begin{definition}
We say $\{f(n)\}_{n=1}^{\infty}$ is increasing if $f(n-1) < f(n)$ for all $n \geq 1$.\\[1ex]
We say $\{f(n)\}_{n=1}^{\infty}$ is decreasing if $f(n-1) > f(n)$ for all $n \geq 1$. When a function is either increasing or decreasing, it is referred to as monotonic. A function that is neither increasing nor decreasing is referred to as non-monotonic.
\end{definition}

\begin{example}
The following sequence is decreasing:
\begin{align*}
    &\Big\{\dfrac{1}{n}\Big\}_{n=1}^{\infty}\\[2ex]
    \text{since} \hspace{4pt} \dfrac{1}{n+1} &< \dfrac{1}{n} \hspace{4pt} \text{for all} \hspace{4pt} n
\end{align*}
\end{example}

\begin{example}
When it's not so obvious, we use direct comparisons, showing first $n=1$ compared with $n=2$. Then we assume it is true for some arbitrary $n=k$ compared with $n=k-1$, and we use this result to show it's true for $n=k+1$. By showing this holds for arbitrary natural number $k+1$, we can safely state this hold for all natural numbers. This process is called induction, and we will show it through the sequence $\Big\{\dfrac{1}{n} + 1\Big\}_{n=1}^{\infty}$
\begin{align*}
    &n=1: \hspace{20pt} f(1) = \dfrac{1}{1} + 1\\[2ex]
    &n=2: \hspace{20pt} f(2) = \dfrac{1}{2} + 1 < \dfrac{1}{1} + 1 = f(1)\\[2ex]
    \text{Assume} \hspace{4pt} &n=k: f(k) = \dfrac{1}{k} + 1 < f(k-1) = \dfrac{1}{k-1} + 1\\[2ex]
    \text{Now, we know}& \hspace{4pt} \dfrac{1}{k+1} < \dfrac{1}{k} \hspace{20pt} \Longleftrightarrow \hspace{20pt} \dfrac{1}{k+1} + 1 < \dfrac{1}{k} + 1\\[2ex]
    \text{Thus, we have}& \hspace{4pt} f(k+1) = \dfrac{1}{k+1} + 1 < \dfrac{1}{k} + 1 = f(k)
\end{align*}
and the sequence is monotonically decreasing.
\end{example}

\begin{exercise}
Through induction, show the following sequence is monotonic decreasing:
\begin{align*}
    \Big\{\dfrac{3}{n+5}\Big\}_{n=1}^{\infty}
\end{align*}
\end{exercise}

\begin{exercise}
Give an example of an increasing sequence. Through induction, show that it is increasing.
\end{exercise}

\begin{definition}
A sequence $\{f(n)\}_{n=1}^{\infty}$ is bounded above if
\begin{align*}
    \text{there exists an} \hspace{4pt} M \in \mathbb{Z} \hspace{4pt} \text{such that} \hspace{4pt} f(n) \leq M \hspace{4pt} \text{for all} \hspace{4pt} n
\end{align*}
A sequence $\{f(n)\}_{n=1}^{\infty}$ is bounded below if
\begin{align*}
    \text{there exists an} \hspace{4pt} M \hspace{4pt} \text{such that} \hspace{4pt} M \in \mathbb{Z} \hspace{4pt} \text{and} \hspace{4pt} f(n) \geq M \hspace{4pt} \text{for all} \hspace{4pt} n
\end{align*}
A sequence $\{f(n)\}_{n=1}^{\infty}$ is bounded if it is both bounded above and bounded below.
\end{definition}

\begin{example}
It is clear the sequence $\Big\{\dfrac{1}{n}\Big\}_{n=1}^{\infty}$ is bounded above by $1$. If we write the first few terms of the sequence
\begin{align*}
    \Big\{\dfrac{1}{1}, \hspace{4pt} \dfrac{1}{2}, \hspace{4pt} \dfrac{1}{3}, \hspace{4pt} \dfrac{1}{4}, \hspace{4pt} \dfrac{1}{5}, \hspace{4pt} \dfrac{1}{6}, \hspace{4pt} \dfrac{1}{7}, \hspace{4pt} \cdots \Big\}
\end{align*}
it appears that $0$ will be a lower bound. We can test this using induction.
\begin{align*}
    &\text{For} \hspace{4pt} n=1: \hspace{20pt} 0 < f(1)\\[2ex]
    &\text{For} \hspace{4pt} n=k: \hspace{20pt} 0 < f(k) \hspace{4pt} \text{is assumed true}\\[2ex]
    &\text{Since the sequence is decreasing, we have} \hspace{4pt} f(k+1) < f(k)\\[2ex]
    &\text{and we have} \hspace{4pt} f(k) = \dfrac{1}{k} > f(k+1) = \dfrac{1}{k+1} > 0 
\end{align*}
Thus, the sequence is bounded below by $0$. Being bounded both above and below, the sequence is bounded.
\end{example}

\begin{exercise}
Through induction, show the sequence $\Big\{\dfrac{1}{n} + 1\Big\}_{n=1}^{\infty}$ is bounded.
\end{exercise}

\begin{theorem}
Every bounded monotonic sequence is convergent.
\label{monotone_bounded_convergent}
\end{theorem}

\begin{example}
We already know, for the sequence $\Big\{\dfrac{1}{n}\Big\}_{n=1}^{\infty}$
\begin{align*}
    \lim_{n \longrightarrow \infty} \dfrac{1}{n} = 0.
\end{align*}
\end{example}

\begin{exercise}
Find the limit of the sequence
\begin{align*}
    \Big\{\dfrac{1}{n} + 1\Big\}_{n=1}^{\infty}
\end{align*}
\end{exercise}

\begin{exercise}
Suppose you know that $\{f(n)\}_{n=1}^{\infty}$ is a decreasing sequence and all its terms lie between $[5, 8]$. Explain why the sequence has a limit. What can you say about the value of the limit?
\end{exercise}

\begin{exercise}
Determine whether the sequence is increasing, decreasing, or not monotonic. Show, using induction. Is the sequence bounded? Show, using induction.
\begin{flalign*}
\text{a.} \hspace{20pt} &\{(-2)^{n+1}\}_{n=1}^{\infty}&\\[2ex]
\text{b.} \hspace{20pt} &\Big\{\dfrac{1}{2n+3}\Big\}_{n=1}^{\infty}&\\[2ex]
\text{c.} \hspace{20pt} &\Big\{\dfrac{2n-3}{3n+4}\Big\}_{n=1}^{\infty}&\\[2ex]
\text{d.} \hspace{20pt} &\{n(-1)^{n}\}_{n=1}^{\infty}&\\[2ex]
\text{e.} \hspace{20pt} &\Big\{\dfrac{n}{n^{2}+1}\Big\}_{n=1}^{\infty}&\\[2ex]
\text{f.} \hspace{20pt} &\Big\{n + \dfrac{1}{n}\Big\}_{n=1}^{\infty}&\\[2ex]
\end{flalign*}
\end{exercise}

\begin{theorem}
For any $p > 0$, we have
\begin{align*}
    \lim_{n \longrightarrow \infty} n^{-p} = 0
\end{align*}
\begin{proof}
\begin{align*}
    1 > n^{-p} = \dfrac{1}{n^{p}} > \dfrac{1}{(n+k)^{p}} > 0 \hspace{4pt} \text{for all} \hspace{4pt} k \in \mathbb{N}
\end{align*}
Choose $\epsilon = \dfrac{1}{(m+k)^p}$. Then for all $n \geq m+k$, we have
\begin{align*}
    \Big\lvert \dfrac{1}{n^{p}} - 0 \Big\rvert = \dfrac{1}{n^{p}} < \dfrac{1}{(m + k)^{p}} = \epsilon
\end{align*}
\end{proof}
\label{n_denom_limit}
\end{theorem}

\begin{exercise}
Use Theorem \ref{squeeze_theorem} and Theorem \ref{n_denom_limit} to find the limit of the following sequence:
\begin{align*}
    \Big\{\dfrac{\sin(2n)}{1+\sqrt{n}}\Big\}_{n=1}^{\infty}
\end{align*}
\end{exercise}

\begin{exercise}
A sequence $\{f(n)\}_{n=1}^{\infty}$ is defined by 
\begin{align*}
    f(1) = 1, \hspace{20pt} f(n) = \dfrac{1}{1+f(n-1)}, \hspace{20pt} n \geq 1
\end{align*}
Does the sequence have a limit? If so, find it.
\end{exercise}

\begin{exercise}
A sequence $\{f(n)\}_{n=1}^{\infty}$ is given by 
\begin{align*}
    f(1) = \sqrt{2}, \hspace{20pt} f(n) = \sqrt{2 + f(n-1)}, \hspace{20pt} n \geq 1
\end{align*}
By induction, show the sequence is increasing. Also, show the sequence is bounded above by $3$. Does the sequence have a limit (yes/no)? Support your answer with relevant definitions or theorems.
\end{exercise}

\begin{exercise}
Prove that if $\lim_{n \longrightarrow \infty} f(n) = 0$ and if $\{g(n)\}_{n=1}^{\infty}$ is bounded, then 
\begin{align*}
    \lim_{n \longrightarrow \infty} f(n)g(n) = 0
\end{align*}
\end{exercise}

\newpage
\section{Sequences: Miscellaneous} 

\begin{theorem}
Let $\{f(n)\}_{n=1}^{\infty}$ be a sequence of non-negative numbers that converges to $L$. Then we have 
\begin{align*}
    \lim_{n \longrightarrow \infty} \sqrt{f(n)} = \sqrt{L}
\end{align*}
\label{limit_passes_under_square_root}
\end{theorem}