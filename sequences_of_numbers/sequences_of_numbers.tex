\documentclass{article}
\usepackage[utf8]{inputenc}
\usepackage{amsmath}
\usepackage{amsthm}
\usepackage{amssymb}
\usepackage{geometry}
\usepackage{amsfonts} 

\title{Sequences of Numbers}
\author{Jonathan Parker}
\date{Last Updated on \today}

\renewcommand*\contentsname{Table of Contents}

\newtheorem{theorem}{Theorem}[section]

\newtheorem{definition}{Definition}[section]

\newtheorem{note}{Note}[section]

\newcounter{example}[section]
\newenvironment{example}[1][]{\refstepcounter{example}\par\medskip
   \noindent \textbf{Example~\theexample. #1} \rmfamily}{\medskip}

\newcounter{exercise}[section]
\newenvironment{exercise}[1][]{\refstepcounter{exercise}\par\medskip
   \noindent \textbf{Exercise~\theexercise. #1} \rmfamily}{\medskip}
   
\newcounter{recall}[section]
\newenvironment{recall}[1][]{\refstepcounter{recall}\par\medskip
   \noindent \textbf{Recall~\therecall. #1} \rmfamily}{\medskip}
   
\setlength{\parindent}{0pt}

\begin{document}


\maketitle
\tableofcontents

\newpage
\section{Introduction to Sequences of Numbers}\label{introduction_to_sequences_of_numbers}

\begin{definition}
A sequence of real numbers is a mapping
\begin{align*}
    f: \mathbb{N} \longrightarrow \mathbb{R}: f(n) \mapsto a_{n}
\end{align*}
We typically write the sequence in the following concise way:
\begin{align*}
    \{f(n)\}_{n=1}^{\infty} \hspace{20pt} \text{or} \hspace{20pt} \{a_{n}\}_{n=1}^{\infty}
\end{align*}
where $a_{n}$ is the real number function value at index $n$.
\end{definition}

\begin{example}
\begin{align*}
    &\Big\{\dfrac{1}{n}\Big\}_{n=1}^{\infty}\\[2ex]
    = \hspace{4pt} &\Big\{\dfrac{1}{1}, \hspace{4pt} \dfrac{1}{2}, \hspace{4pt} \dfrac{1}{3}, \hspace{4pt} \cdots \Big\}
\end{align*}
Here, $f(n) = \dfrac{1}{n}$ is the function definition from the natural numbers $\mathbb{N}$ to the real numbers $\mathbb{R}$.
\end{example}

\begin{exercise}
Write the $5^{\text{th}}$ term for the following sequence of real numbers:
\begin{align*}
    \Big\{\dfrac{n}{n+1}\Big\}_{n=1}^{\infty}
\end{align*}
\end{exercise}

\begin{exercise}
Write the first five terms for the following sequence of real numbers:
\begin{align*}
    \Big\{\dfrac{(-1)^{n}(n+1)}{3^{n}}\Big\}_{n=1}^{\infty}
\end{align*}
\end{exercise}

Sometimes, the domain is extended or attenuated a bit to accommodate the function definition.

\begin{exercise}
Write the first five terms for the following sequence of real numbers:
\begin{align*}
    \{\sqrt{n-10}\}_{n=10}^{\infty}
\end{align*}
\end{exercise}

\begin{exercise}
Write the first five terms for the following sequence of real numbers:
\begin{align*}
    \Big\{\cos{\Big(\dfrac{n\pi}{4}\Big)}\Big\}_{n=0}^{\infty}
\end{align*}
\end{exercise}

A tricky part of this concept is using a sequence's function values to determine the general formula.

\begin{example}
Let's take the following sequence of terms:
\begin{align*}
    \Big\{\dfrac{3}{5}, \hspace{4pt} -\dfrac{4}{25}, \hspace{4pt} \dfrac{5}{125}, \hspace{4pt} -\dfrac{6}{625}, \hspace{4pt} \dfrac{7}{3125}, \hspace{4pt} \cdots \Big\}
\end{align*}
Our job is to discover a pattern. Well, it seems every other term is negative. We know this can be established with the factor $(-1)^{n}$, but given that there are other factors we've yet to analyze thoroughly, it's not certain that this is a factor in our general expression for the sequence. It also seems the numerator begins at $3$ and increments by one unit as the sequence progresses. Given that the sequence begins at $n=3$, will our first term in the sequence be positive? We see $(-1)^{3}$ is negative. So, we fix this by adding or subtracting a $1$ from the power $n$. While it is not always the case that either adding or subtracting a $1$ will work, in this example, either approach will. So, we add a $1$, giving us $(-1)^{n+1}$ as a factor in our general sequence. Finally, the denominator seems to be some power of $5$ in each term of the sequence, and it seems those powers begin at $1$, which is equal to $3-2$. Checking the power in our second term, we have a power of $2$, which is equal to $4-2$. It seems our powers progress through the expression $n-2$, where $n$ is the index of the sequence. Thus, we have the following general expression for our sequence:
\begin{align*}
    \Big\{\dfrac{(-1)^{n+1}n}{5^{n-2}}\Big\}_{n=3}^{\infty}
\end{align*}
\end{example}

\begin{exercise}
Find the general expression of the sequence following the pattern:
\begin{align*}
    \Big\{1, \hspace{4pt} -\dfrac{2}{3}, \hspace{4pt} \dfrac{4}{9}, \hspace{4pt} -\dfrac{8}{27}, \hspace{4pt} \cdots \Big\}
\end{align*}
\end{exercise}

\newpage
\section{Limits of Sequences of Numbers}\label{limits_of_sequences_of_numbers}

\begin{definition}
A sequence $\{f(n)\}_{n=1}^{\infty}$ has a limit $L$, which we denote as 
\begin{align*}
    \lim_{n \longrightarrow \infty} f(n) = L\\[2ex]
    \text{if for all} \hspace{4pt} \epsilon > 0 \hspace{4pt} \text{there exists a natural number} \hspace{4pt} &N \hspace{4pt} \text{such that for all} \hspace{4pt} n \geq N \hspace{4pt} \text{we have}\\[2ex]
    \lvert f(n) - L \rvert < \epsilon
\end{align*}
Any sequence with a limit can be referred to as convergent.
\label{definition_limit_sequence_numbers}
\end{definition}

\begin{example}
$\lim_{n \longrightarrow \infty} \dfrac{1}{n} = 0$
\begin{proof}
Take $\epsilon = \dfrac{1}{k}$, where $k \in \mathbb{N}$ is arbitrary. Then by Definition \ref{definition_limit_sequence_numbers} we have
\begin{align*}
    &\Big\lvert \dfrac{1}{k+n} - 0 \Big\rvert\\[2ex]
    &= \Big\lvert \dfrac{1}{k+n} \Big\rvert\\[2ex]
    &< \dfrac{1}{k} = \epsilon, \hspace{4pt} \text{for all} \hspace{4pt} n \in \mathbb{N}
\end{align*}
\end{proof}
\label{limit_one_over_n}
\end{example}

\begin{theorem}
If $\{f(n)\}_{n=1}^{\infty}$ and $\{g(n)\}_{n=1}^{\infty}$ are convergent sequences, and if $c \in \mathbb{R}$, then
\begin{align*}
    &\lim_{n \longrightarrow \infty} (f(n) + g(n)) = \lim_{n \longrightarrow \infty} f(n) + \lim_{n \longrightarrow \infty} g(n) \\[2ex]
    &\lim_{n \longrightarrow \infty}(f(n) - g(n)) = \lim_{n \longrightarrow \infty} f(n) - \lim_{n \longrightarrow \infty} g(n)\\[2ex]
    &\lim_{n \longrightarrow \infty} cf(n) = c\lim_{n \longrightarrow \infty} f(n)\\[2ex]
    &\lim_{n \longrightarrow \infty} (f(n) \cdot g(n)) = \lim_{n \longrightarrow \infty} f(n) \cdot \lim_{n \longrightarrow \infty} g(n)\\[2ex]
    &\lim_{n \longrightarrow \infty}\dfrac{f(n)}{g(n)} = \dfrac{\lim_{n \longrightarrow \infty} f(n)}{\lim_{n \longrightarrow \infty} g(n)}, \hspace{4pt} \text{when} \hspace{4pt} \lim_{n \longrightarrow \infty} g(n) \neq 0\\[2ex]
    &\lim_{n \longrightarrow \infty} (f(n))^{p} = (\lim_{n \longrightarrow \infty} f(n))^{p}, \hspace{4pt} \text{where} \hspace{4pt} p > 0 \hspace{4pt} \text{and} \hspace{4pt} f(n) > 0\\[2ex]
    &\lim_{n \longrightarrow \infty} c = c
\end{align*}
\label{properties_limit_sequence_numbers}
\end{theorem}

Now that we have these result, we can use them to find limits of other sequences.

\begin{example}
$\lim_{n \longrightarrow \infty} \dfrac{n}{n+1} = 1$
\begin{proof}
We have a trick we can employ to find the above limit. By Example \ref{limit_one_over_n} and Theorem \ref{properties_limit_sequence_numbers},
\begin{align*}
    \dfrac{n}{n+1} &= 1 \cdot \dfrac{n}{n+1}
    = \dfrac{1/n}{1/n} \cdot \dfrac{n}{n+1}
    = \dfrac{(1/n) \cdot n}{(1/n) \cdot (n+1)}
    = \dfrac{1}{1 + (1/n)}\\[2ex]
    \Longrightarrow &\lim_{n \longrightarrow \infty} \dfrac{n}{n+1}
    = \lim_{n \longrightarrow \infty} \dfrac{1}{1+(1/n)}
    = \dfrac{\lim_{n \longrightarrow \infty}1}{\lim_{n \longrightarrow \infty}1 + \lim_{n \longrightarrow \infty}(1/n)}
    = \dfrac{1}{1 + 0} = 1
\end{align*}
\end{proof}
\end{example}

\begin{exercise}
Find the limit of 
\begin{align*}
    \Big\{\dfrac{n^{3}}{n^{3}+1}\Big\}_{n=1}^{\infty}
\end{align*}
\end{exercise}

\begin{exercise}
Find the limit of 
\begin{align*}
    \Big\{\dfrac{3+5n^{2}}{n+n^{2}}\Big\}_{n=1}^{\infty}
\end{align*}
\end{exercise}

\begin{exercise}
Find the limit of 
\begin{align*}
    \{e^{1/n}\}_{n=1}^{\infty}
\end{align*}
\end{exercise}

\begin{exercise}
Find the limit of
\begin{align*}
    \Big\{\tan\Big(\dfrac{2n\pi}{1+8n}\Big)\Big\}_{n=1}^{\infty}
\end{align*}
\end{exercise}

\begin{exercise}
Find the limit of
\begin{align*}
    \Big\{\sqrt{\dfrac{n+1}{9n+1}}\Big\}_{n=1}^{\infty}
\end{align*}
\end{exercise}

\begin{definition}
Take a sequence $\{f(n)\}_{n=1}^{\infty}$ 
\begin{align*}
    \text{If} \hspace{4pt} \lim_{n \longrightarrow \infty} f(n) &= \infty, \hspace{4pt} \text{then}\\[2ex]
    \text{for all} \hspace{4pt} N \in \mathbb{N}, \hspace{4pt} \text{there exists an} \hspace{4pt} M \in \mathbb{N} \hspace{4pt} &\text{such that for all} \hspace{4pt} n > M \hspace{4pt} \text{we have} \hspace{4pt} f(n) > N.
\end{align*}
\begin{align*}
    \text{If} \hspace{4pt} \lim_{n \longrightarrow \infty} f(n) &= -\infty, \hspace{4pt} \text{then}\\[2ex]
    \text{for all} \hspace{4pt} N \in \mathbb{N}, \hspace{4pt} \text{there exists an} \hspace{4pt} M \in \mathbb{N} \hspace{4pt} &\text{such that for all} \hspace{4pt} n > M \hspace{4pt} \text{we have} \hspace{4pt} f(n) < -N.
\end{align*}
Any sequence with an infinite limit is said to be divergent.
\label{definition_infinite_limit_sequence}
\end{definition}

\begin{example}
Take $\{n\}_{n=1}^{\infty}$. Clearly, as $n \longrightarrow \infty$, the sequence of function values push to infinity.
\end{example}

\begin{exercise}
Show the sequence $\Big\{\dfrac{n^{3}}{n+1}\Big\}_{n=1}^{\infty}$ has an infinite limit. 
\end{exercise}

\begin{exercise}
Show the sequence $\Big\{\dfrac{n!}{2^{n}}\Big\}_{n=1}^{\infty}$ has an infinite limit. 
\end{exercise}

\begin{theorem}
A sequence of numbers can have at most one limit.
\label{limit_uniqueness}
\end{theorem}

\begin{note}
Any sequence that does not converge to a single, finite value is considered divergent.
\end{note}

\begin{exercise}
Find the general expression of the divergent sequence following the pattern:
\begin{align*}
    \{5, \hspace{4pt} 1, \hspace{4pt} 5, \hspace{4pt} 1, \hspace{4pt} 5, \hspace{4pt} 1, \hspace{4pt} \cdots\}
\end{align*}
\end{exercise}

\begin{exercise}
Find the general expression of the divergent sequence following the pattern:
\begin{align*}
    \{2, \hspace{4pt} 7, \hspace{4pt} 12, \hspace{4pt} 17, \hspace{4pt} \cdots\}
\end{align*}
\end{exercise}

\begin{exercise}
Determine whether the sequence defined by the following is convergent or divergent:
\begin{align*}
    f(1) = 1 \hspace{20pt} f(n) = 4 - f(n-1), \hspace{20pt} n \geq 1
\end{align*}
\end{exercise}

\begin{exercise}
Determine whether the sequence defined by the following is convergent or divergent:
\begin{align*}
    f(1) = 2 \hspace{20pt} f(n) = 4 - f(n-1), \hspace{20pt} n \geq 1
\end{align*}
\end{exercise}

\begin{recall}
For all $a, b \in \mathbb{R}$, we have
\begin{align*}
    \Big\lvert \lvert a \rvert - \lvert b \rvert \Big\rvert &\leq \lvert a - b \rvert \\[2ex]
    \lvert a + b \rvert &\leq \lvert a \rvert + \lvert b \rvert
\end{align*}
\end{recall}



\end{document}
