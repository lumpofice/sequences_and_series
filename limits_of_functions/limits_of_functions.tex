\section{Limits of Functions}

\begin{definition}
We say $f$ as a function of $x$ has a limit $L$ at $c \in \mathbb{R}$, denoted by
\begin{align*}
    &\lim_{x \longrightarrow c} f(x) = L \hspace{4pt} \text{if} \hspace{20pt} &&\text{for all} \hspace{4pt} \epsilon > 0, \hspace{4pt} \text{there exists a} \hspace{4pt} \delta > 0 \hspace{4pt} \text{such that}\\[2ex]
    &\text{for all} \hspace{4pt} x \in \text{Dom($f$)} \hspace{4pt} \text{satisfying} \hspace{4pt} \lvert x - c \rvert < \delta &&\text{we have} \hspace{4pt} \lvert f(x) - L \rvert < \epsilon
\end{align*}
\end{definition}

\begin{example}
Below is a visual of how this $\epsilon - \delta$ interplay occurs. In general, if $L \in \mathbb{R}$ is the limit of $f$ at $c \in \mathbb{R}$, then for an arbitrarily small, open interval $(L-\epsilon, L+\epsilon)$ with center $L$, there is an open interval $(c-\delta, c+\delta)$ with center $c$ containing members $x \in$ Dom($f$). With this specific example, $c=1/2$. For any $x \in$ Dom($f$) satisfying 
\begin{align*}
    \Big\lvert x - \dfrac{1}{2} \Big\rvert < \delta \hspace{20pt} \text{we have} \hspace{20pt} \lvert f(x) - L \rvert = \lvert \sin(x) - L \rvert < \epsilon \hspace{20pt} \Longleftrightarrow \hspace{20pt} \lim_{x \longrightarrow 1/2} \sin(x) = L
\end{align*}
which is equivalent to the following: For any $x \in$ Dom($f$) satisfying
\begin{align*}
    x \in \Big(\dfrac{1}{2} - \delta, \dfrac{1}{2} + \delta\Big) \hspace{20pt} \text{we have} \hspace{4pt} f(x) \in (L - \epsilon, L + \epsilon)
\end{align*}

\resizebox{30em}{30em}{%
\begin{tikzpicture}[scale=\textwidth/4.2cm]
    % title and axes
    \node at (0.7, 1.1) {\tiny$f(x)=\sin x \hspace{4pt} x \in \Big[0, \dfrac{1}{2}\Big) \cup \Big(\dfrac{1}{2}, 1 \Big]$};
    \draw (0, 0) -- (1, 0)
        node[right] {\tiny$x$};
    \draw (0, 0) -- (0, 1)
        node[above] {\tiny$f(x)$};
    % ----------------------------------
    % range boundaries lower
    \draw[dotted] (0, {sin(0.45 r)}) -- (0.6, {sin(0.45 r)});
    \node[] at (-0.15, {sin(0.45 r)}) {\tiny$L-\epsilon$};
    \node [rotate=90] at (0, {sin(0.45 r)}) {\tiny(};
    % range boundaries upper
    \draw[dotted] (0, {sin(0.55 r)}) -- (0.6, {sin(0.55 r)});
    \node[] at (-0.15, {sin(0.55 r)}) {\tiny$L+\epsilon$};
    \node [rotate=-90] at (0, {sin(0.55 r)}) {\tiny(};
    % ----------------------------------
    % domain boundaries lower
    \draw[dotted] (0.47, 0) -- (0.47, {sin(0.6 r)});
    \node [rotate=45] at (0.35, -0.15) {\tiny$\dfrac{1}{2}-\delta$ $\longrightarrow$};
    \node [] at (0.47, 0) {\tiny(};
    % domain boundaries upper
    \draw[dotted] (0.53, 0) -- (0.53, {sin(0.6 r)});
    \node [rotate=-45] at (0.65, -0.15) {\tiny$\longleftarrow$ \tiny$\dfrac{1}{2}+\delta$};
    \node [] at (0.53, 0) {\tiny)};
    % ----------------------------------
    % graph
    \draw[blue] plot[smooth] file {limits_of_functions/python_generated_tables/sine_0_1_piece_0.table};
    \draw[blue] plot[smooth] file {limits_of_functions/python_generated_tables/sine_0_1_piece_1.table};
    \draw[blue, fill=white] (0.5,{sin(0.5 r)}) circle (.25mm);
\end{tikzpicture}
}
\end{example}

\begin{exercise}
Find the following:
\begin{align*}
    \lim_{x \longrightarrow 1/2} \sin(x) \hspace{20pt} x \in [0, 1]
\end{align*}
\end{exercise}

\begin{theorem}
If $f(x) = x, \hspace{4pt} x \in \mathbb{R}$, then $\lim_{x \longrightarrow c} f(x)$ exists for all $c \in \mathbb{R}$. Because $f$ is defined on the entire real line, we have 
\begin{align*}
    \lim_{x \longrightarrow c} f(x) = f(c) = c
\end{align*}
\label{limit_identity_function}
\end{theorem}

\begin{theorem}
Suppose $c \in \mathbb{R}$ and suppose 
\begin{align*}
    \lim_{x \longrightarrow a} f(x) \hspace{20pt} \text{and} \hspace{20pt} \lim_{x \longrightarrow a} g(x)
\end{align*}
both exist. Then we have
\begin{align*}
    &\lim_{x \longrightarrow a} [f(x) + g(x)] = \lim_{x \longrightarrow a} f(x) + \lim_{x \longrightarrow a} g(x)\\[2ex]
    &\lim_{x \longrightarrow a} [f(x) - g(x)] = \lim_{x \longrightarrow a} f(x) - \lim_{x \longrightarrow a} g(x)\\[2ex]
    &\lim_{x \longrightarrow a} [c \cdot f(x)] = c \cdot \lim_{x \longrightarrow a} f(x)\\[2ex]
    &\lim_{x \longrightarrow a} [f(x) \cdot g(x)] = \lim_{x \longrightarrow a} f(x) \cdot \lim_{x \longrightarrow a} g(x)\\[2ex]
    &\lim_{x \longrightarrow a} \dfrac{f(x)}{g(x)} = \dfrac{\lim_{x \longrightarrow a} f(x)}{\lim_{x \longrightarrow a} g(x)} \hspace{20pt} \text{when} \hspace{4pt} \lim_{x \longrightarrow a} g(x) \neq 0
\end{align*}
\label{properties_limit_functions}
\end{theorem}

\begin{example}
Here is another example, taking the mathematical approach, using Theorems \ref{limit_identity_function}, \ref{properties_limit_functions}, to find the limit,
\begin{align*}
    \lim_{x \longrightarrow 1} \dfrac{x-1}{x^{2} - 1} = \lim_{x \longrightarrow 1} \dfrac{x-1}{(x-1)(x+1)} = \lim_{x \longrightarrow 1} \dfrac{1}{(x+1)} = \dfrac{1}{(1+1)} = \dfrac{1}{2}
\end{align*}

\resizebox{30em}{30em}{%
\begin{tikzpicture}[scale=\textwidth/6.2cm]
    % title and axes
    \node at (1.5, 2.1) {\tiny$f(x)=\dfrac{x-1}{x^{2}-1} \hspace{4pt} x \in \Big[\dfrac{1}{2}, 1\Big) \cup (1, 2]$};
    \draw (-0.5, 0) -- (2, 0)
        node[right] {\tiny$x$};
    \draw (0, 0) -- (0, 2)
        node[above] {\tiny$f(x)$};
    % ----------------------------------
    % range boundaries lower
    \draw[dotted] (0, 0.45) -- (1.05, 0.45);
    \node[rotate=45] at (-0.20, 0.28) {\tiny$\dfrac{1}{2}-\epsilon \longrightarrow$};
    \node [rotate=90] at (0, 0.45) {\tiny(};
    % range boundaries upper
    \draw[dotted] (0, 0.55) -- (1.05, 0.55);
    \node[rotate=-45] at (-0.20, 0.72) {\tiny$\dfrac{1}{2}+\epsilon \longrightarrow$};
    \node [rotate=-90] at (0, 0.55) {\tiny(};
    % ----------------------------------
    % domain boundaries lower
    \draw[dotted] (0.95, 0) -- (0.95, 0.55);
    \node [rotate=45] at (0.75, -0.2) {\tiny$1-\delta$ $\longrightarrow$};
    \node [] at (0.95, 0) {\tiny(};
    % domain boundaries upper
    \draw[dotted] (1.05, 0) -- (1.05, 0.55);
    \node [rotate=-45] at (1.25, -0.2) {\tiny$\longleftarrow$ \tiny$1+\delta$};
    \node [] at (1.05, 0) {\tiny)};
    % ----------------------------------
    % graph
    \draw[blue] plot[smooth] file {limits_of_functions/python_generated_tables/rational_neg1half_2_piece_0.table};
    \draw[blue] plot[smooth] file {limits_of_functions/python_generated_tables/rational_neg1half_2_piece_1.table};
    \draw[blue, fill=white] (1,0.5) circle (.25mm);
\end{tikzpicture}
}
\end{example}

\begin{exercise}
Find the following:
\begin{align*}
    \lim_{x \longrightarrow 1} g(x) \hspace{20pt} \text{when} \hspace{20pt} g(x) = 
    \begin{cases}
    \dfrac{x-1}{x^{2}-1}, \hspace{4pt} &x \neq 1\\[2ex]
    2, \hspace{4pt} &x = 1
    \end{cases}
\end{align*}
\end{exercise}

\begin{exercise} Find the following:
\begin{align*}
    \lim_{h \longrightarrow 0} \dfrac{(3+h)^{2} - 9}{h}
\end{align*}
\end{exercise}

\begin{exercise}
Find the following:
\begin{align*}
    \lim_{x \longrightarrow 2} \dfrac{x^{2}+x-6}{x-2}
\end{align*}
\end{exercise}

\begin{exercise}
Find the following:
\begin{align*}
    \lim_{x \longrightarrow 2} \dfrac{x^{2}-2x}{x^{2}-x-2}
\end{align*}
\end{exercise}

\begin{definition}
We say a function $f$ has a right-sided limit
\begin{align*}
    \lim_{x \longrightarrow c^{+}} f(x) = L
\end{align*}
if for all $\epsilon > 0$, there exists a $\delta > 0$ such that
\begin{align*}
    \lvert x - c \rvert < \delta \hspace{20pt} \Longrightarrow \hspace{20pt} \lvert f(x) - L \rvert < \epsilon
\end{align*}
when $x \in \text{Dom($f$)} \cap (c, \infty)$.
\end{definition}

\begin{definition}
We say a function $f$ has a left-sided limit
\begin{align*}
    \lim_{x \longrightarrow c^{-}} f(x) = L
\end{align*}
if for all $\epsilon > 0$, there exists a $\delta > 0$ such that
\begin{align*}
    \lvert x - c \rvert < \delta \hspace{20pt} \Longrightarrow \hspace{20pt} \lvert f(x) - L \rvert < \epsilon
\end{align*}
when $x \in \text{Dom($f$)} \cap (-\infty, c)$.
\end{definition}

\begin{theorem}
We say function $f$ has a limit
\begin{align*}
    \lim_{x \longrightarrow c} f(x) = L \hspace{20pt} \text{if and only if} \hspace{20pt} \lim_{x \longrightarrow c^{+}} f(x) = L = \lim_{x \longrightarrow c^{-}} f(x)
\end{align*}
\end{theorem}

\begin{example}
Here is an example of a function with a one-sided limit. We call this the floor function, and we observe it on an attenuated domain $[0, 2]$, as opposed to its full domain, $\mathbb{R}$. 
\begin{align*}
    f(x) = \lfloor x \rfloor, \hspace{4pt} x \in [0, 2]
\end{align*}
With this example, we will show the right-sided limit $L^{+} = 1$.
\begin{proof}
Take $\epsilon = \delta$, where $n \in \mathbb{N}$ is arbitrary. We only need there to exist a $\delta > 0$ such that $x \in (1, 1+\delta)$, when $x$ is in the domain of $f$. For any $x$ in the domain of $f$ greater than and close to $1$, we have $x \in \Big(1, 1+\dfrac{1}{k}\Big)$. So, we choose $\delta = \dfrac{1}{k}$. Thus,
\begin{align*}
    \text{when} \hspace{4pt} x \in (1, 1 + \delta) \hspace{20pt} \text{we have} \hspace{20pt} (f(x) - 1) = (\lfloor x \rfloor - 1) = (1 - 1) = 0 < \dfrac{1}{k} = \delta = \epsilon
\end{align*}
\end{proof}

\resizebox{30em}{30em}{%
\begin{tikzpicture}[scale=\textwidth/4.2cm]
    % title and axes
    \node at (1.3, 1.5) {$f(x)=\lfloor x \rfloor \hspace{4pt} x \in [0, 2]$};
    \draw (0, 0) -- (2, 0)
        node[right] {$x$};
    \draw (0, 0) -- (0, 1.3)
        node[above] {$f(x)$};
    % ----------------------------------
    % range boundaries lower
    \draw[dotted] (0, 0.95) -- (1.05, 0.95);
    \node[] at (-0.2, 0.95) {$1-\epsilon$};
    \node [rotate=90] at (0, 0.95) {(};
    % range boundaries upper
    \draw[dotted] (0, 1.05) -- (1.05, 1.05);
    \node[] at (-0.2, 1.05) {$1+\epsilon$};
    \node [rotate=-90] at (0, 1.05) {(};
    % ----------------------------------
    % domain boundaries lower
    \draw[dotted] (0.95, 0) -- (0.95, 1.1);
    \node [rotate=45] at (0.80, -0.18) {$1-\delta$ $\longrightarrow$};
    \node [] at (0.95, 0) {(};
    % domain boundaries upper
    \draw[dotted] (1.05, 0) -- (1.05, 1.1);
    \node [rotate=-45] at (1.21, -0.18) {$\longleftarrow$ $1+\delta$};
    \node [] at (1.05, 0) {)};
    % ----------------------------------
    % graph
    \draw[blue, very thick] plot[smooth] file {limits_of_functions/python_generated_tables/floor_0_2_piece_0.table};
    \draw[blue, very thick] plot[smooth] file {limits_of_functions/python_generated_tables/floor_0_2_piece_1.table};
    \draw[blue, fill=white] (1,0) circle (.25mm);
    \draw[blue, fill=white] (2,1) circle (.25mm);
\end{tikzpicture}
}
\end{example}

\begin{exercise}
Find the following: 
\begin{align*}
    \lim_{x \longrightarrow 1^{-}} f(x) \hspace{20pt} \text{where} \hspace{20pt} f(x) = \lfloor x \rfloor,  \hspace{4pt} x \in [0, 2]
\end{align*}
\end{exercise}

\begin{exercise}
Prove the following:
\begin{align*}
    \lim_{x \longrightarrow n} \lfloor x \rfloor \hspace{8pt} \text{DNE for any integer} \hspace{8pt} n 
\end{align*}
\end{exercise}

\begin{exercise}
Prove the following:
\begin{align*}
    \lim_{x \longrightarrow 0} \dfrac{\lvert x \rvert}{x} \hspace{8pt} \text{DNE}
\end{align*}
\end{exercise}

\begin{exercise}
Determine if the following exists.
\begin{align*}
    \lim_{x \longrightarrow 4} f(x) \hspace{8pt} \text{given} \hspace{8pt} f(x) = 
    \begin{cases}
    \sqrt{x-4}, \hspace{4pt} &x > 4,\\[2ex]
    8-2x, \hspace{4pt} &x < 4
    \end{cases}
\end{align*}
\end{exercise}

\begin{exercise}
Determine if the following exists:
\begin{align*}
\lim_{x \longrightarrow -1} \dfrac{x^{2}-2x}{x^{2}-x-2}
\end{align*}
\end{exercise}

\begin{definition}
We say $f$ has an infinite limit 
\begin{align*}
    \lim_{x \longrightarrow c} f(x) = \infty
\end{align*}
if for all $\alpha \in \mathbb{R}$ we have some $\delta > 0$ such that
\begin{align*}
    \text{for all} \hspace{4pt} x \hspace{4pt} \text{satisfying} \hspace{4pt} \lvert x - c \rvert < \delta \hspace{4pt} \text{we have} \hspace{4pt} f(x) > \alpha
\end{align*}
\end{definition}

\begin{definition}
We say $f$ has an infinite limit 
\begin{align*}
    \lim_{x \longrightarrow c} f(x) = -\infty
\end{align*}
if for all $\alpha \in \mathbb{R}$ we have some $\delta > 0$ such that
\begin{align*}
    \text{for all} \hspace{4pt} x \hspace{4pt} \text{satisfying} \hspace{4pt} \lvert x - c \rvert < \delta \hspace{4pt} \text{we have} \hspace{4pt} f(x) < \alpha
\end{align*}
\end{definition}

\begin{recall}
Vertical asymptotes
\begin{align*}
    \lim_{x \longrightarrow c^{+}} f(x) = \pm\infty \hspace{20pt} \lim_{x \longrightarrow c^{-}} f(x) = \pm\infty 
\end{align*}
are one-sided infinite limits.
\end{recall}

\begin{definition}
We say $f$ has a limit $L$ at $\infty$ 
\begin{align*}
    \lim_{x \longrightarrow \infty} f(x) = L
\end{align*}
if for each $\alpha \in \mathbb{R}$ and all $\epsilon > 0$ 
\begin{align*}
    \text{there exists} \hspace{4pt} K \in \mathbb{N} \hspace{4pt} \text{such that} \hspace{4pt} K > \alpha \hspace{4pt} \text{and for any} \hspace{4pt} x > K \hspace{4pt} \text{we have} \hspace{4pt} \lvert f(x) - L \rvert < \epsilon
\end{align*}
\end{definition}

\begin{definition}
We say $f$ has a limit $L$ at $-\infty$ 
\begin{align*}
    \lim_{x \longrightarrow -\infty} f(x) = L
\end{align*}
if for each $\alpha \in \mathbb{R}$ and all $\epsilon > 0$ 
\begin{align*}
    \text{there exists} \hspace{4pt} K \in \mathbb{N} \hspace{4pt} \text{such that} \hspace{4pt} -K < \alpha \hspace{4pt} \text{and for any} \hspace{4pt} x < -K \hspace{4pt} \text{we have} \hspace{4pt} \lvert f(x) - L \rvert < \epsilon
\end{align*}
\end{definition}

\begin{exercise}
Find the following:
\begin{align*}
    \lim_{x \longrightarrow -3^{+}} \dfrac{x+2}{x+3}
\end{align*}
\end{exercise}

\begin{exercise}
Find the following:
\begin{align*}
    \lim_{x \longrightarrow -3^{-}} \dfrac{x+2}{x+3}
\end{align*}
\end{exercise}

\begin{exercise}
Find the following:
\begin{align*}
    \lim_{x \longrightarrow 1} \dfrac{2-x}{(x-1)^{2}}
\end{align*}
\end{exercise}

\begin{exercise}
Find the following:
\begin{align*}
    \lim_{x \longrightarrow 3^{+}} \ln(x^{2} - 9)
\end{align*}
\end{exercise}

\begin{exercise}
Find the following:
\begin{align*}
    \lim_{x \longrightarrow 5^{-}} \dfrac{e^{x}}{(x-5)^{3}}
\end{align*}
\end{exercise}