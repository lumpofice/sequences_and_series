\section{Limits of Functions}

\begin{definition}
We say $f$ as a function of $x$ has a limit $L$ at $c \in \mathbb{R}$, denoted by
\begin{align*}
    &\lim_{x \longrightarrow c} f(x) = L \hspace{4pt} \text{if} \hspace{20pt} &&\text{for all} \hspace{4pt} \epsilon > 0, \hspace{4pt} \text{there exists a} \hspace{4pt} \delta > 0 \hspace{4pt} \text{such that}\\[2ex]
    &\text{for all} \hspace{4pt} x \in \text{Dom($f$)} \hspace{4pt} \text{satisfying} \hspace{4pt} \lvert x - c \rvert < \delta &&\text{we have} \hspace{4pt} \lvert f(x) - L \rvert < \epsilon
\end{align*}
\end{definition}

\begin{example}
Below is a visual of how this $\epsilon - \delta$ interplay occurs. In general, if $L \in \mathbb{R}$ is the limit of $f$ at $c \in \mathbb{R}$, then for an arbitrarily small, open interval $(L-\epsilon, L+\epsilon)$ with center $L$, there is an open interval $(c-\delta, c+\delta)$ with center $c$ containing members $x \in$ Dom($f$). With this specific example, $c=1/2$. For any $x \in$ Dom($f$) satisfying 
\begin{align*}
    \Big\lvert x - \dfrac{1}{2} \Big\rvert < \delta \hspace{20pt} \text{we have} \hspace{4pt} \lvert f(x) - L \rvert = \lvert \sin(x) - L \rvert < \epsilon
\end{align*}
which is equivalent to the following: For any $x \in$ Dom($f$) satisfying
\begin{align*}
    x \in \Big(\dfrac{1}{2} - \delta, \dfrac{1}{2} + \delta\Big) \hspace{20pt} \text{we have} \hspace{4pt} f(x) \in (L - \epsilon, L + \epsilon)
\end{align*}

\resizebox{30em}{30em}{%
\begin{tikzpicture}[scale=\textwidth/4.2cm]
    % title and axes
    \node at (0.7, 1.1) {\tiny$f(x)=\sin x \hspace{4pt} x \in \Big[0, \dfrac{1}{2}\Big) \cup \Big(\dfrac{1}{2}, 1 \Big]$};
    \draw (0, 0) -- (1, 0)
        node[right] {\tiny$x$};
    \draw (0, 0) -- (0, 1)
        node[above] {\tiny$f(x)$};
    % ----------------------------------
    % range boundaries lower
    \draw[dotted] (0, {sin(0.45 r)}) -- (0.6, {sin(0.45 r)});
    \node[] at (-0.15, {sin(0.45 r)}) {\tiny$L-\epsilon$};
    \node [rotate=90] at (0, {sin(0.45 r)}) {\tiny(};
    % range boundaries upper
    \draw[dotted] (0, {sin(0.55 r)}) -- (0.6, {sin(0.55 r)});
    \node[] at (-0.15, {sin(0.55 r)}) {\tiny$L+\epsilon$};
    \node [rotate=-90] at (0, {sin(0.55 r)}) {\tiny(};
    % ----------------------------------
    % domain boundaries lower
    \draw[dotted] (0.47, 0) -- (0.47, {sin(0.6 r)});
    \node [rotate=45] at (0.35, -0.15) {\tiny$\dfrac{1}{2}-\delta$ $\longrightarrow$};
    \node [] at (0.47, 0) {\tiny(};
    % domain boundaries upper
    \draw[dotted] (0.53, 0) -- (0.53, {sin(0.6 r)});
    \node [rotate=-45] at (0.65, -0.15) {\tiny$\longleftarrow$ \tiny$\dfrac{1}{2}+\delta$};
    \node [] at (0.53, 0) {\tiny)};
    % ----------------------------------
    % graph
    \draw[blue] plot[smooth] file {limits_of_functions/sine_0_1_piece_0.table};
    \draw[blue] plot[smooth] file {limits_of_functions/sine_0_1_piece_1.table};
    \draw[blue, fill=white] (0.5,{sin(0.5 r)}) circle (.25mm);
\end{tikzpicture}
}
\end{example}

\begin{example}
As a purely, mathematically symbolic example,
\begin{align*}
    \lim_{x \longrightarrow 1} \dfrac{x-1}{x^{2} - 1} = \lim_{x \longrightarrow 1} \dfrac{x-1}{(x-1)(x+1)} = \lim_{x \longrightarrow 1} \dfrac{1}{(x+1)} = \dfrac{1}{(1+1)} = \dfrac{1}{2}
\end{align*}
\end{example}

\begin{example}
Find the following:
\end{example}

\begin{example}
Here is an example of a function with a one-sided limit. We call this the floor function, and we observe it on an attenuated domain $[0, 2]$, as opposed to its full domain, $\mathbb{R}$. 
\begin{align*}
    f(x) = \lfloor x \rfloor, \hspace{4pt} x \in [0, 2]
\end{align*}
With this example, we will show the right-sided limit $L^{+} = 1$.
\begin{proof}
Take $\epsilon = \delta$, where $n \in \mathbb{N}$ is arbitrary. We only need there to exist a $\delta > 0$ such that $x \in (1, 1+\delta)$, when $x$ is in the domain of $f$. For any $x$ in the domain of $f$ greater than and close to $1$, we have $x \in \Big(1, 1+\dfrac{1}{k}\Big)$. So, we choose $\delta = \dfrac{1}{k}$. Thus,
\begin{align*}
    \text{when} \hspace{4pt} x \in (1, 1 + \delta) \hspace{20pt} \text{we have} \hspace{20pt} (f(x) - 1) = (\lfloor x \rfloor - 1) = (1 - 1) = 0 < \dfrac{1}{k} = \delta = \epsilon
\end{align*}
\end{proof}

\resizebox{30em}{30em}{%
\begin{tikzpicture}[scale=\textwidth/4.2cm]
    % title and axes
    \node at (1.3, 1.5) {$f(x)=\lfloor x \rfloor \hspace{4pt} x \in [0, 2]$};
    \draw (0, 0) -- (2, 0)
        node[right] {$x$};
    \draw (0, 0) -- (0, 1.3)
        node[above] {$f(x)$};
    % ----------------------------------
    % range boundaries lower
    \draw[dotted] (0, 0.95) -- (1.05, 0.95);
    \node[] at (-0.2, 0.95) {$L-\epsilon$};
    \node [rotate=90] at (0, 0.95) {(};
    % range boundaries upper
    \draw[dotted] (0, 1.05) -- (1.05, 1.05);
    \node[] at (-0.2, 1.05) {$L+\epsilon$};
    \node [rotate=-90] at (0, 1.05) {(};
    % ----------------------------------
    % domain boundaries lower
    \draw[dotted] (0.95, 0) -- (0.95, 1.1);
    \node [rotate=45] at (0.80, -0.18) {$1-\delta$ $\longrightarrow$};
    \node [] at (0.95, 0) {(};
    % domain boundaries upper
    \draw[dotted] (1.05, 0) -- (1.05, 1.1);
    \node [rotate=-45] at (1.21, -0.18) {$\longleftarrow$ $1+\delta$};
    \node [] at (1.05, 0) {)};
    % ----------------------------------
    % graph
    \draw[blue, very thick] plot[smooth] file {limits_of_functions/floor_0_2_piece_0.table};
    \draw[blue, very thick] plot[smooth] file {limits_of_functions/floor_0_2_piece_1.table};
    \draw[blue, fill=white] (1,0) circle (.25mm);
    \draw[blue, fill=white] (2,1) circle (.25mm);
\end{tikzpicture}
}
\end{example}

\begin{exercise}
Find the following: 
\begin{align*}
    \lim_{x \longrightarrow 1^{-}} f(x) \hspace{20pt} \text{where} \hspace{20pt} f(x) = \lfloor x \rfloor,  \hspace{4pt} x \in [0, 2]
\end{align*}
\end{exercise}