\section{Series of Numbers}

\begin{definition}
Take the sequence $\{f(n)\}_{n=1}^{\infty}$ and sum all of its terms
\begin{align*}
    f(1) + f(2) + \cdots = S \hspace{20pt} S \in \mathbb{R} \hspace{4pt} \cup \hspace{4pt} \{-\infty, \infty\}
\end{align*}
We refer to $S$ as a series, and we denote this series as 
\begin{align*}
    S = \sum_{n=1}^{\infty} f(n)
\end{align*}
If $S$ is finite, then the series is convergent. If $S$ is infinite, then the series is divergent. There are some series that begin with $n=0$ as opposed to $n=1$. We will discover that the important results covered in this section focus less on where we begin in a series and more on where we end.
\end{definition}
Geometric\\
Integral Test\\
Comparison Tests\\
Alternating Series\\
Ratio, Root Tests\\

