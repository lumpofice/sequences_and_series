\section{The Derivative}

\begin{definition}
We say a function $f$ has derivative $L$ at $c \in$ Dom($f$) if 
\begin{align*}
    &\text{For all} \hspace{4pt} \epsilon > 0 \hspace{4pt} \text{there exists a} \hspace{4pt} \delta > 0 \hspace{4pt} \text{such that}\\[2ex]
    &\text{all} \hspace{4pt} x \in \text{Dom($f$)} \hspace{4pt} \text{satisfying} \hspace{4pt} \lvert x-c \rvert < \delta \hspace{20pt} \Longrightarrow \hspace{20pt} \Big\lvert \dfrac{f(x)-f(c)}{x-c}-L \Big\rvert < \epsilon
\end{align*}
We denote $L$ as
\begin{align*}
f^{'}(c) = \lim_{x \longrightarrow c} \dfrac{f(x)-f(c)}{x-c}
\end{align*}
which can be written equivalently as
\begin{align*}
    f^{'}(c) = \lim_{h \longrightarrow 0} \dfrac{f(c+h)-f(c)}{h}
\end{align*}
\end{definition}

\begin{example}
The derivative is strongly associated with the concept of a slope, whether it be the slope of a $1$D line in $2$D space or the slope of a $2$D plane in $3$D space or the slope of a $(n-1)$D plane in $n$D space. Below, we show a subset of the secant lines (in black) passing through the point $(\pi/4, \sqrt{2}/2)$ that can be formed as we push towards the derivative (tangent line in red) passing through the point $(\pi/4, \sqrt{2}/2)$.
\begin{align*}
    \lim_{h \longrightarrow 0} \dfrac{\sin\Big(\dfrac{\pi}{4} + h\Big) - \sin\Big(\dfrac{\pi}{4}\Big)}{h} 
\end{align*}

\resizebox{30em}{30em}{%
\begin{tikzpicture}[scale=\textwidth/4.2cm]
    % title and axes
    \node at (0.6, 1.2) {$f(x)=\sin x \hspace{4pt} x \in [0, \pi]$};
    \draw (0, 0) -- (pi, 0)
        node[right] {$x$};
    \draw (0, 0) -- (0, 2.4)
        node[above] {$f(x)$};
    % graph
    \draw[red, very thick] plot[smooth] file {derivatives_of_functions/python_generated_tables/sin_0_pi_tangent.table};
    \draw[black] plot[smooth] file {derivatives_of_functions/python_generated_tables/sin_0_pi_line_5.table};
    \draw[black] plot[smooth] file {derivatives_of_functions/python_generated_tables/sin_0_pi_line_6.table};
    \draw[blue, very thick] plot[smooth] file {derivatives_of_functions/python_generated_tables/sin_0_pi.table};
    \draw[black] plot[smooth] file {derivatives_of_functions/python_generated_tables/sin_0_pi_line_1.table};
    \draw[black] plot[smooth] file {derivatives_of_functions/python_generated_tables/sin_0_pi_line_2.table};
    \draw[black] plot[smooth] file {derivatives_of_functions/python_generated_tables/sin_0_pi_line_3.table};
    \draw[black] plot[smooth] file {derivatives_of_functions/python_generated_tables/sin_0_pi_line_4.table};
    \draw[blue, fill=red] (pi/4, {sqrt(2)/2}) circle (.25mm);
    \draw[black, fill=black] (2*pi/3, {sqrt(3)/2}) circle (.25mm);
    \draw[black, fill=black] (pi/2, 1) circle (.25mm);
    \node at (pi/4, -0.1) {$\pi/4$};
    \node at (2*pi/3, -0.1) {$2\pi/3$};
    \node at (pi/2, -0.1) {$\pi/2$};
    \draw[dotted] (pi/4, 0) -- (pi/4, {sqrt(2)/2});
    \draw[dotted] (pi/2, 0) -- (pi/2, 1);
    \draw[dotted] (2*pi/3, 0) -- (2*pi/3, {sqrt(3)/2});
\end{tikzpicture}
}
\end{example}