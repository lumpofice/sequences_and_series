\section{The Derivative}\label{the_derivative}

\begin{definition}
We say a function $f$ has derivative $L \in \mathbb{R}$ at $c \in$ Dom($f$) if 
\begin{align*}
    &\text{For all} \hspace{4pt} \epsilon > 0 \hspace{4pt} \text{there exists a} \hspace{4pt} \delta > 0 \hspace{4pt} \text{such that}\\[2ex]
    &\text{all} \hspace{4pt} x \in \text{Dom($f$)} \hspace{4pt} \text{satisfying} \hspace{4pt} \lvert x-c \rvert < \delta \hspace{20pt} \Longrightarrow \hspace{20pt} \Big\lvert \dfrac{f(x)-f(c)}{x-c}-L \Big\rvert < \epsilon
\end{align*}
We denote $L$ as
\begin{align*}
f^{'}(c) = \lim_{x \longrightarrow c} \dfrac{f(x)-f(c)}{x-c}
\end{align*}
where $f^{'}$ is read '$f$ prime', which we refer to as prime notation. The above can be written equivalently as
\begin{align*}
    f^{'}(c) = \lim_{h \longrightarrow 0} \dfrac{f(c+h)-f(c)}{h}
\end{align*}
\end{definition}

\begin{example}
The derivative of a function $f$ at a point $c \in$ Dom($f$) can be thought of as the slope of the linear function $T(x)$ tangent to function $f$ at point $c$. The slope of a $1$-D line in $2$-D space is what we will deal with most in this course, but the slope of a $2$-D plane in $3$-D space or the slope of a $(n-1)$-D plane in $n$-D space can also be a derivative of some function in these respective spaces. Below, we show a subset of the secant lines (in black) passing through the point $(\pi/4, \sqrt{2}/2)$, all of which can be formed as we push towards the derivative (tangent line in red) passing through the point $(\pi/4, \sqrt{2}/2)$.
\begin{align*}
    &\text{When} \hspace{4pt} f(x) = \sin x\\[2ex]
    &\text{and when} \hspace{4pt} f^{'}\Big(\dfrac{\pi}{4}\Big) = \lim_{h \longrightarrow 0} \dfrac{\sin\Big(\dfrac{\pi}{4} + h\Big) - \sin\Big(\dfrac{\pi}{4}\Big)}{h}\\[2ex]
    &\text{then we have the tangent line} \hspace{4pt} T(x) = (f^{'}(x))x + b \hspace{20pt}\\[2ex]
    &\text{where $b$ is the intercept of the linear function $T$}
\end{align*}

\resizebox{30em}{30em}{%
\begin{tikzpicture}[scale=\textwidth/4.2cm]
    % title and axes
    \node at (0.6, 2.2) {$f(x)=\sin x, \hspace{4pt} x \in [0, \pi]$};
    \draw (0, 0) -- (pi, 0)
        node[right] {$x$};
    \draw (0, 0) -- (0, 2.4)
        node[above] {$f(x)$};
    % graph
    \draw[red, very thick] plot[smooth] file {derivatives_of_functions/python_generated_tables/sin_0_pi_tangent.table};
    \draw[black] plot[smooth] file {derivatives_of_functions/python_generated_tables/sin_0_pi_line_5.table};
    \draw[black] plot[smooth] file {derivatives_of_functions/python_generated_tables/sin_0_pi_line_6.table};
    \draw[blue, very thick] plot[smooth] file {derivatives_of_functions/python_generated_tables/sin_0_pi.table};
    \draw[black] plot[smooth] file {derivatives_of_functions/python_generated_tables/sin_0_pi_line_1.table};
    \draw[black] plot[smooth] file {derivatives_of_functions/python_generated_tables/sin_0_pi_line_2.table};
    \draw[black] plot[smooth] file {derivatives_of_functions/python_generated_tables/sin_0_pi_line_3.table};
    \draw[black] plot[smooth] file {derivatives_of_functions/python_generated_tables/sin_0_pi_line_4.table};
    \draw[blue, fill=red] (pi/4, {sqrt(2)/2}) circle (.25mm);
    \draw[black, fill=black] (2*pi/3, {sqrt(3)/2}) circle (.25mm);
    \draw[black, fill=black] (pi/2, 1) circle (.25mm);
    \node at (pi/4, -0.1) {$\pi/4$};
    \node at (2*pi/3, -0.1) {$2\pi/3$};
    \node at (pi/2, -0.1) {$\pi/2$};
    \draw[dotted] (pi/4, 0) -- (pi/4, {sqrt(2)/2});
    \draw[dotted] (pi/2, 0) -- (pi/2, 1);
    \draw[dotted] (2*pi/3, 0) -- (2*pi/3, {sqrt(3)/2});
\end{tikzpicture}
}
\label{sin_0_pi_tangent}
\end{example}

\begin{exercise}
Use the definition of the derivative to find the derivative $f^{'}(x)$ for
\begin{align*}
    f(x) = x^{3} \hspace{20pt} x \in \mathbb{R}
\end{align*}
\end{exercise}

\begin{exercise}
Use the definition of the derivative to find the derivative $f^{'}(x)$ for
\begin{align*}
    f(x) = \dfrac{1}{x} \hspace{20pt} x \in \mathbb{R}, \hspace{4pt} x \neq 0
\end{align*}
\end{exercise}

\begin{exercise}
Use the definition of the derivative to find the derivative $f^{'}(x)$ for
\begin{align*}
    f(x) = \sqrt{x} \hspace{20pt} x > 0
\end{align*}
\end{exercise}

\begin{exercise}
Use the definition of the derivative to find the derivative $f^{'}(x)$ for
\begin{align*}
    f(x) = \dfrac{1}{\sqrt{x}} \hspace{20pt} x > 0
\end{align*}
\end{exercise}

\begin{exercise}
Use the definition of the derivative to find the derivative $f^{'}(x)$ for
\begin{align*}
    f(x) = x^{3} - x
\end{align*}
\end{exercise}

\begin{exercise}
Use the definition of the derivative to find the derivative $f^{'}(x)$ for
\begin{align*}
    f(x) = \dfrac{1-x}{2+x}
\end{align*}
\end{exercise}

\begin{exercise}
Use the definition of the derivative to find the derivative $f^{'}(x)$ for
\begin{align*}
    f(x) = \lvert x \rvert
\end{align*}
\end{exercise}

\newpage
\section{Derivatives of Functions and Properties of Derivatives}

\begin{note}
In addition to the prime notation, we will use $\dfrac{d}{dx}$ to represent the derivative operator. The derivative operator acts on a function $f$ to generate the derivative $f^{'}$ like so
\begin{align*}
    \dfrac{d}{dx}f(x) = f^{'}(x)
\end{align*}
\end{note}

\begin{theorem}
If $f$ has a derivative at $c \in$ Dom($f$), then $f$ is continuous at $c$.
\end{theorem}

\begin{exercise}
TRUE or FALSE: If $f$ does not have a derivative at $c \in$ Dom($f$) (we say $f$ is not differentiable at $c$ when $f$ does not have a derivative at $c$) then $f$ is discontinuous at $c$.
\end{exercise}

\begin{exercise}
TRUE or FALSE: If $f$ is not continuous at $c \in$ Dom($f$)  then $f$ is not differentiable at $c$.
\end{exercise}

\begin{exercise}
TRUE or FALSE: If $f$ does not have a derivative at $c$ then $f$ does not have a limit at $c$.
\end{exercise}

\begin{theorem}
Derivative of constant $a$:
\begin{align*}
    \dfrac{d}{dx}a = 0 \hspace{20pt} a \in \mathbb{R}
\end{align*}
\end{theorem}

\begin{theorem}
Derivative of $x^{a}$:
\begin{align*}
    \dfrac{d}{dx}x^{a} = ax^{a-1} \hspace{20pt} a \in \mathbb{R}
\end{align*}
\end{theorem}

\begin{theorem}
Let $f$ and $g$ be functions differentiable at $c \in$ Dom($f$), Dom($g$). Then
\begin{align*}
    \text{For $a \in \mathbb{R}$} \hspace{20pt} &(af)^{'}(c) = af^{'}(c)\\[2ex]
    &(f+g)^{'}(c) = f^{'}(c) + g^{'}(c)\\[2ex]
    &(f-g)^{'}(c) = f^{'}(c) - g^{'}(c)\\[2ex]
    &(fg)^{'}(c) = f^{'}(c)g(c) + f(c)g^{'}(c) \hspace{20pt} \text{(Product Rule)}\\[2ex]
    \text{Assuming $g(c) \neq 0$} \hspace{20pt} &\Big(\dfrac{f}{g}\Big)^{'}(c) = \dfrac{f^{'}(c)g(c) - f(c)g^{'}(c)}{(g(c))^{2}} \hspace{20pt} \text{(Quotient Rule)}
\end{align*}
\end{theorem}

\begin{theorem}
Derivative of $e^{x}$:
\begin{align*}
    \dfrac{d}{dx}e^{x} = e^{x}
\end{align*}
\end{theorem}

\begin{theorem}
Derivatives of six of the trigonometric functions:
\begin{align*}
    &\dfrac{d}{dx}\sin x = \cos x\\[2ex]
    &\dfrac{d}{dx}\cos x = \sin x\\[2ex]
    &\dfrac{d}{dx}\tan x = \sec^{2} x\\[2ex]
    &\dfrac{d}{dx}\csc x = -\csc x \cot x\\[2ex]
    &\dfrac{d}{dx}\sec x = \sec x \tan x\\[2ex]
    &\dfrac{d}{dx}\cot x = -\csc^{2} x
\end{align*}
\end{theorem}

\begin{exercise}
Find the derivative
\begin{align*}
    f(x) = 2 - \dfrac{2}{3}x
\end{align*}
\end{exercise}

\begin{exercise}
Find the derivative 
\begin{align*}
    f(x) = \dfrac{1}{2}x^{6} - 3x^{4} + x
\end{align*}
\end{exercise}

\begin{exercise}
Find the derivative
\begin{align*}
    f(x) = (x-2)(2x+3)
\end{align*}
\end{exercise}

\begin{exercise}
Find the derivative
\begin{align*}
    f(x) = \dfrac{x^{2} + 4x + 3}{\sqrt{x}}
\end{align*}
\end{exercise}

\begin{exercise}
Find the derivative
\begin{align*}
    f(x) = \dfrac{A}{x^{10}} + Be^{x}
\end{align*}
\end{exercise}

\begin{exercise}
Find the derivative
\begin{align*}
    f(x) = e^{x+1} + 1
\end{align*}
\end{exercise}

\begin{example}
We revisit the Example \ref{sin_0_pi_tangent} from the previous Section \ref{the_derivative}. We have learned that
\begin{align*}
    \dfrac{d}{dx} \sin x = \cos x
\end{align*}
and we know that at $x = \dfrac{\pi}{4}$ we have
\begin{align*}
    \cos \Big(\dfrac{\pi}{4} \Big) = \dfrac{\sqrt{2}}{2}
\end{align*}
This is the slope $f^{'}\Big(\dfrac{\pi}{4}\Big)$ of our equation for $T$. All that remains now is to find the intercept. 
\begin{align*}
    &\text{With} \hspace{4pt} T(x) = \Big(f^{'}\Big(\dfrac{\pi}{4}\Big)\Big)x + b = \Big(\dfrac{\sqrt{2}}{2}\Big)x + b\\[2ex]
    &\text{we can use the point} \hspace{4pt} \Big(\dfrac{\pi}{4}, \dfrac{\sqrt{2}}{2}\Big) \hspace{4pt} \text{to get}\\[2ex] &\dfrac{\sqrt{2}}{2} = T\Big(\dfrac{\pi}{4}\Big) = \Big(\dfrac{\sqrt{2}}{2}\Big)\Big(\dfrac{\pi}{4}\Big) + b \hspace{20pt} \Longleftrightarrow \hspace{20pt} b = \dfrac{\sqrt{2}}{2} - \Big(\dfrac{\sqrt{2}}{2}\Big) \Big(\dfrac{\pi}{4}\Big)
\end{align*}

\resizebox{30em}{30em}{%
\begin{tikzpicture}[scale=\textwidth/4.2cm]
    % title and axes
    \node at (2.1, 0.4) {$f(x)=\sin x, \hspace{4pt} x \in [0, \pi] \longrightarrow$};
    \node at (1.8, 2.2) {$T(x)=\Big(\dfrac{\sqrt{2}}{2} \Big) x + \dfrac{\sqrt{2}}{2} \Big(1 - \dfrac{\pi}{4}\Big), \hspace{4pt} x \in [0, \pi] \longrightarrow$};
    \draw (0, 0) -- (pi, 0)
        node[right] {$x$};
    \draw (0, 0) -- (0, 2.4)
        node[above] {$f(x)$};
    % graph
    \draw[red, very thick] plot[smooth] file {derivatives_of_functions/python_generated_tables/sin_0_pi_tangent.table};
    \draw[blue, very thick] plot[smooth] file {derivatives_of_functions/python_generated_tables/sin_0_pi.table};
    \draw[blue, fill=red] (pi/4, {sqrt(2)/2}) circle (.25mm);
    \node at (pi/4, -0.1) {$\pi/4$};
    \node at (-0.2, {sqrt(2)/2}) {$\sqrt{2}/2$};
\end{tikzpicture}
}
\end{example}

\begin{exercise}
Find an equation of the tangent line to the function $f$ at the point $(1,1)$
\begin{align*}
    f(x) = \sqrt[\leftroot{2}\uproot{2}]{x}
\end{align*}
\end{exercise}

\begin{exercise}
Find an equation of the tangent line to the function $f$ at the point $(1,2)$
\begin{align*}
    f(x) = x^{4} + 2x^{2} - x
\end{align*}
\end{exercise}

\begin{exercise}
Find the point(s) along $f(x) = 2x^{3} + 3x^{2} -12x + 1$ where the tangent line has a slope of zero. 
\end{exercise}

\begin{exercise}
Show that the function $f(x) = 6x^{3} + 5x - 3$ has no tangent line with a slope of $4$.
\end{exercise}

\begin{exercise}
Find an equation of the line tangent to the function $f(x) = x\sqrt{x}$ that is parallel to the line $f(x) = 1 + 3x$.
\end{exercise}

\begin{exercise}
Is $f$ differentiable at $x=1$? Sketch $f$ and $f^{'}$.
\begin{align*}
    f(x) = \begin{cases}
    2-x, &x\leq 1\\[2ex]
    x^{2} - 2x + 2, &x> 1
    \end{cases}
\end{align*}
\end{exercise}

\begin{exercise}
Find the values of $m$ and $b$ for which $f$ is differentiable everywhere.
\begin{align*}
    f(x) = \begin{cases}
    x^{2}, &x\leq 2\\[2ex]
    mx + b, &x>2
    \end{cases}
\end{align*}
\end{exercise}

\begin{exercise}
Find the derivative
\begin{align*}
    f(x) = (x^{-2} + x^{-3})(x^{5} - 2x^{2})
\end{align*}
\end{exercise}

\begin{exercise}
Find the derivative
\begin{align*}
    f(x) = \dfrac{x+1}{x^{3}+x-2}
\end{align*}
\end{exercise}

\begin{exercise}
Find the derivative
\begin{align*}
    f(x) = \dfrac{x}{(x-1)^{2}}
\end{align*}
\end{exercise}

\begin{exercise}
Find an equation of the line tangent to $f$ at the point $(1, 1)$
\begin{align*}
    f(x) = \dfrac{2x}{x+1}
\end{align*}
\end{exercise}

\begin{exercise}
Find an equation of the line tangent to $f$ at the point $(1, e)$
\begin{align*}
    f(x) = \dfrac{e^{x}}{x}
\end{align*}
\end{exercise}

\begin{exercise}
How many tangent lines to $f$ pass through the point $(1, 2)$? At which points do these lines touch $f$?
\begin{align*}
    f(x) = \dfrac{x}{x+1}
\end{align*}
\end{exercise}

\begin{exercise}
Find equations of the tangent lines to $f$ that are parallel to the line $T$
\begin{align*}
    f(x) = \dfrac{x-1}{x+1} \hspace{20pt} T(x) = \dfrac{x-2}{2}
\end{align*}
\end{exercise}

\begin{exercise}
Use the quotient rule to prove
\begin{align*}
    \dfrac{d}{dx} \csc x = -\csc x \cot x
\end{align*}
\end{exercise}

\begin{exercise}
Use the quotient rule to prove
\begin{align*}
    \dfrac{d}{dx} \sec x = \sec x \tan x
\end{align*}
\end{exercise}

\begin{exercise}
Use the quotient rule to prove
\begin{align*}
    \dfrac{d}{dx} \cot x = -\csc^{2} x
\end{align*}
\end{exercise}

\begin{exercise}
Use the quotient rule to prove
\begin{align*}
    \dfrac{d}{dx} \tan x = \sec^{2} x
\end{align*}
\end{exercise}

\begin{exercise}
Use the definition of the derivative to prove
\begin{align*}
    \dfrac{d}{dx} \cos x = -\sin x
\end{align*}
\end{exercise}

\begin{exercise}
Use the definition of the derivative to prove
\begin{align*}
    \dfrac{d}{dx} \sin x = \cos x
\end{align*}
\end{exercise}

\begin{exercise}
Find the derivative
\begin{align*}
    f(x) = xe^{x}\csc x
\end{align*}
\end{exercise}

\begin{exercise}
Find the derivative
\begin{align*}
    f(x) = x^{2}\sin x \tan x
\end{align*}
\end{exercise}

\begin{exercise}
Find an equation of the line tangent to $f$ at $\Big(\dfrac{\pi}{3}, 2\Big)$
\begin{align*}
    f(x) = \sec x
\end{align*}
\end{exercise}

\begin{exercise}
Find an equation of the line tangent to $f$ at $(0, 1)$
\begin{align*}
    f(x) = e^{x}\cos x 
\end{align*}
\end{exercise}

\begin{exercise}
Find an equation of the line tangent to $f$ at $(0, 1)$
\begin{align*}
    f(x) = x + \cos x
\end{align*}
\end{exercise}

\begin{exercise}
Find an equation of the line tangent to $f$ at $(0, 1)$
\begin{align*}
    f(x) = \dfrac{1}{\sin x + \cos x}
\end{align*}
\end{exercise}

\newpage
\section{The Chain Rule}

\begin{theorem}
Let $f$ on Dom($f$) and $g$ on Dom($g$) be functions, where Dom($f$) and Dom($g$) are intervals and Dom($f$) is a subset of Dom($g$). 
\begin{align*}
    \text{If} \hspace{4pt} f \hspace{4pt} \text{is differentiable at} \hspace{4pt} c \in \hspace{4pt} &\text{Dom($f$) and if} \hspace{4pt} g \hspace{4pt} \text{is differentiable at} \hspace{4pt} f(c) \in \hspace{4pt} \text{Dom($g$) then}\\[2ex]
    &g \circ f \hspace{4pt} \text{is differentiable at} \hspace{4pt} c\\[2ex]
    \text{and} \hspace{4pt} &(g\circ f)^{'}(c) = g^{'}(f(c))\cdot f^{'}(c)
\end{align*}
This is known as The Chain Rule
\end{theorem}

\begin{example}
Let $g\circ f(x) = \sqrt{x^{2}}$, with $g(f) = \sqrt{f}$ and $f(x) = x^{2}$. Then
\begin{align*}
    &\dfrac{d}{dx}g\circ f(x) = \dfrac{d(g)}{df}\dfrac{df}{dx} \hspace{20pt} \text{giving us}\\[2ex]
    \dfrac{d}{dx}g\circ f(x) = \dfrac{d}{dx}\sqrt{x^{2}} = \dfrac{d}{dx}(x^{2})^{1/2} &= \dfrac{1}{2}(x^{2})^{-1/2}\dfrac{d}{dx}x^{2} = \dfrac{1}{2x^{1/2}}\dfrac{d}{dx}x^{2} = \dfrac{2x}{2\sqrt{x}} = \dfrac{x}{\sqrt{x}} = \sqrt{x}
\end{align*}
\end{example}

\begin{exercise}
Take the derivative of $g\circ f(x) = x^{2}$,  identifying the functions $g$ and $f$ in your response. 
\end{exercise}

\begin{exercise}
Differentiate the following
\begin{align*}
    f(x) = \sin 4x
\end{align*}
\end{exercise}

\begin{exercise}
Differentiate the following
\begin{align*}
    f(x) = \tan (\sin x)
\end{align*}
\end{exercise}

\begin{exercise}
Differentiate the following
\begin{align*}
    f(x) = 3\cot (nx)
\end{align*}
\end{exercise}

\begin{theorem}
When $f(x) = a^{x}$, where $a > 0$
\begin{align*}
    f^{'}(x) = (\ln a)a^{x}
\end{align*}
We show this with the chain rule. Let's change the variables for this function. We define
\begin{align*}
    &g\circ x(y) = a^{y} = e^{\ln a^{y}} = e^{(\ln a)y}, \hspace{20pt} \text{where} \hspace{20pt} g(x) = a^{x}, \hspace{4pt} x(y) = (\ln a)y\\[2ex]
    &\text{By The Chain Rule} \hspace{4pt} \dfrac{d}{dy} a^{y} = \dfrac{d}{dy} e^{(\ln a)y} = \dfrac{d}{dy} e^{x} = \dfrac{d(e^{x})}{dx} \dfrac{dx}{dy} = e^{x}\dfrac{dx}{dy} = e^{x} \dfrac{d}{dy} (\ln a)y = e^{x} \ln a
\end{align*}
Since $x = x(y) = (\ln a)y$, and since $e^{\ln b} = b$ for any $b$
\begin{align*}
    e^{x} \ln a = e^{x(y)} \ln a = e^{(\ln a)y} \ln a = e^{\ln a^{y}} \ln a = a^{y} \ln a
\end{align*}
\end{theorem}

\begin{example}
We already know, for $f(x) = e^{x}$
\begin{align*}
    f^{'}(x) = e^{x}
\end{align*}
We define
\begin{align*}
    g \circ f(x) = e^{x} = e^{(\ln e)x}, \hspace{20pt} \text{where} \hspace{20pt} g(f) = e^{f}, \hspace{4pt} f(x) = (\ln e)x
\end{align*}
Then 
\begin{align*}
    \dfrac{d(g)}{df} \dfrac{df}{dx} = e^{(\ln e)x}\ln e = e^{x}
\end{align*}
\end{example}

\begin{exercise}
Differentiate the following
\begin{align*}
    f(x) = e^{\sin x}
\end{align*}
\end{exercise}

\begin{exercise}
Differentiate the following
\begin{align*}
    f(x) = 2^{\sin \pi x}
\end{align*}
\end{exercise}

\begin{exercise}
Differentiate the following
\begin{align*}
    f(x) = 10^{1-x^2}
\end{align*}
\end{exercise}

\begin{exercise}
Differentiate the following
\begin{align*}
    f(x) = \cos\Big(\dfrac{1-e^{2x}}{1+e^{2x}}\Big)
\end{align*}
\end{exercise}

\begin{exercise}
Differentiate the following
\begin{align*}
    f(x) = e^{k \tan \sqrt{x}}
\end{align*}
\end{exercise}

\begin{exercise}
Differentiate the following
\begin{align*}
    f(x) = \tan (e^{x}) + e^{\tan x}
\end{align*}
\end{exercise}

\begin{exercise}
Differentiate the following
\begin{align*}
    f(x) = \sin^{2} (e^{\sin^{2} x})
\end{align*}
\end{exercise}

\begin{exercise}
Differentiate the following
\begin{align*}
    f(x) = 2^{3^{x^{2}}}
\end{align*}
\end{exercise}

\begin{theorem}
Derivative of $\arctan x$:
\begin{align*}
    \dfrac{d}{dx} \arctan x = \dfrac{1}{1+x^{2}}
\end{align*}
We show this with the chain rule. We know
\begin{align*}
    f(x) = \arctan x \hspace{20pt} \Longleftrightarrow \hspace{20pt} \tan(f(x)) = x
\end{align*}
By The Chain Rule
\begin{align*}
    \dfrac{d}{dx}\tan(f(x)) = \dfrac{d}{dx}x \hspace{20pt} \Longleftrightarrow \hspace{20pt} \sec^{2}(f(x))f^{'}(x) = 1
\end{align*}
Rearranging and recalling the trigonometric identity $\sec^{2}\theta = 1 + \tan^{2}\theta$
\begin{align*}
    \sec^{2}(f(x))f^{'}(x) = 1 \hspace{20pt} \Longleftrightarrow \hspace{20pt} f^{'}(x) = \dfrac{1}{\sec^{2}(f(x))} \hspace{20pt} \Longleftrightarrow \hspace{20pt} f^{'}(x) = \dfrac{1}{1 + \tan^{2}(f(x))} 
\end{align*}
Now we make use of the fact that $\tan x$ and $\arctan x$ are inverse functions on an attenuated domain $(-\pi/2, \pi/2)$
\begin{align*}
    f^{'}(x) = \dfrac{1}{1+\tan^{2}(f(x))} \hspace{20pt} \Longleftrightarrow \hspace{20pt} f^{'}(x) = \dfrac{1}{1+\tan^{2}(\arctan x)} \hspace{20pt} \Longleftrightarrow \hspace{20pt} f^{'}(x) = \dfrac{1}{1+x^{2}}
\end{align*}
\end{theorem}

\begin{exercise}
Use The Chain Rule to find the derivative for
\begin{align*}
    f(x) = \arcsin x
\end{align*}
\end{exercise}

\begin{exercise}
Use The Chain Rule to find the derivative for
\begin{align*}
    f(x) = \arccos x
\end{align*}
\end{exercise}

\begin{exercise}
Use The Chain Rule to find the derivative for
\begin{align*}
    f(x) = \arccot x
\end{align*}
\end{exercise}

\begin{exercise}
Use The Chain Rule to find the derivative for
\begin{align*}
    f(x) = \arccsc x
\end{align*}
\end{exercise}

\begin{exercise}
Use The Chain Rule to find the derivative for
\begin{align*}
    f(x) = \arcsec x
\end{align*}
\end{exercise}

\begin{exercise}
Differentiate the following
\begin{align*}
    f(x) = \dfrac{1}{\arcsin x}
\end{align*}
\end{exercise}

\begin{exercise}
Differentiate the following
\begin{align*}
    f(x) = x\arctan \sqrt{x}
\end{align*}
\end{exercise}

\begin{exercise}
Use The Chain Rule and the fact that 
\begin{align*}
f(x) = \log_{a} x \hspace{20pt} \Longleftrightarrow \hspace{20pt} a^{f(x)} = x \hspace{20pt} \text{for} \hspace{4pt} a>0
\end{align*}
to show
\begin{align*}
    \dfrac{d}{dx}\log_{a} x = \dfrac{1}{x \ln a}
\end{align*}
\end{exercise}

\begin{exercise}
Use The Chain Rule and the fact that 
\begin{align*}
f(x) = \log_{a} x \hspace{20pt} \Longleftrightarrow \hspace{20pt} a^{f(x)} = x \hspace{20pt} \text{for} \hspace{4pt} a>0
\end{align*}
to show
\begin{align*}
    \dfrac{d}{dx}\ln x = \dfrac{1}{x}
\end{align*}
\end{exercise}

\begin{exercise}
Differentiate the following
\begin{align*}
    f(x) = \ln \lvert x \rvert
\end{align*}
\end{exercise}

\begin{exercise}
Differentiate the following
\begin{align*}
    f(x) = \ln(x^2 + 10)
\end{align*}
\end{exercise}

\begin{exercise}
Differentiate the following
\begin{align*}
    f(x) = \ln (\sin^{2} x)
\end{align*}
\end{exercise}

\begin{exercise}
Differentiate the following
\begin{align*}
    f(x) = \sqrt[\leftroot{2}\uproot{2}5]{x}
\end{align*}
\end{exercise}

\begin{exercise}
Differentiate the following
\begin{align*}
    f(x) = \sqrt[\leftroot{2}\uproot{2}5]{\ln x}
\end{align*}
\end{exercise}

\begin{exercise}
Differentiate the following
\begin{align*}
    f(x) = \sin x \ln (5x)
\end{align*}
\end{exercise}

\begin{exercise}
Differentiate the following
\begin{align*}
    f(x) = \ln (x + \sqrt{x^2 - 1})
\end{align*}
\end{exercise}

\begin{exercise}
Differentiate the following
\begin{align*}
    f(x) = \dfrac{1}{\ln x}
\end{align*}
\end{exercise}

\begin{exercise}
Find the derivative of $f$ and find the domain of $f$
\begin{align*}
    f(x) = \dfrac{x}{1 - \ln(x-1)}
\end{align*}
\end{exercise}

\begin{exercise}
Find the derivative of $f$ and find the domain of $f$
\begin{align*}
    f(x) = \ln (x^{2} - 2x)
\end{align*}
\end{exercise}

\begin{exercise}
Find an equation of the line tangent to $f$ at $(1, 1)$
\begin{align*}
    f(x) = \ln (xe^{x^{2}})
\end{align*}
\end{exercise}

\begin{exercise}
Find an equation of the line tangent to $f$ at $(2, 0)$
\begin{align*}
    f(x) = \ln(x^{3} - 7)
\end{align*}
\end{exercise}

\begin{theorem}
We will show
\begin{align*}
    e = \lim_{x \longrightarrow 0} (1 + x)^{1/x}
\end{align*}
\begin{proof}
We know, for $f(x) = \ln x$, that $f^{'}(x) = \dfrac{1}{x}$. So, $f^{'}(1) = 1$. Using the definition of the limit
\begin{align*}
    &1 = f^{'}(1) = \lim_{x \longrightarrow 0} \dfrac{\ln (1+x) - \ln (1)}{x} = \lim_{x \longrightarrow 0} \dfrac{1}{x} \ln (x + 1) = \lim_{x \longrightarrow 0} \ln (x+1)^{1/x}\\[2ex]
    &\Longrightarrow \hspace{20pt} e = e^{1} = e^{\lim_{x \longrightarrow 0} \ln (x+1)^{1/x}} = \lim_{x \longrightarrow 0} e^{\ln (x+1)^{1/x}} = \lim_{x \longrightarrow 0} (x + 1)^{1/x}
\end{align*}
\end{proof}
\end{theorem}

\begin{exercise}
Find the derivative of the following
\begin{align*}
    f(x) = x^{x}
\end{align*}
\end{exercise}

\begin{exercise}
Find the derivative of the following
\begin{align*}
    f(x) = (\sqrt{x})^{x}
\end{align*}
\end{exercise}

\begin{exercise}
Find the derivative of the following
\begin{align*}
    f(x) = (\sin x)^{\ln x}
\end{align*}
\end{exercise}

\begin{exercise}
Find the derivative of the following
\begin{align*}
    f(x) = (\tan x)^{1/x}
\end{align*}
\end{exercise}

\begin{exercise}
Show the following
\begin{align*}
    \lim_{n \longrightarrow \infty} \Big(1+\dfrac{x}{n}\Big)^{n} = e^{x} \hspace{20pt} \text{for any} \hspace{4pt} x > 0
\end{align*}
\end{exercise}

\newpage
\section{The $n^{\text{th}}$ Derivative}

\begin{note}
The derivative of a function $f$ is again a function $f^{'}$. 
\end{note}

\begin{example}
Take $f(x) = x^{2} + x$. We know $\dfrac{df}{dx} = f^{'}(x) = 2x + 1$. The second derivative of $f$ is the first derivative of $f^{'}$, and the notation used to represent the second derivative is as follows
\begin{align*}
    f^{''}(x) = \dfrac{d^{2}f}{(dx)^{2}} = \dfrac{d}{dx}\dfrac{df}{dx} = \dfrac{df^{'}}{dx} =  \dfrac{d}{dx} (2x + 1) = 2  
\end{align*}
If we wanted the third derivative of $f$, which would be the second derivative of $f^{'}$, we could write
\begin{align*}
    f^{(3)}(x) = \dfrac{d^{3}f}{(dx)^{3}} = \dfrac{d}{dx}\dfrac{d}{dx}\dfrac{df}{dx} = \dfrac{d}{dx}\dfrac{df^{'}}{dx} = \dfrac{df^{''}}{dx} = \dfrac{d}{dx} (2) = 0
\end{align*}
Of course, we could take further derivatives, since $g(x) \equiv 0$ is infinitely differentiable along $\mathbb{R}$ ---we will come across examples in which a function fails to be differentiable after some $k^{\text{th}}$ derivative. To save us the trouble of writing each derivative, we could write more concisely
\begin{align*}
    \dfrac{d^{n}f}{(dx)^{n}} = f^{(n)}(x) = 0 \hspace{20pt} \text{for all} \hspace{4pt} n \geq 3
\end{align*}
\end{example}

\begin{exercise}
Find $f^{'}$, $f^{''}$, and $f^{(3)}$ for the following
\begin{align*}
    f(x) = 1 + 4x - x^{2}
\end{align*}
\end{exercise}

\begin{exercise}
Find $f^{'}$, $f^{''}$, and $f^{(3)}$ for the following. Establish a pattern and provide a general equation for $f^{(n)}$.
\begin{align*}
    f(x) = \dfrac{1}{x}
\end{align*}
\end{exercise}