\section{The Derivative}

\begin{definition}
We say a function $f$ has derivative $L$ at $c \in$ Dom($f$) if 
\begin{align*}
    &\text{For all} \hspace{4pt} \epsilon > 0 \hspace{4pt} \text{there exists a} \hspace{4pt} \delta > 0 \hspace{4pt} \text{such that}\\[2ex]
    &\text{all} \hspace{4pt} x \in \text{Dom($f$)} \hspace{4pt} \text{satisfying} \hspace{4pt} \lvert x-c \rvert < \delta \hspace{20pt} \Longrightarrow \hspace{20pt} \Big\lvert \dfrac{f(x)-f(c)}{x-c}-L \Big\rvert < \epsilon
\end{align*}
We denote $L$ as
\begin{align*}
f^{'}(c) = \lim_{x \longrightarrow c} \dfrac{f(x)-f(c)}{x-c}
\end{align*}
where $f^{'}$ is read '$f$ prime', which we refer to as prime notation. The above can be written equivalently as
\begin{align*}
    f^{'}(c) = \lim_{h \longrightarrow 0} \dfrac{f(c+h)-f(c)}{h}
\end{align*}
\end{definition}

\begin{example}
The derivative of a function $f$ at a point $c \in$ Dom($f$) can be thought of as the slope of the linear function $L(x)$ tangent to function $f$ at point $c$. The slope of a $1$-D line in $2$-D space is what we will deal with most in this course, but the slope of a $2$-D plane in $3$-D space or the slope of a $(n-1)$-D plane in $n$-D space can also be a derivative of some function in these respective spaces. Below, we show a subset of the secant lines (in black) passing through the point $(\pi/4, \sqrt{2}/2)$, all of which can be formed as we push towards the derivative (tangent line in red) passing through the point $(\pi/4, \sqrt{2}/2)$.
\begin{align*}
    &\text{When} \hspace{4pt} f(x) = \sin x\\[2ex]
    &\text{and when} \hspace{4pt} f^{'}\Big(\dfrac{\pi}{4}\Big) = \lim_{h \longrightarrow 0} \dfrac{\sin\Big(\dfrac{\pi}{4} + h\Big) - \sin\Big(\dfrac{\pi}{4}\Big)}{h}\\[2ex]
    &\text{then we have the tangent line} \hspace{4pt} L(x) = (f^{'}(x))x + b \hspace{20pt}\\[2ex]
    &\text{where $b$ is the intercept of the linear function $L$}
\end{align*}

\resizebox{30em}{30em}{%
\begin{tikzpicture}[scale=\textwidth/4.2cm]
    % title and axes
    \node at (0.6, 1.2) {$f(x)=\sin x \hspace{4pt} x \in [0, \pi]$};
    \draw (0, 0) -- (pi, 0)
        node[right] {$x$};
    \draw (0, 0) -- (0, 2.4)
        node[above] {$f(x)$};
    % graph
    \draw[red, very thick] plot[smooth] file {derivatives_of_functions/python_generated_tables/sin_0_pi_tangent.table};
    \draw[black] plot[smooth] file {derivatives_of_functions/python_generated_tables/sin_0_pi_line_5.table};
    \draw[black] plot[smooth] file {derivatives_of_functions/python_generated_tables/sin_0_pi_line_6.table};
    \draw[blue, very thick] plot[smooth] file {derivatives_of_functions/python_generated_tables/sin_0_pi.table};
    \draw[black] plot[smooth] file {derivatives_of_functions/python_generated_tables/sin_0_pi_line_1.table};
    \draw[black] plot[smooth] file {derivatives_of_functions/python_generated_tables/sin_0_pi_line_2.table};
    \draw[black] plot[smooth] file {derivatives_of_functions/python_generated_tables/sin_0_pi_line_3.table};
    \draw[black] plot[smooth] file {derivatives_of_functions/python_generated_tables/sin_0_pi_line_4.table};
    \draw[blue, fill=red] (pi/4, {sqrt(2)/2}) circle (.25mm);
    \draw[black, fill=black] (2*pi/3, {sqrt(3)/2}) circle (.25mm);
    \draw[black, fill=black] (pi/2, 1) circle (.25mm);
    \node at (pi/4, -0.1) {$\pi/4$};
    \node at (2*pi/3, -0.1) {$2\pi/3$};
    \node at (pi/2, -0.1) {$\pi/2$};
    \draw[dotted] (pi/4, 0) -- (pi/4, {sqrt(2)/2});
    \draw[dotted] (pi/2, 0) -- (pi/2, 1);
    \draw[dotted] (2*pi/3, 0) -- (2*pi/3, {sqrt(3)/2});
\end{tikzpicture}
}
\end{example}

\begin{exercise}
Use the definition of the derivative to find the derivative $f^{'}(x)$ for
\begin{align*}
    f(x) = x^{3} \hspace{20pt} x \in \mathbb{R}
\end{align*}
\end{exercise}

\begin{exercise}
Use the definition of the derivative to find the derivative $f^{'}(x)$ for
\begin{align*}
    f(x) = \dfrac{1}{x} \hspace{20pt} x \in \mathbb{R}, \hspace{4pt} x \neq 0
\end{align*}
\end{exercise}

\begin{exercise}
Use the definition of the derivative to find the derivative $f^{'}(x)$ for
\begin{align*}
    f(x) = \sqrt{x} \hspace{20pt} x > 0
\end{align*}
\end{exercise}

\begin{exercise}
Use the definition of the derivative to find the derivative $f^{'}(x)$ for
\begin{align*}
    f(x) = \dfrac{1}{\sqrt{x}} \hspace{20pt} x > 0
\end{align*}
\end{exercise}

\begin{exercise}
Use the definition of the derivative to find the derivative $f^{'}(x)$ for
\begin{align*}
    f(x) = x^{3} - x
\end{align*}
\end{exercise}

\begin{exercise}
Use the definition of the derivative to find the derivative $f^{'}(x)$ for
\begin{align*}
    f(x) = \dfrac{1-x}{2+x}
\end{align*}
\end{exercise}

\begin{exercise}
Use the definition of the derivative to find the derivative $f^{'}(x)$ for
\begin{align*}
    f(x) = \lvert x \rvert
\end{align*}
\end{exercise}

\newpage
\section{Derivatives of Functions and Properties of Derivatives}

\begin{note}
In addition to the prime notation, we will use $\dfrac{d}{dx}$ to represent the derivative operator. The derivative operator acts on a function $f$ to generate the derivative $f^{'}$ like so
\begin{align*}
    \dfrac{d}{dx}f(x) = f^{'}(x)
\end{align*}
\end{note}

\begin{theorem}
Derivative of constant $a$:
\begin{align*}
    \dfrac{d}{dx}a = 0 \hspace{20pt} a \in \mathbb{R}
\end{align*}
\end{theorem}

\begin{theorem}
Derivative of $x^{a}$:
\begin{align*}
    \dfrac{d}{dx}x^{a} = ax^{a-1} \hspace{20pt} a \in \mathbb{R}
\end{align*}
\end{theorem}

\begin{theorem}
Let $f$ and $g$ be functions differentiable at $c \in$ Dom($f$), Dom($g$). Then
\begin{align*}
    \text{For $a \in \mathbb{R}$} \hspace{20pt} &(af)^{'}(c) = af^{'}(c)\\[2ex]
    &(f+g)^{'}(c) = f^{'}(c) + g^{'}(c)\\[2ex]
    &(f-g)^{'}(c) = f^{'}(c) - g^{'}(c)\\[2ex]
    &(fg)^{'}(c) = f^{'}(c)g(c) + f(c)g^{'}(c)\\[2ex]
    \text{Assuming $g(c) \neq 0$} \hspace{20pt} &\Big(\dfrac{f}{g}\Big)^{'}(c) = \dfrac{f^{'}(c)g(c) - f(c)g^{'}(c)}{(g(c))^{2}}
\end{align*}
\end{theorem}