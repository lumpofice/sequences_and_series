\newpage
\section{Extrema}

\begin{definition}
Let $f$ be a function on $[a, b]$. We say $f$ has a local minimum at $c \in [a, b]$ if for any $\delta > 0$ we have, for all $x$ such that $\lvert x - c \rvert < \delta$, $f(x) \geq f(c)$.
\end{definition}

\begin{definition}
Let $f$ be a function on $[a, b]$. We say $f$ has a local maximum at $c \in [a, b]$ if for any $\delta > 0$ we have, for all $x$ such that $\lvert x - c \rvert < \delta$, $f(x) \leq f(c)$.
\end{definition}

\begin{theorem}
If $f$ has a local minimum or maximum at $c \in [a, b]$ and if $f^{'}(c)$ exists, then $f^{'}(c) = 0$.
\end{theorem}

\begin{definition}
Let $f$ be a function on $[a, b]$. We say $f$ has an absolute minimum, say at $c \in [a, b]$, if $f(c) \leq f(x)$ for all $x \in [a, b]$.
\end{definition}

\begin{definition}
Let $f$ be a function on $[a, b]$. We say $f$ has an absolute maximum, say at $c \in [a, b]$, if $f(c) \geq f(x)$ for all $x \in [a, b]$.
\end{definition}

\begin{theorem}
If function $f$ is defined on a closed and bounded interval $[a, b]$, then the range of $f$ will also be closed and bounded. Meaning
\begin{align*}
    f \hspace{10pt} \text{continuous on} \hspace{10pt} [a, b] \hspace{20pt} \Longrightarrow \hspace{20pt} \text{Rng}(f) = [c, d]
\end{align*}
This necessarily implies that $\text{Rng}(f)$ will contain an absolute maximum and an absolute minimum.
\end{theorem}

\begin{exercise}
If $a$ and $b$ are both positive numbers, then find the maximum value $f$, given $f$ is defined as follows
\begin{align*}
    f(x) = x^{a}(1-x)^{b} \hspace{20pt} x \in [0, 1]
\end{align*}
\end{exercise}

\begin{exercise}
An object with weight $W$ is dragged along a horizontal plane by a force acting along a rope attached to the object. If the rope makes an angle $\theta$ with the plane, then the magnitude of the force is 
\begin{align*}
    F(\theta) = \dfrac{\mu W}{\mu \sin(\theta) + \cos(\theta)} \hspace{20pt} \theta \in \Big[0, \dfrac{\pi}{2}\Big] 
\end{align*}
where $\mu$ is a positive constant called the coefficient of friction. Show that $F$ has a minimum at $\theta = \arctan(\mu)$.
\end{exercise}

\newpage
\section{The Mean Value Theorem}

\begin{theorem}
Suppose that $f$ is continuous on $[a, b]$ and differentiable on $(a, b)$, then there exists a point $c \in (a, b)$ such that
\begin{align*}
    f(b) - f(a) = f^{'}(c)(b - a)
\end{align*}
This is known as the Mean Value Theorem.
\end{theorem}

\begin{exercise}
Let 
\begin{align*}
    f(x) = (x-3)^{-2}
\end{align*}
Show there is no value $c \in (1,4)$ satisfying the Mean Value Theorem, and explain why this does not contradict the Mean Value Theorem. 
\end{exercise}

\begin{exercise}
Let $f$ and $g$ be continuous on $[a, b]$ and differentiable on $(a, b)$. Let $f(a) = g(a)$, let $f^{'}(x) \leq g^{'}(x)$ for all $x \in (a, b)$. Prove, using
\begin{align*}
    h = f - g
\end{align*}
that $f(b) < g(b)$.
\end{exercise}

\begin{exercise}
Suppose two runners start a race at time $t = 0$, and suppose these two runners finish the race in a tie at time $t = T$. Using
\begin{align*}
    f = g - h
\end{align*}
where $g$ and $h$ are position functions of the two runners, along with the fact that the derivative of the position is the velocity (speed), show that at some time during the race the two runners had the same speed.
\end{exercise}

\newpage
\section{L'Hospital's Rule}

L'Hospital's Rule is an application of differentiation that handles indeterminate forms $\dfrac{0}{0}$ and $\dfrac{\infty}{\infty}$ and $\dfrac{\pm \infty}{\mp \infty}$.

\vspace{0.1in}
\begin{theorem}
Let $-\infty \leq a < b \leq \infty$ and let $f, g$ be differentiable on $(a, b)$, with $g(x) \neq 0$ for all $a < x < b$. Suppose 
\begin{align*}
    \lim_{x \longrightarrow a^{+}} f(x) = 0 = \lim_{x \longrightarrow a^{+}} g(x)
\end{align*}
If
\begin{align*}
    \lim_{x \longrightarrow a^{+}} \dfrac{f^{'}(x)}{g^{'}(x)} = L \in \mathbb{R} \cup \{-\infty, \infty\}
\end{align*}
then
\begin{align*}
    \lim_{x \longrightarrow a^{+}} \dfrac{f(x)}{g(x)} = L
\end{align*}
\label{L'Hospital_1}
\end{theorem}

\begin{note}
This theorem applies in the same way for left-sided and two-sided limits.
\end{note}

\begin{theorem}
Let $-\infty \leq a < b \leq \infty$ and let $f, g$ be differentiable on $(a, b)$, where $g(x) \neq 0$ for all $a < x < b$. Suppose
\begin{align*}
    \lim_{x \longrightarrow a^{+}} g(x) = \pm \infty 
\end{align*}
If
\begin{align*}
    \lim_{x \longrightarrow a^{+}} \dfrac{f^{'}(x)}{g^{'}(x)} = L \in \mathbb{R} \cup \{-\infty, \infty\}
\end{align*}
then
\begin{align*}
    \lim_{x \longrightarrow a^{+}} \dfrac{f(x)}{g(x)} = L
\end{align*}
\label{L'Hospital_2}
\end{theorem}

\begin{note}
This theorem applies in the same way for left-sided and two-sided limits.
\end{note}

\begin{note}
For Theorem \ref{L'Hospital_2}, we only require $\lim_{x \longrightarrow a^{+}} g(x) = \pm \infty$. If 
\begin{align*}
    \lim_{x \longrightarrow a^{+}} f(x)
\end{align*} 
is finite,
\begin{align*}
    \lim_{x \longrightarrow a^{+}} \dfrac{f(x)}{g(x)} = 0
\end{align*}
and L'Hospital's Rule is not required. So, L'Hospital's Rule is only used if both $f, g$ have infinite limits.
\end{note}

\begin{note}
Indeterminate forms, such as
\begin{align*}
    &\infty - \infty\\
    &0 \cdot \infty\\
    &1^{\infty}\\
    &0^{0}\\
    &\infty^{0}
\end{align*}
can be reduced to those cases considered previously in Theorem \ref{L'Hospital_1} and Theorem \ref{L'Hospital_2}.
\end{note}