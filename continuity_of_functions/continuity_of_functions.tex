\section{Continuity of Functions}

\begin{definition}
We say a function $f$ is continuous at a point $c \in \text{Dom($f$)}$ if
\begin{align*}
    &\text{For any} \hspace{4pt} \epsilon > 0, \hspace{4pt} \text{there exists a} \hspace{4pt} \delta > 0 \hspace{4pt} \text{such that}\\[2ex]
    &\text{for any} \hspace{4pt} x \hspace{4pt} \text{satisfying} \hspace{4pt} \lvert x - c \rvert < \delta \hspace{20pt} \text{we have} \hspace{20pt} \lvert f(x) - f(c) \rvert < \epsilon
\end{align*}
This is equivalent to saying $\lim_{x \longrightarrow c} f(x) = f(c)$ when $f$ is continuous at $c$. For continuity to hold, we need three things to be true:
\begin{itemize}
    \item $f$ is defined at $c$. Meaning, $f(c)$ must exist.
    \item $\lim_{x \longrightarrow c} f(x)$ must exist.
    \item $lim_{x \longrightarrow c} f(x) = f(c)$
\end{itemize}
In fact, the definition of continuity is almost exactly the same as the definition of a limit, Definition \ref{definition_limit_of_functions} in section \ref{limits_of_functions_section}.
\label{definition_function_continuity}
\end{definition}

\begin{exercise}
What is the difference between the definition of the limit of a function, Definition \ref{definition_limit_of_functions} in section \ref{limits_of_functions_section}, and the definition of continuity of a function, Definition \ref{definition_function_continuity}?
\end{exercise}

\begin{theorem}
If $f$ is continuous at $c$, then $f$ has a limit at $c$.
\end{theorem}

\begin{exercise}
TRUE or FALSE: If $f$ is not continuous at $c$ (we say $f$ is discontinuous when $f$ is not continuous) then $f$ does not have a limit at $c$.
\end{exercise}

\begin{exercise}
Prove the following theorem:
\begin{theorem}
If $f$ and $g$ are functions continuous at $a$, then the following functions are also continuous:
\begin{align*}
    &f + g\\[2ex]
    &f - g\\[2ex]
    &cf \hspace{20pt} \text{where} \hspace{20pt} c \in \mathbb{R}\\[2ex]
    &fg\\[2ex]
    &\dfrac{f}{g} \hspace{20pt} \text{where} \hspace{20pt} g(a) \neq 0
\end{align*}
\end{theorem}
\end{exercise}

\begin{definition}
We say a function $f$ is continuous on Dom($f$) if for every $c \in \text{Dom($f$)}$ we have 
\begin{align*}
    &\text{For any} \hspace{4pt} \epsilon > 0, \hspace{4pt} \text{there exists a} \hspace{4pt} \delta > 0 \hspace{4pt} \text{such that}\\[2ex]
    &\text{for any} \hspace{4pt} x \hspace{4pt} \text{satisfying} \hspace{4pt} \lvert x - c \rvert < \delta \hspace{20pt} \text{we have} \hspace{20pt} \lvert f(x) - f(c) \rvert < \epsilon
\end{align*}
\end{definition}

\begin{example}
Revisiting our example from earlier, Example \ref{limit_of_sin_0_1} in section \ref{limits_of_functions_section}, we discovered that $f(x) = \sin x$, where $x$ belongs to the interval $[0, 1]$, has a limit at $c=1/2$. Since $f$ is defined at $c=1/2$, we also have
\begin{align*}
    \lim_{x \longrightarrow 1/2} f(x) = f\Big(\dfrac{1}{2}\Big) = \sin \Big(\dfrac{1}{2}\Big)
\end{align*}
Since the sine function is defined at every point of the real line, 
\begin{align*}
    \lim_{x \longrightarrow c} \sin x = \sin c \hspace{20pt} \text{for all} x \in \mathbb{R}
\end{align*}

\resizebox{30em}{30em}{%
\begin{tikzpicture}[scale=\textwidth/4.2cm]
    % title and axes
    \node at (0.7, 1.1) {\tiny$f(x)=\sin x \hspace{4pt} x \in [0, 1]$};
    \draw (0, 0) -- (1, 0)
        node[right] {\tiny$x$};
    \draw (0, 0) -- (0, 1)
        node[above] {\tiny$f(x)$};
    % ----------------------------------
    % range boundaries lower
    \draw[dotted] (0, {sin(0.45 r)}) -- (0.6, {sin(0.45 r)});
    \node[] at (-0.15, {sin(0.45 r)}) {\tiny$L-\epsilon$};
    \node [rotate=90] at (0, {sin(0.45 r)}) {\tiny(};
    % range boundaries upper
    \draw[dotted] (0, {sin(0.55 r)}) -- (0.6, {sin(0.55 r)});
    \node[] at (-0.15, {sin(0.55 r)}) {\tiny$L+\epsilon$};
    \node [rotate=-90] at (0, {sin(0.55 r)}) {\tiny(};
    % ----------------------------------
    % domain boundaries lower
    \draw[dotted] (0.47, 0) -- (0.47, {sin(0.6 r)});
    \node [rotate=45] at (0.35, -0.15) {\tiny$\dfrac{1}{2}-\delta$ $\longrightarrow$};
    \node [] at (0.47, 0) {\tiny(};
    % domain boundaries upper
    \draw[dotted] (0.53, 0) -- (0.53, {sin(0.6 r)});
    \node [rotate=-45] at (0.65, -0.15) {\tiny$\longleftarrow$ \tiny$\dfrac{1}{2}+\delta$};
    \node [] at (0.53, 0) {\tiny)};
    % ----------------------------------
    % graph
    \draw[blue] plot[smooth] file {limits_of_functions/python_generated_tables/sine_0_1_piece_0.table};
    \draw[blue] plot[smooth] file {limits_of_functions/python_generated_tables/sine_0_1_piece_1.table};
    \draw[blue, fill=red] (0.5,{sin(0.5 r)}) circle (.25mm);
\end{tikzpicture}
}
\end{example}

\begin{example}
As part of a piecewise function $f$, we again have the sine function on an attenuated domain, but with the function $f$ defined at at $c=1/2$ for a different function. Namely, the constant $1$. We see the function is discontinuous at $c=1/2$:
\begin{align*}
    \lim_{x \longrightarrow 1/2} f(x) \neq f\Big(\dfrac{1}{2}\Big)
\end{align*}

\resizebox{30em}{30em}{%
\begin{tikzpicture}[scale=\textwidth/4.2cm]
    % title and axes
    \node at (0.7, 1.2) {
    \tiny$f(x)= 
    \begin{cases}
    \sin x, &x \neq 1/2\\
    1, &x = 1/2
    \end{cases}$};
    \draw (0, 0) -- (1, 0)
        node[right] {\tiny$x$};
    \draw (0, 0) -- (0, 1)
        node[above] {\tiny$f(x)$};
    % ----------------------------------
    % range boundaries lower
    \draw[dotted] (0, {sin(0.45 r)}) -- (0.6, {sin(0.45 r)});
    \node[] at (-0.15, {sin(0.45 r)}) {\tiny$L-\epsilon$};
    \node [rotate=90] at (0, {sin(0.45 r)}) {\tiny(};
    % range boundaries upper
    \draw[dotted] (0, {sin(0.55 r)}) -- (0.6, {sin(0.55 r)});
    \node[] at (-0.15, {sin(0.55 r)}) {\tiny$L+\epsilon$};
    \node [rotate=-90] at (0, {sin(0.55 r)}) {\tiny(};
    % ----------------------------------
    % domain boundaries lower
    \draw[dotted] (0.47, 0) -- (0.47, {sin(0.6 r)});
    \node [rotate=45] at (0.35, -0.15) {\tiny$\dfrac{1}{2}-\delta$ $\longrightarrow$};
    \node [] at (0.47, 0) {\tiny(};
    % domain boundaries upper
    \draw[dotted] (0.53, 0) -- (0.53, {sin(0.6 r)});
    \node [rotate=-45] at (0.65, -0.15) {\tiny$\longleftarrow$ \tiny$\dfrac{1}{2}+\delta$};
    \node [] at (0.53, 0) {\tiny)};
    % ----------------------------------
    % graph
    \draw[blue] plot[smooth] file {limits_of_functions/python_generated_tables/sine_0_1_piece_0.table};
    \draw[blue] plot[smooth] file {limits_of_functions/python_generated_tables/sine_0_1_piece_1.table};
    \draw[blue, fill=white] (0.5,{sin(0.5 r)}) circle (.25mm);
    \draw[blue, fill=red] (0.5, 1) circle (.25mm);
    \node at (-0.1, 1) {\tiny $1$};
    \node at (-0.1, {sin(1 r)}) {\tiny $\sin (1)$};
    \draw[dotted] (0, {sin(1 r)}) -- (1, {sin(1 r)});
    \draw[dotted] (0, 1) -- (0.5, 1);
\end{tikzpicture}
}
\end{example}

\begin{recall}
From section \ref{limits_of_functions_section}, we stated that 
\begin{align*}
    \lim_{x \longrightarrow c} x = c \hspace{4pt} \text{for all} x \in \mathbb{R} \hspace{20pt} \text{when} \hspace{20pt} f(x) = x 
\end{align*}
This means that $f(x) = x$ is continuous on the real line. This fact brings us to another useful result:
\end{recall}

\begin{theorem}
Every polynomial $p$, defined formally by
\begin{align*}
    &p(x) = a_{n}x^{n} + a_{n-1}x^{n-1} + \cdots + a_{2}x^{2} + a_{1}x + a_{0}\\[2ex]
    &\text{where} \hspace{4pt} \{a_{n}, a_{n-1}, \cdots , a_{3}, a_{2}, a_{1}, a_{0}\} \hspace{4pt} \text{are all real numbers and} \hspace{4pt} a_{n} \neq 0
\end{align*}
is continuous on the real line. This, in turn, provides us with another useful result:
\end{theorem}

\begin{theorem}
Every rational function $R(x)=\dfrac{p(x)}{q(x)}$, where $p$ and $q$ are polynomials, is continuous for all real numbers $c$ such that $q(c) \neq 0$.
\end{theorem}

\begin{exercise}
Where along the real line is the following continuous:
\begin{align*}
    f(x) = \dfrac{x^{2}-x-2}{x-2}
\end{align*}
\end{exercise}

\begin{exercise}
Where along the real line is the following continuous:
\begin{align*}
    f(x) =
    \begin{cases}
    \dfrac{1}{x^{2}}, &x \neq 0\\
    1, &x = 0
    \end{cases}
\end{align*}
\end{exercise}

\begin{exercise}
Where along the real line is the following discontinuous:
\begin{align*}
    f(x) = 
    \begin{cases}
    \dfrac{x^{2}-x-2}{x-2}, &x \neq 2 \\
    1, & x=2
    \end{cases}
\end{align*}
\end{exercise}

\begin{definition}
We say a function $f$ is right-continuous if
\begin{align*}
    \lim_{x \longrightarrow c^{+}} f(x) = f(c)
\end{align*}
We say a function is left-continuous if
\begin{align*}
    \lim_{x \longrightarrow c^{-}} f(x) = f(c)
\end{align*}
\end{definition}

\begin{theorem}
A function $f$ is continuous if and only if
\begin{align*}
    \lim_{x \longrightarrow c^{-}} f(x) = f(c) = \lim_{x \longrightarrow c^{+}} f(x) 
\end{align*}
\end{theorem}

\begin{example}
Below is a visual of the floor function on the attenuated domain $[0, 2)$, with the open, white-colored circles depicting where a function is discontinuous from the left, and the closed, red-colored circles depicting where a function is continuous from the right. We see that $f$ is discontinuous at $1$:
\begin{align*}
    &\lim_{x \longrightarrow 1^{-}} \lfloor x \rfloor = 0 \hspace{20pt} \lim_{x \longrightarrow 1^{+}} \lfloor x \rfloor = 1\\[2ex]
    &\text{Since} \hspace{4pt} \lim_{x \longrightarrow 1^{-}} f(x) \neq f(1) \hspace{4pt} f \hspace{4pt} \text{is not continuous at} \hspace{4pt} 1
\end{align*}

\resizebox{30em}{30em}{%
\begin{tikzpicture}[scale=\textwidth/4.2cm]
    % title and axes
    \node at (1.3, 1.5) {$f(x)=\lfloor x \rfloor \hspace{4pt} x \in [0, 2)$};
    \draw (0, 0) -- (2, 0)
        node[right] {$x$};
    \draw (0, 0) -- (0, 1.3)
        node[above] {$f(x)$};
    % graph
    \draw[blue, very thick] plot[smooth] file {limits_of_functions/python_generated_tables/floor_0_2_piece_0.table};
    \draw[blue, very thick] plot[smooth] file {limits_of_functions/python_generated_tables/floor_0_2_piece_1.table};
    \draw[blue, fill=white] (1,0) circle (.25mm);
    \draw[blue, fill=white] (2,1) circle (.25mm);
    \draw[blue, fill=red] (0,0) circle (.25mm);
    \draw[blue, fill=red] (1,1) circle (.25mm);
    \node at (0, -0.1) {0};
    \node at (1, -0.1) {1};
    \node at (2, -0.1) {2};
\end{tikzpicture}
}
\end{example}

\begin{exercise}
For the floor function $f(x) = \lfloor x \rfloor$, for an arbitrary integer $n$, find
\begin{align*}
    &\lim_{x \longrightarrow n^{-}} f(x)\\
    &\lim_{x \longrightarrow n^{+}} f(x)
\end{align*}
\end{exercise}

\begin{example}
Here we have the ceiling function on the attenuated domain (-1, 1]:
\begin{align*}
    f(x) = \lceil x \rceil \hspace{20pt} x \in [-1, 1)
\end{align*}
We see that it is discontinuous at $0$:
\begin{align*}
    &\lim_{x \longrightarrow 0^{-}} \lceil x \rceil = 0 \hspace{20pt} \lim_{x \longrightarrow 0^{+}} \lceil x \rceil 1\\[2ex]
    &\text{Since} \hspace{4pt} \lim_{x \longrightarrow 0^{+}}f(x) \neq f(0) \hspace{4pt} f \hspace{4pt} \text{is discontinuous at} \hspace{4pt} 0
\end{align*}

\resizebox{30em}{30em}{%
\begin{tikzpicture}[scale=\textwidth/4.2cm]
    % title and axes
    \node at (0.6, 1.2) {$f(x)=\lceil x \rceil \hspace{4pt} x \in (-1, 1]$};
    \draw (-1, 0) -- (1, 0)
        node[right] {$x$};
    \draw (0, 0) -- (0, 1.3)
        node[above] {$f(x)$};
    % graph
    \draw[blue, very thick] plot[smooth] file {continuity_of_functions/python_generated_tables/ceiling_neg1_1_piece_0.table};
    \draw[blue, very thick] plot[smooth] file {continuity_of_functions/python_generated_tables/ceiling_neg1_1_piece_1.table};
    \draw[blue, fill=red] (0,0) circle (.25mm);
    \draw[blue, fill=red] (1,1) circle (.25mm);
    \draw[blue, fill=white] (-1,0) circle (.25mm);
    \draw[blue, fill=white] (0,1) circle (.25mm);
    \node at (-1, -0.1) {-1};
    \node at (0, -0.1) {0};
    \node at (1, -0.1) {1};
\end{tikzpicture}
}
\end{example}

\begin{exercise}
For the ceiling function $f(x) = \lceil x \rceil$, for an arbitrary integer $n$, find
\begin{align*}
    &\lim_{x \longrightarrow n^{-}} f(x)\\
    &\lim_{x \longrightarrow n^{+}} f(x)
\end{align*}
\end{exercise}

\begin{exercise}
Is $f$ continuous at $c=-2$? If so, then:
\begin{align*}
    \text{For} \hspace{4pt} f(x) = \dfrac{x^{3}+2x^{2}-1}{5-3x} \hspace{20pt} \text{find} \hspace{20pt} \lim_{x \longrightarrow -2} f(x)
\end{align*}
\end{exercise}

\begin{exercise}
State the domain of the function and find where it is continuous:
\begin{align*}
    f(x) = \dfrac{\ln x + \arctan x}{x^{2}-1}
\end{align*}
\end{exercise}

\begin{exercise}
State the domain of the function and find where it is continuous:
\begin{align*}
    &f(x) = \dfrac{\sin x}{2 + \cos x}\\
    \text{Find} \hspace{4pt} &\lim_{x \longrightarrow \pi} f(x)
\end{align*}
\end{exercise}

\begin{exercise}
State the domain of the function and find where it is continuous:
\begin{align*}
    &f(x) = \dfrac{\sin x}{2 + 2\cos x}\\
    \text{Find} \hspace{4pt} &\lim_{x \longrightarrow -\pi} f(x)
\end{align*}
\end{exercise}

\begin{theorem}
If $f$ is continuous at $b$ and $\lim_{x \longrightarrow a} g(x) = b$, then
\begin{align*}
    \lim_{x \longrightarrow a} f(g(x)) = f(b) \hspace{20pt} \text{which means, equivalently} \hspace{20pt} \lim_{x \longrightarrow a} f(g(x)) = f\Big(\lim_{x \longrightarrow a} g(x) \Big)
\end{align*}
\label{limit_passes_function}
\end{theorem}

\begin{exercise}
Find the following:
\begin{align*}
    \lim_{x \longrightarrow 1} \arcsin \Big( \dfrac{1 - \sqrt{x}}{1 - x} \Big)
\end{align*}
\end{exercise}

\begin{theorem}
If $g$ is continuous at $a$ and if $f$ is continuous at $g(a)$, then
\begin{align*}
    f(g(x)) \hspace{20pt} \text{is continuous at} \hspace{20pt} a
\end{align*}
\label{continuity_passes_function}
\end{theorem}

\begin{exercise}
Regarding Theorems \ref{limit_passes_function}, \ref{continuity_passes_function}, find the following:
\begin{align*}
    &\text{TRUE or FALSE:} \hspace{20pt} \text{For Theorem \ref{limit_passes_function}, $b$ necessarily belongs to Dom($f$)}\\[2ex]
    &\text{TRUE or FALSE:} \hspace{20pt} \text{For Theorem \ref{limit_passes_function}, $a$ necessarily belongs to Dom($g$)}\\[2ex]
    &\text{TRUE or FALSE:} \hspace{20pt} \text{For Theorem \ref{limit_passes_function}, $\lim_{x \longrightarrow a} g(x)$ necessarily belongs to Dom($f$)}\\[2ex]
    &\text{TRUE or FALSE:} \hspace{20pt} \text{For Theorem \ref{continuity_passes_function}, $a$ necessarily belongs to Dom($g$)}\\[2ex]
    &\text{TRUE or FALSE:} \hspace{20pt} \text{For Theorem \ref{continuity_passes_function}, $g(a)$ necessarily belongs to Dom($f$)}
\end{align*}
\end{exercise}

\begin{exercise}
State the domain of the function and state where it is continuous:
\begin{align*}
    f(x) = \ln (1 + \cos x)
\end{align*}
\end{exercise}

\newpage
\section{More Theorems and Properties of Continuity}

\begin{theorem}
We have the following equivalence:
\begin{align*}
    &\lim_{x \longrightarrow c} g(x) = g(c) \hspace{20pt} \text{if and only if}\\[2ex]
    &\text{for all} \hspace{4pt} \{f(n)\}_{n=1}^{\infty} \hspace{4pt} \text{contained in Dom ($g$) such that} \hspace{4pt} \lim_{n \longrightarrow \infty} f(n)=c\\[2ex]
    &\text{we have} \hspace{20pt} \lim_{n \longrightarrow \infty} g(f(n)) = g(c)
\end{align*}
\label{sequential_criterion_for_continuity}
\end{theorem}

\begin{exercise}
Regarding Theorems \ref{sequential_criterion_for_limits}, \ref{sequential_criterion_for_continuity}, find the following:
\begin{align*}
    &\text{TRUE or FALSE:} \hspace{20pt} \text{For Theorem \ref{sequential_criterion_for_limits}, $c$ necessarily belongs to Dom($g$)}\\[2ex]
    &\text{TRUE or FALSE:} \hspace{20pt} \text{For Theorem \ref{sequential_criterion_for_limits}, $\lim_{n \longrightarrow \infty} f(n)$ necessarily belongs to Dom($g$)}\\[2ex]
    &\text{TRUE or FALSE:} \hspace{20pt} \text{For Theorem \ref{sequential_criterion_for_continuity}, $c$ necessarily belongs to Dom($g$)}\\[2ex]
    &\text{TRUE or FALSE:} \hspace{20pt} \text{For Theorem \ref{sequential_criterion_for_continuity}, $\lim_{n \longrightarrow \infty} f(n)$ necessarily belongs to Dom($g$)}
\end{align*}
\end{exercise}

\begin{exercise}
At which points is the following function $f$ discontinuous? Also, find the following:
\begin{flalign*}
    &\text{a)} \hspace{4pt} \lim_{x \longrightarrow -1} f(x) &&\\[2ex]
    &\text{b)} \hspace{4pt} \lim_{x \longrightarrow 0} f(x) &&\\[2ex]
    &\text{c)} \hspace{4pt} \lim_{x \longrightarrow 1} f(x) &&\\[2ex]
    &\text{d)} \hspace{4pt} \lim_{x \longrightarrow 2} f(x) &&
\end{flalign*}

\resizebox{30em}{30em}{%
\begin{tikzpicture}[scale=\textwidth/4.2cm]
    % axes
    \draw (-1, 0) -- (2, 0)
        node[right] {$x$};
    \draw (0, -1) -- (0, 2.1)
        node[above] {$f(x)$};
    % graph
    \draw[blue, very thick] plot[smooth] file {continuity_of_functions/python_generated_tables/arctan_neg1_1_piece_0.table};
    \draw[blue, very thick] plot[smooth] file {continuity_of_functions/python_generated_tables/arctan_neg1_1_piece_1.table};
    \draw[blue, very thick] plot[smooth] file {continuity_of_functions/python_generated_tables/id_1_2.table};
    \draw[blue, fill=white] (0,0) circle (.25mm);
    \draw[blue, fill=red] (-1,-pi/4) circle (.25mm);
    \draw[blue, fill=white] (1, pi/4) circle (.25mm);
    \draw[blue, fill=red] (1,1) circle (.25mm);
    \draw[blue, fill=white] (2,2) circle (.25mm);
    \draw[blue, fill=red] (0, {sin(pi/6 r)}) circle (.25mm);
    \draw[blue, fill=red] (2, {sin(pi/6 r)}) circle (.25mm);
    \node at (-1, -0.1) {-1};
    \node at (0.05, -0.1) {0};
    \node at (1, -0.1) {1};
    \node at (2, -0.1) {2};
    \draw[dotted] (0, -pi/4) -- (-1, -pi/4);
    \node at (0.2, -pi/4) {$-\pi/4$};
    \draw[dotted] (0, {sin(pi/6 r)}) -- (2, {sin(pi/6 r)});
    \node at (-0.3, {sin(pi/6 r)}) {$\sin \Big(\dfrac{\pi}{6}\Big)$};
    \draw[dotted] (0, pi/4) -- (1, pi/4);
    \node at (-0.2, pi/4) {$\pi/4$};
    \draw[dotted] (0, 1) -- (1, 1);
    \node at (-0.2, 1) {1};
    \draw[dotted] (0, 2) -- (2, 2);
    \node at (-0.2, 2) {2};
\end{tikzpicture}
}
\end{exercise}

\begin{exercise}
Find the following:
\begin{align*}
    \lim_{x \longrightarrow 4} \dfrac{5 + \sqrt{x}}{\sqrt{5 + x}}
\end{align*}
\end{exercise}

\begin{exercise}
Find the following:
\begin{align*}
    \lim_{x \longrightarrow \pi} \sin (x + \sin x)
\end{align*}
\end{exercise}

\begin{exercise}
Find the following:
\begin{align*}
    \lim_{x \longrightarrow 1} e^{x^{2}-x}
\end{align*}
\end{exercise}

\begin{exercise}
Find the following:
\begin{align*}
    \lim_{x \longrightarrow 2} \arctan \Big( \dfrac{x^{2}-4}{3x^{2}-6x} \Big)
\end{align*}
\end{exercise}

\begin{exercise}
Prove that $f$ is continuous at $c$ if and only if
\begin{align*}
    \lim_{h \longrightarrow 0} f(c + h) = f(c)
\end{align*}
\end{exercise}

\newpage
\section{Intermediate Value Theorem}

\begin{theorem}
Let $I = [c, d]$. For function $f$ continuous on $I$
\begin{align*}
    &\text{if} \hspace{4pt} a, b \in I \hspace{4pt} \text{such that} f(a)<k<f(b) \hspace{4pt} \text{for some} \hspace{4pt} k \in \mathbb{R}\\[2ex]
    &\text{then there exists a} \hspace{4pt} c \in (a, b) \hspace{4pt} \text{such that} \hspace{4pt} f(c) = k 
\end{align*}
\end{theorem}

\begin{exercise}
For the following $f$, show that there exists some $c \in (1, 2)$ such that $f(c)=0$
\begin{align*}
    f(x) = x^{4} + x - 3
\end{align*}
\end{exercise}

\begin{exercise}
For the following $f$, show that there exists some $c \in (0, 1)$ such that $f(c)=0$
\begin{align*}
    f(x) = 1 - x - \sqrt[\leftroot{2}\uproot{2}3]{x}
\end{align*}
\end{exercise}

\begin{exercise}
For the following $f$, show that there exists some $c \in (0, 1)$ such that $f(c)=0$
\begin{align*}
    f(x) = \cos x - x
\end{align*}
\end{exercise}

\begin{exercise}
For the following $f$, show that there exists some $c \in (1, 2)$ such that $f(c)=0$
\begin{align*}
    f(x) = \ln x - e^{-x}
\end{align*}
\end{exercise}

