\section{Power Series}

\begin{definition}
Given a sequence $\{c_{n}\}_{n=0}^{\infty}$ of real numbers and
\begin{align*}
    f(x) = \sum_{n=0}^{\infty} c_{n} (x - a)^{n}
\end{align*}
where $a$ is a real number, we say $f$ is a power series centered at $a$.
\end{definition}

\begin{note}
The reader may replace the word ``radius" in this section with the words ``open interval", but in general, ``radius" means ``open ball", whether that ``open ball" be a one-dimensional open interval, a two-dimensional open disk, a three-dimensional open sphere, etc. Thus, for any natural number $n$, one may refer to a ``radius" in $n$-dimensional space as an ``open ball".
\end{note}

Function $f$ will have a radius of convergence for some $x$, since
\begin{align*}
    f(x) &= c_{0} (x - a)^{0} + c_{1} (x - a)^{1} + c_{2} (x - a)^{2} + \cdots\\[2ex]
    &= c_{0} + c_{1} (x - a) + c_{2} (x - a)^{2} + \cdots
\end{align*}
which becomes the finite number, $c_{0}$, as soon as $x = a$. Our job is to find this radius of convergence and the object to which the series converges. 

\begin{note}
A series converges if its sequence of partial sums converges.
\end{note}

We begin with finding the radius of convergence, for which there are three possible outcomes
\begin{itemize}
    \item $f$ converges for $x \in \{a\}$
    \item $f$ converges for $x \in (-\infty, \infty)$
    \item $f$ converges for $x \in (-R, R)$, where $R$ is finite. 
\end{itemize}

\begin{example}
Take the function
\begin{align*}
    f(x) = \sum_{n=0}^{\infty} x^{n} = 1 + x + x^{2} + x^{3} + \cdots = \dfrac{1}{1-x}
\end{align*}
which we discovered is convergent when $x \in (-1, 1)$. Function $f$ is the power series centered about $a=0$. Using the ratio test
\begin{align*}
    \lim_{n \longrightarrow \infty} \Big\lvert \dfrac{x^{n+1}}{x^{n}} \Big\rvert = \lim_{n \longrightarrow \infty} \lvert x \rvert = \lvert x \rvert < 1 \hspace{20pt} \text{if and only if} \hspace{20pt} -1 < x < 1
\end{align*}
\label{example_1_power_series}
\end{example}

By Example \ref{example_1_power_series}, we have that, for any $-1 < x < 1$, $f$ converges absolutely. We may plug in the endpoints for this interval to determine the interval of convergence. Sometimes, these endpoints may be included in the interval of convergence by way of conditional convergence.

\begin{example}
Consider the power series
\begin{align*}
    f(x) = \sum_{n = 1}^{\infty} \dfrac{x^{n}}{n}
\end{align*}
Finding the radius of convergence through the ratio test
\begin{align*}
    \lim_{n \longrightarrow \infty}\Big\lvert \dfrac{x^{n+1}}{n+1} \cdot \dfrac{n}{x^{n}} \Big\rvert = \lim_{n \longrightarrow \infty} \Big\lvert \dfrac{x \cdot n}{n+1} \Big\rvert = \lvert x \rvert < 1
\end{align*}
we have that $f$ is absolute convergent for all $-1 < x < 1$. Now we test the endpoints $\{-1, 1\}$. We know from Example \ref{example_3_conditional_convergence} that
\begin{align*}
    f(-1) = \sum_{n = 1}^{\infty} \dfrac{(-1)^{n}}{n}
\end{align*} 
converges conditionally. Meaning, by Definition \ref{conditional_convergence}, 
\begin{align*}
    f(1) = \sum_{n = 1}^{\infty} \dfrac{1^{n}}{n} = \sum_{n = 1}^{\infty} \Big\lvert \dfrac{(-1)^{n}}{n} \Big\rvert
\end{align*}
is not convergent. Thus, the interval of convergence is $[-1 , 1)$, with $-1 < x < 1$ being absolutely convergent and $\{-1\}$ being conditionally convergent. 
\end{example}

\begin{theorem}
Suppose
\begin{align*}
    f(x) = \sum_{n=0}^{\infty} c_{n} (x-a)^{n}
\end{align*}
converges on $\lvert x - a \rvert < R$. Then $f$ converges on $(a-R, a+R)$, $f$ is continuous and differentiable on $(a-R, a+R)$, and 
\begin{align*}
    f^{'}(x) = \sum_{n=0}^{\infty} n c_{n} x^{n-1}
\end{align*}
\label{derivative_of_power_series}
\end{theorem}

Since $f$ is continuous on $(a-R, a+R)$, we have that $f$ is integrable on $(a-R, a+R)$.

\begin{theorem}
Let
\begin{align*}
    f(x) = \sum_{n=0}^{\infty} c_{n} (x-a)^{n}
\end{align*}
Then $f$ has derivatives of all orders on $(a-R, a+R)$, represented as 
\begin{align*}
    f^{(k)}(x) = \sum_{n=k}^{\infty} n(n-1) \cdots (n-(k-1)) c_{n} (x-a)^{n-k}
\end{align*}
Particularly, we have
\begin{align*}
    f^{(k)}(a) = k!c_{k} \hspace{20pt} \text{for all} \hspace{10pt} k \in \{0, 1, 2, 3, \cdots \}
\end{align*}
\end{theorem}

What follows is known as Taylor's Inequality.

\begin{theorem}
Let $n$ be a natural number, let $f$ be a function such that $f$ and all of its derivatives are continuous on $[\alpha, \beta]$ and $f^{(n+1)}$ exists on $(\alpha, \beta)$. If $a$ is in $[\alpha, \beta]$, then for any $x$ in $[\alpha, \beta]$, there exists a point $c$ between $x$ and $a$ such that
\begin{align*}
    f(x) = f(a) + f^{(1)}(a)(x-a) + \dfrac{f^{(2)}(a)}{2!}(x-a)^{2} + \cdots + \dfrac{f^{(n)}(a)}{n!}(x-a)^{n} + \dfrac{f^{(n+1)}(c)}{(n+1)!}(x-a)^{n+1}
\end{align*}
where
\begin{align*}
    f(a) + f^{(1)}(a)(x-a) + \dfrac{f^{(2)}(a)}{2!}(x-a)^{2} + \cdots + \dfrac{f^{(n)}(a)}{n!}(x-a)^{n}
\end{align*}
is the $n^{th}$-term Taylor polynomial in the sequence
\begin{align*}
    \Big\{\sum_{i=0}^{k} \dfrac{f^{(i)}(a)}{i!}(x-a)^{i}\Big\}_{k=0}^{\infty}
\end{align*}
which we may abbreviate as $T_{n}(x)$, and 
\begin{align*}
    \dfrac{f^{(n+1)}(c)}{(n+1)!}(x-a)^{n+1}
\end{align*}
is referred to as the remainder, which we may abbreviate as $R_{n}(x)$.
\label{taylors_theorem}
\end{theorem}


We refer to $f$ as the Taylor Series and can write it as
\begin{align*}
    f(x) = \sum_{n=0}^{\infty} \dfrac{f^{(n)}(a)}{n!}(x-a)^{n} \hspace{20pt} \text{for} \hspace{4pt} x \in (-R + a, R + a)
\end{align*}
if and only if
\begin{align*}
    \lim_{n \longrightarrow \infty} R_{n}(x) = 0 \hspace{20pt} \text{for each} \hspace{4pt} x \in (-R + a, R + a)
\end{align*}
We may obtain this limit if there exists a bound on $f^{(n)}(a)$ in a sufficiently small enough window of $a$. Using the limit
\begin{align*}
    \lim_{n \longrightarrow \infty} \dfrac{x^{n}}{n!} = 0
\end{align*}
we have the following
\begin{align*}
    &\text{If} \hspace{10pt} \lvert f^{(n+1)}(x) \rvert \leq M \hspace{10pt} \text{on} \hspace{10pt} \lvert x - a \rvert \leq \delta \\[2ex]
    &\text{then} \hspace{10pt} \lvert R_{n}(x) \rvert \leq \dfrac{M}{(n+1)!} \lvert x - a \rvert^{n+1} \hspace{10pt} \text{on} \hspace{10pt} \lvert x - a \rvert < \delta
\end{align*}

\begin{note}
Each $T_{n}(x)$ is a partial sum and is referred to as the $n^{th}$ Taylor polynomial.
\end{note}

\begin{note}
From the definition of a power series to the Taylor polynomial, we see
\begin{align*}
    c_{n} = \dfrac{f^{(n)}(a)}{n!}
\end{align*}
\end{note}

\begin{example}
The Taylor series for $f(x) = e^{x}$ centered at $0$ can be found as follows. We have the general setup
\begin{align*}
    e^{x} = f(x) = c_{0} + c_{1} (x - a) + c_{2} (x - a)^{2} + c_{3} (x - a)^{3} + c_{4} (x - a)^{4} + \cdots 
\end{align*}
Knowing that $a = 0$,
\begin{align*}
    e^{x} = f(x) = c_{0} + c_{1}x + c_{2}x^{2} + c_{3}x^{3} + c_{4}x^{4} + \cdots 
\end{align*}
We may use Theorem \ref{taylors_theorem} to write out the terms of this series, or we may evaluate each derivative at $0$ to establish a pattern
\begin{align*}
    &1 = e^{0} = f(0) = c_{0} + 0 + 0 + \cdots = c_{0} \hspace{20pt} \Longleftrightarrow \hspace{20pt} c_{0} = 1\\[2ex]
    &1 = e^{0} = f^{(1)}(0) = 1 \cdot c_{1} + 0 + 0 + \cdots = 1!c_{1} \hspace{20pt} \Longleftrightarrow \hspace{20pt} c_{1} = \dfrac{1}{1!}\\[2ex]
    &1 = e^{0} = f^{(2)}(0) = 1 \cdot 2 \cdot c_{2} + 0 + 0 + \cdots = 2!c_{2} \hspace{20pt} \Longleftrightarrow \hspace{20pt} c_{2} = \dfrac{1}{2!}\\[2ex]
    &1 = e^{0} = f(0) = 1 \cdot 2 \cdot 3 \cdot c_{3} + 0 + 0 + \cdots = 3!c_{3} \hspace{20pt} \Longleftrightarrow \hspace{20pt} c_{3} = \dfrac{1}{3!}\\[1ex]
    &\cdots\\[1ex]
    &\cdots\\[1ex]
    &\cdots
\end{align*}
Now we may rewrite the series
\begin{align*}
    e^{x} = f(x) &= 1 + \dfrac{1}{1!}x + \dfrac{1}{2!}x^{2} + \dfrac{1}{3!}x^{3} + \dfrac{1}{4!}x^{4} + \cdots = \sum_{n=0}^{\infty} \dfrac{1}{n!}x^{n}
\end{align*}
We prove that $e^{x}$ may be represented by this Taylor series by showing $lim_{n \longrightarrow \infty}R_{n}(x) = 0$. For any positive integer $n$
\begin{align*}
    \lvert R_{n}(x) \rvert \leq \dfrac{1}{(n+1)!} \lvert x \rvert^{n+1} < \dfrac{e^{\delta}}{(n+1)!} \lvert x \rvert^{n+1} < \delta \hspace{20pt} \text{since} \hspace{10pt} \delta > 0
\end{align*}
By Squeeze Theorem, we have our desired result.
\end{example}


\begin{example}
The Taylor series for $f(x) = \dfrac{1}{1-x}$, centered at $0$, can be found from the general setup of a power series for this function
\begin{align*}
    \dfrac{1}{1-x} = f(x) = c_{0} + c_{1}x + c_{2}x^{2} + c_{3}x^{3} + c_{4}x^{4} + \cdots
\end{align*}
as follows
\begin{align*}
    &1 = \dfrac{1}{1-0} = f(0) = c_{0} + 0 + 0 + \cdots = c_{0} \hspace{20pt} \Longleftrightarrow \hspace{20pt} c_{0} = 1\\[2ex]
    &1! = \dfrac{(-1)}{(1-0)^{2}}(-1) = f^{(1)}(0) = 1 \cdot c_{1} + 0 + 0 + \cdots = 1!c_{1} \hspace{20pt} \Longleftrightarrow \hspace{20pt} c_{1} = 1\\[2ex]
    &2! = \dfrac{(-1)(-2)}{(1-0)^{3}}(-1)(-1) = f^{(2)}(0) = 1 \cdot 2 \cdot c_{2} + 0 + 0 + \cdots = 2!c_{2} \hspace{20pt} \Longleftrightarrow \hspace{20pt} c_{2} = 1\\[2ex]
    &3! = \dfrac{(-1)(-2)(-3)}{(1-0)^{4}}(-1)(-1)(-1) = f(0) = 1 \cdot 2 \cdot 3 \cdot c_{3} + 0 + 0 + \cdots = 3!c_{3} \hspace{20pt} \Longleftrightarrow \hspace{20pt} c_{3} = 1\\[1ex]
    &\cdots\\[1ex]
    &\cdots\\[1ex]
    &\cdots
\end{align*}
which gives us
\begin{align*}
    \dfrac{1}{1-x} = f(x) = 1 + x + x^{2} + x^{3} + \cdots = \sum_{n=0}^{\infty} \dfrac{f^{(n)}(0)x^{n}}{n!} = \sum_{n=0}^{\infty} x^{n}
\end{align*}
as expected.
\end{example}

\begin{exercise}
Find the Taylor series for $f(x) = \sin x$ centered about $a = 0$. Prove that this series represents $\sin x$ for all $x$.
\end{exercise}

\begin{exercise}
Find the Taylor series for $f(x) = \cos x$ centered about $a = 0$. Prove that this series represents $\cos x$ for all $x$.
\end{exercise}

\begin{exercise}
Find the Taylor series $f(x) = x \cos x$ centered about $a = 0$.
\end{exercise}

\begin{exercise}
Find the Taylor series $f(x) = \sin x$ centered about $a = \pi/3$.
\end{exercise}

\begin{exercise}
Find the Taylor series for $f(x) = \sin \pi x$ centered about $a = 0$. Find the radius of convergence. Prove that this series represents $\sin \pi x$ for all $x$.
\end{exercise}

\begin{exercise}
Find the Taylor series for $f(x) = \cos 3x$ centered about $a = 0$. Find the radius of convergence.
\end{exercise}

\begin{exercise}
Find the Taylor series for $f(x) = e^{5x}$ centered about $a = 0$. Find the radius of convergence.
\end{exercise}

\begin{exercise}
Find the Taylor series for $f(x) = xe^{x}$ centered about $a = 0$. Find the radius of convergence.
\end{exercise}

\begin{exercise}
Find the Taylor series for $f(x) = \sin x$ centered about $a = \pi/2$. Find the radius of convergence. Prove that this series represents $\sin x$ for all $x$.
\end{exercise}

\begin{example}
Regarding Theorem \ref{derivative_of_power_series}, take the following function
\begin{align*}
    f(x) = \dfrac{1}{1-x}
\end{align*}
Expressing $f$ as a series, we have
\begin{align*}
    \dfrac{1}{1-x} = 1 + x + x^{2} + x^{3} + \cdots = \sum_{n=0}^{\infty} x^{n}
\end{align*}
Differentiating $f$ gives us
\begin{align*}
    \dfrac{1}{(1-x)^{2}} = 1 + 2x + 3x^{2} + \cdots = \sum_{n=1}^{\infty} nx^{n-1}
\end{align*}
The radius of convergence for this function, $f^{'}$, is 
\begin{align*}
    \lim_{n \longrightarrow \infty} \Big\lvert \dfrac{(n+1)x^{n}}{nx^{n-1}}\Big\rvert = \lim_{n \longrightarrow \infty} \Big\lvert \dfrac{(n+1)x}{n}\Big\rvert = \lvert x \rvert < 1
\end{align*}
which is the same result we found for $f$ earlier within this section. While we have our radius of convergence, sometimes it is requested that we find the interval of convergence. 
\begin{align*}
    \sum_{n=1}^{\infty} nx^{n-1} = \dfrac{1}{(1-x)^{2}} \hspace{20pt} \text{converges for} \hspace{4pt} -1 < x < 1 \hspace{20pt} \text{(and possibly $\{-1, 1\}$)}
\end{align*}
We test those two possible points by plugging each one in for argument $x$ and then running tests on the resulting series. For $x = -1$,
\begin{align*}
    \sum_{n=1}^{\infty} n(-1)^{n-1}
\end{align*}
we see the series diverges, since
\begin{align*}
    \lim_{n \longrightarrow \infty} n(-1)^{n-1} \neq 0
\end{align*}
Likewise, for $x = 1$,
\begin{align*}
    \sum_{n=1}^{\infty} n(1)^{n-1} = \sum_{n=1}^{\infty} n
\end{align*}
diverges, since
\begin{align*}
    \lim_{n \longrightarrow \infty} n \neq 0
\end{align*}
Thus, the interval of convergence for $f^{'}$ would be $(-1, 1)$.
\end{example}

\newpage
\section{$n^{\text{th}}$-term Taylor Polynomials}

The remainder term of a Taylor series, under condition that $\lvert f^{(n+1)}(x) \rvert \leq M$ on $\lvert x - a \rvert \leq \delta$, may be bounded as follows   
\begin{align*}
    \lvert R_{n}(x) \rvert \leq \dfrac{M}{(n+1)!} \lvert x - a \rvert^{n+1}
\end{align*}
Since Taylor's Inequality, Theorem \ref{taylors_theorem}, states
\begin{align*}
    f(x) &= f(a) + f^{(1)}(a)(x-a) + \dfrac{f^{(2)}(a)}{2!}(x-a)^{2} + \cdots + \dfrac{f^{(n)}(a)}{n!}(x-a)^{n} + \dfrac{f^{(n+1)}(c)}{(n+1)!}(x-a)^{n+1}\\[2ex] 
    &= T_{n}(x) + R_{n}(x) \hspace{20pt} \text{we have}\\[2ex]
    &\lvert f(x) - T_{n}(x) \rvert = \lvert R_{n}(x) \rvert \leq \dfrac{M}{(n+1)!} \lvert x - a \rvert^{n+1}
\end{align*}
which allows us to approximate $f$ by $T_{n}$ to an accuracy of our choosing.

\begin{example}
Take the function
\begin{align*}
    f(x) = (1+x)^{k} \hspace{20pt} \lvert x \rvert < 1
\end{align*}
This is the binomial series, centered at $0$, which expands by the definition
\begin{align*}
    f(x) = \sum_{n=0}^{\infty} \binom{k}{n} x^{n} &= \sum_{n=0}^{\infty} \dfrac{k(k-1)(k-2) \cdots (k-n+1)}{n!} x^{n}\\[2ex]
    &= 1 + kx + \dfrac{k(k-1)x^{2}}{2!} + \cdots
\end{align*}
If we set restrictions on both $x$ and $k$, say
\begin{align*}
    0 < x < 0.01 \hspace{20pt} \text{and} \hspace{20pt} 0 < k < 30
\end{align*}
we could set a bound on $f^{(2)}(0)$, say
\begin{align*}
    \lvert f^{(2)}(0) \rvert \leq 30^{2}
\end{align*}
Then we would get
\begin{align*}
    \lvert R_{1}(x) \rvert \leq \dfrac{30^{2}\lvert x \rvert^{2}}{2!} \leq \dfrac{0.09}{2!} = 0.045
\end{align*}
which gives us
\begin{align*}
    (1+x)^{k} \approx 1 + kx
\end{align*}
\end{example}

\begin{theorem}
    \label{taylor_consequence_0}
    Let $f$ be a function such that 
    \begin{align*}
        f(a) \hspace{2pt} , \hspace{4pt} f^{(1)}(a) \hspace{2pt} , \hspace{4pt} \dots \hspace{2pt} , \hspace{4pt} f^{(n)}(a)
    \end{align*}
    all exist. Let the following polynomial be the Taylor approximation, $T_{n}(x)$, of $f$ at $a$
    \begin{align*}
        T_{n}(x) \hspace{2pt} = \hspace{2pt} f(a) + f^{(1)}(a)(x - a) + \dfrac{f^{(2)}(a)}{2!}(x - a)^{2} + \cdots + \dfrac{f^{(n)}(a)}{n!}(x - a)^{n}
    \end{align*}
    Then we have
    \begin{align*}
        \lim_{x \longrightarrow a} \dfrac{f(x) - T_{n}(x)}{(x - a)^{n}} \hspace{2pt} = \hspace{2pt} 0
    \end{align*}
    \begin{proof}
        Take
        \begin{align*}
            \dfrac{f(x) - T_{n}(x)}{(x - a)^{n}} \hspace{2pt} &= \hspace{2pt} \dfrac{f(x) - \Big(f(a) + \cdots + \dfrac{f^{(n-1)}(a) (x - a)^{n-1}}{(n - 1)!} \Big)}{(x - a)^{n}} - \dfrac{f^{(n)}(a)}{n!} \\[2ex]
            &= \hspace{2pt} \dfrac{f(x) - Q(x)}{g(x)} - \dfrac{f^{(n)}(a)}{n!} 
        \end{align*}
        We can take derivatives of $Q$ and evaluate those derivative at $a$, as follows
        \begin{align*}
            Q^{(k)}(a) \hspace{2pt} = \hspace{2pt} f^{(k)}(a) \hspace{10pt} k \in \{1, 2, \dots , n-1\}
        \end{align*}
        and we know that
        \begin{align*}
            g^{(k)}(x) \hspace{2pt} = \hspace{2pt} \dfrac{n! \hspace{2pt} (x - a)^{n - k}}{(n - k)!}
        \end{align*}
        Additionally, we have that
        \begin{align*}
            \lim_{x \longrightarrow a} Q^{(k)}(x) \hspace{2pt} = \hspace{2pt} 0 \hspace{10pt} k \in \{1, 2, \dots , n-2\}
        \end{align*}
        and
        \begin{align*}
            \lim_{x \longrightarrow a} g^{(k)}(x) \hspace{2pt} = \hspace{2pt} 0 \hspace{10pt} k \in \{1, 2, \dots , n-2\}
        \end{align*}
        Finally, we know that 
        \begin{align*}
            Q^{(n-1)}(x) \hspace{2pt} = \hspace{2pt} f^{(n-1)}(a)
        \end{align*}
        So, using L'Hospital's Rule, iteratively, with the $\dfrac{0}{0}$ indeterminate form, we have
        \begin{align*}
            \lim_{x \longrightarrow a} \dfrac{f(x) - Q(x)}{g(x)} \hspace{2pt} &= \hspace{2pt} \lim_{x \longrightarrow a} \dfrac{f^{(1)}(x) - Q^{(1)}(x)}{g^{(1)}(x)} \hspace{2pt} = \hspace{2pt} \cdots \hspace{2pt} = \hspace{2pt} \lim_{x \longrightarrow a} \dfrac{f^{(n-1)}(x) - Q^{(n-1)}(x)}{g^{(n-1)}(x)} \\[2ex]
            &= \hspace{2pt} \lim_{x \longrightarrow a} \dfrac{f^{(n-1)}(x) - f^{(n-1)(a)}}{\dfrac{n! \hspace{2pt} (x - a)}{1!}} \hspace{2pt} = \hspace{2pt} \lim_{x \longrightarrow a} \dfrac{f^{(n-1)}(x) - f^{(n-1)}(a)}{n! (x - a)} \hspace{2pt} = \hspace{2pt} \dfrac{f^{(n)}(a)}{n!}
        \end{align*}
        So, we have our result
        \begin{align*}
            \lim_{x \longrightarrow a} \dfrac{f(x) - T_{n}(x)}{(x-a)^{n}} \hspace{2pt} &= \hspace{2pt} \lim_{x \longrightarrow a} \dfrac{f(x) - Q(x)}{g(x)} - \dfrac{f^{(n)}(a)}{n!} \hspace{2pt} = \hspace{2pt} \lim_{x \longrightarrow a} \dfrac{f^{(n-1)}(x) - f^{(n-1)}(a)}{n!(x - a)} - \dfrac{f^{(n)}(a)}{n!} \\[2ex]
            &= \dfrac{f^{(n)}(a)}{n!} - \dfrac{f^{(n)}(a)}{n!} \hspace{2pt} = \hspace{2pt} 0
        \end{align*}
    \end{proof}
\end{theorem}
