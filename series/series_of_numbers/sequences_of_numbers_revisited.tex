\section{Series of Numbers}

\begin{definition}
Take the sequence $\{f(n)\}_{n=1}^{\infty}$ and sum $k$ of its terms
\begin{align*}
    f(1) + f(2) + \cdots + f(k) = S_{k} \hspace{20pt} S_{k} \in \mathbb{R} \hspace{4pt} \cup \hspace{4pt} \{-\infty, \infty\}
\end{align*}
We can take a sequence of these sums as follows
\begin{align*}
    \{S_{k}\}_{k=1}^{\infty}
\end{align*}
with each $S_{k}$ referred to as a partial sum. Now we take the limit on these sequence
\begin{align*}
    \lim_{k \longrightarrow \infty} S_{k} = S
\end{align*}
This limit on the sequence $\{S_{k}\}_{k=1}^{\infty}$ we refer to as a series, and we denote this series as 
\begin{align*}
    S = \sum_{n=1}^{\infty} f(n)
\end{align*}
If $S$ is finite, then the series is convergent. If $S$ is infinite, then the series is divergent. There are some series that begin with $n=0$ as opposed to $n=1$. We will discover that the important results covered in this section focus less on where we begin in a series and more on where we end.
\end{definition}
Geometric\\
Integral Test\\
Comparison Tests\\
Alternating Series\\
Ratio, Root Tests\\

