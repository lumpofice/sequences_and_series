\section{Series of Numbers}

\begin{definition}
Take the sequence $\{f(n)\}_{n=1}^{\infty}$ and sum $k$ of its terms
\begin{align*}
    f(1) + f(2) + \cdots + f(k) = S_{k} \hspace{20pt} S_{k} \in \mathbb{R} \hspace{4pt} \cup \hspace{4pt} \{-\infty, \infty\}
\end{align*}
We can take a sequence of these sums as follows
\begin{align*}
    \{S_{k}\}_{k=1}^{\infty}
\end{align*}
with each $S_{k}$ referred to as a partial sum. Now we take the limit on these sequence
\begin{align*}
    \lim_{k \longrightarrow \infty} S_{k} = S
\end{align*}
This limit on the sequence $\{S_{k}\}_{k=1}^{\infty}$ we refer to as a series, and we denote this series as 
\begin{align*}
    S = \sum_{n=1}^{\infty} f(n)
\end{align*}
If $S$ is finite, then the series is convergent. If $S$ is infinite, then the series is divergent. There are some series that begin with $n=0$ as opposed to $n=1$. We will discover that the important results covered in this section focus less on where we begin in a series and more on where we end. In fact, there is a theorem for this:
\end{definition}

\begin{theorem}
$\sum f(n)$ converges if and only if for every $\epsilon > 0$ there is a natural number $N$ such that 
\begin{align*}
    \Big\lvert \sum_{n = k}^{m} f(n) \Big\rvert \leq \epsilon
\end{align*}
for all $k \geq m \geq N$. So, we have
\begin{align*}
    \Big\lvert \sum_{n = k}^{m} f(n) \Big\rvert = \lvert f(m) \rvert \leq \epsilon
\end{align*}
when $k = m$. This leads us to the following theorem:
\end{theorem}

\begin{theorem}
If $\sum f(n)$ converges, then $\lim_{n \longrightarrow \infty} f(n) = 0$
\end{theorem}

\begin{theorem}
If, for a fixed natural number $N$, $\lvert f(n) \rvert \leq g(n)$ for all $n \geq N$ then
\begin{align*}
    \sum f(n) \hspace{4pt} \text{converges} \hspace{20pt} \text{if} \hspace{20pt} \sum g(n) \hspace{4pt} \text{converges}
\end{align*}
If, for a fixed natural number $N$, $f(n) \geq g(n) \geq 0$ for all $n \geq N$ then
\begin{align*}
    \sum f(n) \hspace{4pt} \text{diverges} \hspace{20pt} \text{if} \hspace{20pt} \sum g(n) \hspace{4pt} \text{diverges}
\end{align*}
\end{theorem}

\begin{recall}
By Theorem \ref{geometric_term_sequence}, we have, for the sequence $\{r^{n}\}_{n=1}^{\infty}$
\begin{align*}
    \lim_{n \longrightarrow \infty} r^{n} = \begin{cases}
    0, \hspace{4pt} \text{if} \hspace{4pt} -1 < r < 1,\\[2ex]
    1, \hspace{4pt} \text{if} \hspace{4pt} r = 1
    \end{cases}
\end{align*}
\end{recall}

Integral Test\\
Comparison Tests\\
Alternating Series\\
Ratio, Root Tests\\

