\documentclass{article}
\usepackage[utf8]{inputenc}

% Standard math packages
\usepackage{amsmath}
\usepackage{amsthm}
\usepackage{amssymb}
\usepackage{amsfonts}

% Packages for images
\usepackage{tikz}

% Package for the layout
\usepackage{geometry}

% Title Page
\title{Calculus}
\author{Jonathan Parker}
\date{Last Updated on \today}

% Counters
\renewcommand*\contentsname{Table of Contents}

\newtheorem{theorem}{Theorem}[section]

\newtheorem{definition}{Definition}[section]

\newtheorem{note}{Note}[section]

\newcounter{example}[section]
\newenvironment{example}[1][]{\refstepcounter{example}\par\medskip
   \noindent \textbf{Example~\theexample. #1} \rmfamily}{\medskip}

\newcounter{exercise}[section]
\newenvironment{exercise}[1][]{\refstepcounter{exercise}\par\medskip
   \noindent \textbf{Exercise~\theexercise. #1} \rmfamily}{\medskip}
   
\newcounter{recall}[section]
\newenvironment{recall}[1][]{\refstepcounter{recall}\par\medskip
   \noindent \textbf{Recall~\therecall. #1} \rmfamily}{\medskip}
   
% Removing indentations
\setlength{\parindent}{0pt}


\begin{document}

\maketitle
\tableofcontents

\newpage
\documentclass{article}
\usepackage[utf8]{inputenc}
\usepackage{amsmath}
\usepackage{amsthm}
\usepackage{amssymb}
\usepackage{geometry}
\usepackage{amsfonts} 

\title{Sequences of Numbers}
\author{Jonathan Parker}
\date{Last Updated on \today}

\renewcommand*\contentsname{Table of Contents}

\newtheorem{theorem}{Theorem}[section]

\newtheorem{definition}{Definition}[section]

\newtheorem{note}{Note}[section]

\newcounter{example}[section]
\newenvironment{example}[1][]{\refstepcounter{example}\par\medskip
   \noindent \textbf{Example~\theexample. #1} \rmfamily}{\medskip}

\newcounter{exercise}[section]
\newenvironment{exercise}[1][]{\refstepcounter{exercise}\par\medskip
   \noindent \textbf{Exercise~\theexercise. #1} \rmfamily}{\medskip}
   
\newcounter{recall}[section]
\newenvironment{recall}[1][]{\refstepcounter{recall}\par\medskip
   \noindent \textbf{Recall~\therecall. #1} \rmfamily}{\medskip}
   
\setlength{\parindent}{0pt}

\begin{document}


\maketitle
\tableofcontents

\newpage
\section{Introduction to Sequences of Numbers}\label{introduction_to_sequences_of_numbers}

\begin{definition}
A sequence of real numbers is a mapping
\begin{align*}
    f: \mathbb{N} \longrightarrow \mathbb{R}: f(n) \mapsto a_{n}
\end{align*}
We typically write the sequence in the following concise way:
\begin{align*}
    \{f(n)\}_{n=1}^{\infty} \hspace{20pt} \text{or} \hspace{20pt} \{a_{n}\}_{n=1}^{\infty}
\end{align*}
where $a_{n}$ is the real number function value at index $n$.
\end{definition}

\begin{example}
\begin{align*}
    &\Big\{\dfrac{1}{n}\Big\}_{n=1}^{\infty}\\[2ex]
    = \hspace{4pt} &\Big\{\dfrac{1}{1}, \hspace{4pt} \dfrac{1}{2}, \hspace{4pt} \dfrac{1}{3}, \hspace{4pt} \cdots \Big\}
\end{align*}
Here, $f(n) = \dfrac{1}{n}$ is the function definition from the natural numbers $\mathbb{N}$ to the real numbers $\mathbb{R}$.
\end{example}

\begin{exercise}
Write the $5^{\text{th}}$ term for the following sequence of real numbers:
\begin{align*}
    \Big\{\dfrac{n}{n+1}\Big\}_{n=1}^{\infty}
\end{align*}
\end{exercise}

\begin{exercise}
Write the first five terms for the following sequence of real numbers:
\begin{align*}
    \Big\{\dfrac{(-1)^{n}(n+1)}{3^{n}}\Big\}_{n=1}^{\infty}
\end{align*}
\end{exercise}

Sometimes, the domain is extended or attenuated a bit to accommodate the function definition.

\begin{exercise}
Write the first five terms for the following sequence of real numbers:
\begin{align*}
    \{\sqrt{n-10}\}_{n=10}^{\infty}
\end{align*}
\end{exercise}

\begin{exercise}
Write the first five terms for the following sequence of real numbers:
\begin{align*}
    \Big\{\cos{\Big(\dfrac{n\pi}{4}\Big)}\Big\}_{n=0}^{\infty}
\end{align*}
\end{exercise}

A tricky part of this concept is using a sequence's function values to determine the general formula.

\begin{example}
Let's take the following sequence of terms:
\begin{align*}
    \Big\{\dfrac{3}{5}, \hspace{4pt} -\dfrac{4}{25}, \hspace{4pt} \dfrac{5}{125}, \hspace{4pt} -\dfrac{6}{625}, \hspace{4pt} \dfrac{7}{3125}, \hspace{4pt} \cdots \Big\}
\end{align*}
Our job is to discover a pattern. Well, it seems every other term is negative. We know this can be established with the factor $(-1)^{n}$, but given that there are other factors we've yet to analyze thoroughly, it's not certain that this is a factor in our general expression for the sequence. It also seems the numerator begins at $3$ and increments by one unit as the sequence progresses. Given that the sequence begins at $n=3$, will our first term in the sequence be positive? We see $(-1)^{3}$ is negative. So, we fix this by adding or subtracting a $1$ from the power $n$. While it is not always the case that either adding or subtracting a $1$ will work, in this example, either approach will. So, we add a $1$, giving us $(-1)^{n+1}$ as a factor in our general sequence. Finally, the denominator seems to be some power of $5$ in each term of the sequence, and it seems those powers begin at $1$, which is equal to $3-2$. Checking the power in our second term, we have a power of $2$, which is equal to $4-2$. It seems our powers progress through the expression $n-2$, where $n$ is the index of the sequence. Thus, we have the following general expression for our sequence:
\begin{align*}
    \Big\{\dfrac{(-1)^{n+1}n}{5^{n-2}}\Big\}_{n=3}^{\infty}
\end{align*}
\end{example}

\begin{exercise}
Find the general expression of the sequence following the pattern:
\begin{align*}
    \Big\{1, \hspace{4pt} -\dfrac{2}{3}, \hspace{4pt} \dfrac{4}{9}, \hspace{4pt} -\dfrac{8}{27}, \hspace{4pt} \cdots \Big\}
\end{align*}
\end{exercise}

\newpage
\section{Limits of Sequences of Numbers}\label{limits_of_sequences_of_numbers}

\begin{definition}
A sequence $\{f(n)\}_{n=1}^{\infty}$ has a limit $L$, which we denote as 
\begin{align*}
    \lim_{n \longrightarrow \infty} f(n) = L\\[2ex]
    \text{if for all} \hspace{4pt} \epsilon > 0 \hspace{4pt} \text{there exists a natural number} \hspace{4pt} &N \hspace{4pt} \text{such that for all} \hspace{4pt} n \geq N \hspace{4pt} \text{we have}\\[2ex]
    \lvert f(n) - L \rvert < \epsilon
\end{align*}
Any sequence with a limit can be referred to as convergent.
\label{definition_limit_sequence_numbers}
\end{definition}

\begin{example}
$\lim_{n \longrightarrow \infty} \dfrac{1}{n} = 0$
\begin{proof}
Take $\epsilon = \dfrac{1}{k}$, where $k \in \mathbb{N}$ is arbitrary. Then by Definition \ref{definition_limit_sequence_numbers} we have
\begin{align*}
    &\Big\lvert \dfrac{1}{k+n} - 0 \Big\rvert\\[2ex]
    &= \Big\lvert \dfrac{1}{k+n} \Big\rvert\\[2ex]
    &< \dfrac{1}{k} = \epsilon, \hspace{4pt} \text{for all} \hspace{4pt} n \in \mathbb{N}
\end{align*}
\end{proof}
\label{limit_one_over_n}
\end{example}

\begin{theorem}
If $\{f(n)\}_{n=1}^{\infty}$ and $\{g(n)\}_{n=1}^{\infty}$ are convergent sequences, and if $c \in \mathbb{R}$, then
\begin{align*}
    &\lim_{n \longrightarrow \infty} (f(n) + g(n)) = \lim_{n \longrightarrow \infty} f(n) + \lim_{n \longrightarrow \infty} g(n) \\[2ex]
    &\lim_{n \longrightarrow \infty}(f(n) - g(n)) = \lim_{n \longrightarrow \infty} f(n) - \lim_{n \longrightarrow \infty} g(n)\\[2ex]
    &\lim_{n \longrightarrow \infty} cf(n) = c\lim_{n \longrightarrow \infty} f(n)\\[2ex]
    &\lim_{n \longrightarrow \infty} (f(n) \cdot g(n)) = \lim_{n \longrightarrow \infty} f(n) \cdot \lim_{n \longrightarrow \infty} g(n)\\[2ex]
    &\lim_{n \longrightarrow \infty}\dfrac{f(n)}{g(n)} = \dfrac{\lim_{n \longrightarrow \infty} f(n)}{\lim_{n \longrightarrow \infty} g(n)}, \hspace{4pt} \text{when} \hspace{4pt} \lim_{n \longrightarrow \infty} g(n) \neq 0\\[2ex]
    &\lim_{n \longrightarrow \infty} (f(n))^{p} = (\lim_{n \longrightarrow \infty} f(n))^{p}, \hspace{4pt} \text{where} \hspace{4pt} p > 0 \hspace{4pt} \text{and} \hspace{4pt} f(n) > 0\\[2ex]
    &\lim_{n \longrightarrow \infty} c = c
\end{align*}
\label{properties_limit_sequence_numbers}
\end{theorem}

Now that we have these result, we can use them to find limits of other sequences.

\begin{example}
$\lim_{n \longrightarrow \infty} \dfrac{n}{n+1} = 1$
\begin{proof}
We have a trick we can employ to find the above limit. By Example \ref{limit_one_over_n} and Theorem \ref{properties_limit_sequence_numbers},
\begin{align*}
    \dfrac{n}{n+1} &= 1 \cdot \dfrac{n}{n+1}
    = \dfrac{1/n}{1/n} \cdot \dfrac{n}{n+1}
    = \dfrac{(1/n) \cdot n}{(1/n) \cdot (n+1)}
    = \dfrac{1}{1 + (1/n)}\\[2ex]
    \Longrightarrow &\lim_{n \longrightarrow \infty} \dfrac{n}{n+1}
    = \lim_{n \longrightarrow \infty} \dfrac{1}{1+(1/n)}
    = \dfrac{\lim_{n \longrightarrow \infty}1}{\lim_{n \longrightarrow \infty}1 + \lim_{n \longrightarrow \infty}(1/n)}
    = \dfrac{1}{1 + 0} = 1
\end{align*}
\end{proof}
\end{example}

\begin{exercise}
Find the limit of 
\begin{align*}
    \Big\{\dfrac{n^{3}}{n^{3}+1}\Big\}_{n=1}^{\infty}
\end{align*}
\end{exercise}

\begin{exercise}
Find the limit of 
\begin{align*}
    \Big\{\dfrac{3+5n^{2}}{n+n^{2}}\Big\}_{n=1}^{\infty}
\end{align*}
\end{exercise}

\begin{exercise}
Find the limit of 
\begin{align*}
    \{e^{1/n}\}_{n=1}^{\infty}
\end{align*}
\end{exercise}

\begin{exercise}
Find the limit of
\begin{align*}
    \Big\{\tan\Big(\dfrac{2n\pi}{1+8n}\Big)\Big\}_{n=1}^{\infty}
\end{align*}
\end{exercise}

\begin{exercise}
Find the limit of
\begin{align*}
    \Big\{\sqrt{\dfrac{n+1}{9n+1}}\Big\}_{n=1}^{\infty}
\end{align*}
\end{exercise}

\begin{definition}
Take a sequence $\{f(n)\}_{n=1}^{\infty}$ 
\begin{align*}
    \text{If} \hspace{4pt} \lim_{n \longrightarrow \infty} f(n) &= \infty, \hspace{4pt} \text{then}\\[2ex]
    \text{for all} \hspace{4pt} N \in \mathbb{N}, \hspace{4pt} \text{there exists an} \hspace{4pt} M \in \mathbb{N} \hspace{4pt} &\text{such that for all} \hspace{4pt} n > M \hspace{4pt} \text{we have} \hspace{4pt} f(n) > N.
\end{align*}
\begin{align*}
    \text{If} \hspace{4pt} \lim_{n \longrightarrow \infty} f(n) &= -\infty, \hspace{4pt} \text{then}\\[2ex]
    \text{for all} \hspace{4pt} N \in \mathbb{N}, \hspace{4pt} \text{there exists an} \hspace{4pt} M \in \mathbb{N} \hspace{4pt} &\text{such that for all} \hspace{4pt} n > M \hspace{4pt} \text{we have} \hspace{4pt} f(n) < -N.
\end{align*}
Any sequence with an infinite limit is said to be divergent.
\label{definition_infinite_limit_sequence}
\end{definition}

\begin{example}
Take $\{n\}_{n=1}^{\infty}$. Clearly, as $n \longrightarrow \infty$, the sequence of function values push to infinity.
\end{example}

\begin{exercise}
Show the sequence $\Big\{\dfrac{n^{3}}{n+1}\Big\}_{n=1}^{\infty}$ has an infinite limit. 
\end{exercise}

\begin{exercise}
Show the sequence $\Big\{\dfrac{n!}{2^{n}}\Big\}_{n=1}^{\infty}$ has an infinite limit. 
\end{exercise}

\begin{theorem}
A sequence of numbers can have at most one limit.
\label{limit_uniqueness}
\end{theorem}

\begin{note}
Any sequence that does not converge to a single, finite value is considered divergent.
\end{note}

\begin{exercise}
Find the general expression of the divergent sequence following the pattern:
\begin{align*}
    \{5, \hspace{4pt} 1, \hspace{4pt} 5, \hspace{4pt} 1, \hspace{4pt} 5, \hspace{4pt} 1, \hspace{4pt} \cdots\}
\end{align*}
\end{exercise}

\begin{exercise}
Find the general expression of the divergent sequence following the pattern:
\begin{align*}
    \{2, \hspace{4pt} 7, \hspace{4pt} 12, \hspace{4pt} 17, \hspace{4pt} \cdots\}
\end{align*}
\end{exercise}

\begin{exercise}
Determine whether the sequence defined by the following is convergent or divergent:
\begin{align*}
    f(1) = 1 \hspace{20pt} f(n) = 4 - f(n-1), \hspace{20pt} n \geq 1
\end{align*}
\end{exercise}

\begin{exercise}
Determine whether the sequence defined by the following is convergent or divergent:
\begin{align*}
    f(1) = 2 \hspace{20pt} f(n) = 4 - f(n-1), \hspace{20pt} n \geq 1
\end{align*}
\end{exercise}

\begin{recall}
For all $a, b \in \mathbb{R}$, we have
\begin{align*}
    \Big\lvert \lvert a \rvert - \lvert b \rvert \Big\rvert &\leq \lvert a - b \rvert \\[2ex]
    \lvert a + b \rvert &\leq \lvert a \rvert + \lvert b \rvert
\end{align*}
\label{triangle_inequality}
\end{recall}

\begin{theorem}
Let $\{f(n)\}_{n=1}^{\infty}, \hspace{4pt} \{g(n)\}_{n=1}^{\infty}$ and $\{h(n)\}_{n=1}^{\infty}$ be sequences. If
\begin{align*}
    \text{If} \hspace{4pt} f(n)\leq g(n)\leq h(n) \hspace{4pt} \text{for all} \hspace{4pt} n \in \mathbb{N} \hspace{20pt}
    & \text{and if}  \hspace{20pt} \lim_{n \longrightarrow \infty} f(n) = \lim_{n \longrightarrow \infty} g(n) = L \\[2ex]
    & \text{then} \hspace{4pt} \lim_{n \longrightarrow \infty} h(n) = L 
\end{align*}
\label{squeeze_theorem}
\end{theorem}

\begin{exercise}
Show the following two different ways:
\begin{itemize}
    \item [1.] Using the triangle inequality from Recall \ref{triangle_inequality}
    \item [2.] Using Theorem \ref{squeeze_theorem}
\end{itemize}
\begin{align*}
    \lim_{n \longrightarrow \infty} \dfrac{(-1)^{n-1}n}{n^{2}+1} = 0
\end{align*}
\end{exercise}

\begin{exercise}
Prove the following theorem two different ways:
\begin{itemize}
    \item [1.] Using the triangle inequality from Recall \ref{triangle_inequality}
    \item [2.] Using Theorem \ref{squeeze_theorem}
\end{itemize}
\begin{align*}
    \text{If} \hspace{4pt} \lim_{n \longrightarrow \infty} \lvert f(n) \rvert = L \hspace{20pt} \text{then} \hspace{20pt} \lim_{n \longrightarrow \infty} f(n) = L
\end{align*}
\end{exercise}

\begin{exercise}
Use Theorem \ref{squeeze_theorem} to find the limit of the following sequence:
\begin{align*}
    \Big\{\dfrac{\sin(n)}{n}\Big\}_{n=1}^{\infty}
\end{align*}
\end{exercise}

\begin{exercise}
Use Theorem \ref{squeeze_theorem} to find the limit of the following sequence:
\begin{align*}
    \Big\{\dfrac{\cos^{2}(n)}{2^{n}}\Big\}_{n=1}^{\infty}
\end{align*}
\end{exercise}

\begin{exercise}
Use Theorem \ref{squeeze_theorem} to find the limit of the following sequence:
\begin{align*}
    \Big\{\dfrac{\sin(2n)}{1+\sqrt{n}}\Big\}_{n=1}^{\infty}
\end{align*}
\end{exercise}

\begin{exercise}
Find the limit of the following sequence:
\begin{align*}
    \Big\{\dfrac{n!}{2^{n}}\Big\}_{n=1}^{\infty}
\end{align*}
\end{exercise}

\begin{exercise}
Find the limit of the following sequence:
\begin{align*}
    \Big\{\dfrac{(-3)^{2}}{n!}\Big\}_{n=1}^{\infty}
\end{align*}
\end{exercise}

\end{document}


\newpage
\section{Limits of Functions}

\begin{definition}
We say $f$ as a function of $x$ has a limit $L$ at $c \in \mathbb{R}$, denoted by
\begin{align*}
    &\lim_{x \longrightarrow c} f(x) = L \hspace{4pt} \text{if} \hspace{20pt} &&\text{for all} \hspace{4pt} \epsilon > 0, \hspace{4pt} \text{there exists a} \hspace{4pt} \delta > 0 \hspace{4pt} \text{such that}\\[2ex]
    &\text{for all} \hspace{4pt} x \in \text{Dom($f$)} \hspace{4pt} \text{satisfying} \hspace{4pt} \lvert x - c \rvert < \delta &&\text{we have} \hspace{4pt} \lvert f(x) - L \rvert < \epsilon
\end{align*}
\end{definition}

\begin{example}
Below is a visual of how this $\epsilon - \delta$ interplay occurs. In general, if $L \in \mathbb{R}$ is the limit of $f$ at $c \in \mathbb{R}$, then for an arbitrarily small, open interval $(L-\epsilon, L+\epsilon)$ with center $L$, there is an open interval $(c-\delta, c+\delta)$ with center $c$ containing members $x \in$ Dom($f$). With this specific example, $c=1/2$. For any $x \in$ Dom($f$) satisfying 
\begin{align*}
    \Big\lvert x - \dfrac{1}{2} \Big\rvert < \delta \hspace{20pt} \text{we have} \hspace{20pt} \lvert f(x) - L \rvert = \lvert \sin(x) - L \rvert < \epsilon \hspace{20pt} \Longleftrightarrow \hspace{20pt} \lim_{x \longrightarrow 1/2} \sin(x) = L
\end{align*}
which is equivalent to the following: For any $x \in$ Dom($f$) satisfying
\begin{align*}
    x \in \Big(\dfrac{1}{2} - \delta, \dfrac{1}{2} + \delta\Big) \hspace{20pt} \text{we have} \hspace{4pt} f(x) \in (L - \epsilon, L + \epsilon)
\end{align*}

\resizebox{30em}{30em}{%
\begin{tikzpicture}[scale=\textwidth/4.2cm]
    % title and axes
    \node at (0.7, 1.1) {\tiny$f(x)=\sin x \hspace{4pt} x \in \Big[0, \dfrac{1}{2}\Big) \cup \Big(\dfrac{1}{2}, 1 \Big]$};
    \draw (0, 0) -- (1, 0)
        node[right] {\tiny$x$};
    \draw (0, 0) -- (0, 1)
        node[above] {\tiny$f(x)$};
    % ----------------------------------
    % range boundaries lower
    \draw[dotted] (0, {sin(0.45 r)}) -- (0.6, {sin(0.45 r)});
    \node[] at (-0.15, {sin(0.45 r)}) {\tiny$L-\epsilon$};
    \node [rotate=90] at (0, {sin(0.45 r)}) {\tiny(};
    % range boundaries upper
    \draw[dotted] (0, {sin(0.55 r)}) -- (0.6, {sin(0.55 r)});
    \node[] at (-0.15, {sin(0.55 r)}) {\tiny$L+\epsilon$};
    \node [rotate=-90] at (0, {sin(0.55 r)}) {\tiny(};
    % ----------------------------------
    % domain boundaries lower
    \draw[dotted] (0.47, 0) -- (0.47, {sin(0.6 r)});
    \node [rotate=45] at (0.35, -0.15) {\tiny$\dfrac{1}{2}-\delta$ $\longrightarrow$};
    \node [] at (0.47, 0) {\tiny(};
    % domain boundaries upper
    \draw[dotted] (0.53, 0) -- (0.53, {sin(0.6 r)});
    \node [rotate=-45] at (0.65, -0.15) {\tiny$\longleftarrow$ \tiny$\dfrac{1}{2}+\delta$};
    \node [] at (0.53, 0) {\tiny)};
    % ----------------------------------
    % graph
    \draw[blue] plot[smooth] file {limits_of_functions/python_generated_tables/sine_0_1_piece_0.table};
    \draw[blue] plot[smooth] file {limits_of_functions/python_generated_tables/sine_0_1_piece_1.table};
    \draw[blue, fill=white] (0.5,{sin(0.5 r)}) circle (.25mm);
\end{tikzpicture}
}
\end{example}

\begin{exercise}
Find the following:
\begin{align*}
    \lim_{x \longrightarrow 1/2} \sin(x) \hspace{20pt} x \in [0, 1]
\end{align*}
\end{exercise}

\begin{theorem}
If $f(x) = x, \hspace{4pt} x \in \mathbb{R}$, then $\lim_{x \longrightarrow c} f(x)$ exists for all $c \in \mathbb{R}$. Because $f$ is defined on the entire real line, we have 
\begin{align*}
    \lim_{x \longrightarrow c} f(x) = f(c) = c
\end{align*}
\label{limit_identity_function}
\end{theorem}

\begin{theorem}
Suppose $c \in \mathbb{R}$ and suppose 
\begin{align*}
    \lim_{x \longrightarrow a} f(x) \hspace{20pt} \text{and} \hspace{20pt} \lim_{x \longrightarrow a} g(x)
\end{align*}
both exist. Then we have
\begin{align*}
    &\lim_{x \longrightarrow a} [f(x) + g(x)] = \lim_{x \longrightarrow a} f(x) + \lim_{x \longrightarrow a} g(x)\\[2ex]
    &\lim_{x \longrightarrow a} [f(x) - g(x)] = \lim_{x \longrightarrow a} f(x) - \lim_{x \longrightarrow a} g(x)\\[2ex]
    &\lim_{x \longrightarrow a} [c \cdot f(x)] = c \cdot \lim_{x \longrightarrow a} f(x)\\[2ex]
    &\lim_{x \longrightarrow a} [f(x) \cdot g(x)] = \lim_{x \longrightarrow a} f(x) \cdot \lim_{x \longrightarrow a} g(x)\\[2ex]
    &\lim_{x \longrightarrow a} \dfrac{f(x)}{g(x)} = \dfrac{\lim_{x \longrightarrow a} f(x)}{\lim_{x \longrightarrow a} g(x)} \hspace{20pt} \text{when} \hspace{4pt} \lim_{x \longrightarrow a} g(x) \neq 0
\end{align*}
\label{properties_limit_functions}
\end{theorem}

\begin{example}
Here is another example, taking the mathematical approach, using Theorems \ref{limit_identity_function}, \ref{properties_limit_functions}, to find the limit,
\begin{align*}
    \lim_{x \longrightarrow 1} \dfrac{x-1}{x^{2} - 1} = \lim_{x \longrightarrow 1} \dfrac{x-1}{(x-1)(x+1)} = \lim_{x \longrightarrow 1} \dfrac{1}{(x+1)} = \dfrac{1}{(1+1)} = \dfrac{1}{2}
\end{align*}

\resizebox{30em}{30em}{%
\begin{tikzpicture}[scale=\textwidth/6.2cm]
    % title and axes
    \node at (1.5, 2.1) {\tiny$f(x)=\dfrac{x-1}{x^{2}-1} \hspace{4pt} x \in \Big[\dfrac{1}{2}, 1\Big) \cup (1, 2]$};
    \draw (-0.5, 0) -- (2, 0)
        node[right] {\tiny$x$};
    \draw (0, 0) -- (0, 2)
        node[above] {\tiny$f(x)$};
    % ----------------------------------
    % range boundaries lower
    \draw[dotted] (0, 0.45) -- (1.05, 0.45);
    \node[rotate=45] at (-0.20, 0.28) {\tiny$\dfrac{1}{2}-\epsilon \longrightarrow$};
    \node [rotate=90] at (0, 0.45) {\tiny(};
    % range boundaries upper
    \draw[dotted] (0, 0.55) -- (1.05, 0.55);
    \node[rotate=-45] at (-0.20, 0.72) {\tiny$\dfrac{1}{2}+\epsilon \longrightarrow$};
    \node [rotate=-90] at (0, 0.55) {\tiny(};
    % ----------------------------------
    % domain boundaries lower
    \draw[dotted] (0.95, 0) -- (0.95, 0.55);
    \node [rotate=45] at (0.75, -0.2) {\tiny$1-\delta$ $\longrightarrow$};
    \node [] at (0.95, 0) {\tiny(};
    % domain boundaries upper
    \draw[dotted] (1.05, 0) -- (1.05, 0.55);
    \node [rotate=-45] at (1.25, -0.2) {\tiny$\longleftarrow$ \tiny$1+\delta$};
    \node [] at (1.05, 0) {\tiny)};
    % ----------------------------------
    % graph
    \draw[blue] plot[smooth] file {limits_of_functions/python_generated_tables/rational_neg1half_2_piece_0.table};
    \draw[blue] plot[smooth] file {limits_of_functions/python_generated_tables/rational_neg1half_2_piece_1.table};
    \draw[blue, fill=white] (1,0.5) circle (.25mm);
\end{tikzpicture}
}
\end{example}

\begin{exercise}
Find the following:
\begin{align*}
    \lim_{x \longrightarrow 1} g(x) \hspace{20pt} \text{when} \hspace{20pt} g(x) = 
    \begin{cases}
    \dfrac{x-1}{x^{2}-1}, \hspace{4pt} &x \neq 1\\[2ex]
    2, \hspace{4pt} &x = 1
    \end{cases}
\end{align*}
\end{exercise}

\begin{exercise} Find the following:
\begin{align*}
    \lim_{h \longrightarrow 0} \dfrac{(3+h)^{2} - 9}{h}
\end{align*}
\end{exercise}

\begin{exercise}
Find the following:
\begin{align*}
    \lim_{x \longrightarrow 2} \dfrac{x^{2}+x-6}{x-2}
\end{align*}
\end{exercise}

\begin{exercise}
Find the following:
\begin{align*}
    \lim_{x \longrightarrow 2} \dfrac{x^{2}-2x}{x^{2}-x-2}
\end{align*}
\end{exercise}

\begin{definition}
We say a function $f$ has a right-sided limit
\begin{align*}
    \lim_{x \longrightarrow c^{+}} f(x) = L
\end{align*}
if for all $\epsilon > 0$, there exists a $\delta > 0$ such that
\begin{align*}
    \lvert x - c \rvert < \delta \hspace{20pt} \Longrightarrow \hspace{20pt} \lvert f(x) - L \rvert < \epsilon
\end{align*}
when $x \in \text{Dom($f$)} \cap (c, \infty)$.
\end{definition}

\begin{definition}
We say a function $f$ has a left-sided limit
\begin{align*}
    \lim_{x \longrightarrow c^{-}} f(x) = L
\end{align*}
if for all $\epsilon > 0$, there exists a $\delta > 0$ such that
\begin{align*}
    \lvert x - c \rvert < \delta \hspace{20pt} \Longrightarrow \hspace{20pt} \lvert f(x) - L \rvert < \epsilon
\end{align*}
when $x \in \text{Dom($f$)} \cap (-\infty, c)$.
\end{definition}

\begin{theorem}
We say function $f$ has a limit
\begin{align*}
    \lim_{x \longrightarrow c} f(x) = L \hspace{20pt} \text{if and only if} \hspace{20pt} \lim_{x \longrightarrow c^{+}} f(x) = L = \lim_{x \longrightarrow c^{-}} f(x)
\end{align*}
\end{theorem}

\begin{example}
Here is an example of a function with a one-sided limit. We call this the floor function, and we observe it on an attenuated domain $[0, 2]$, as opposed to its full domain, $\mathbb{R}$. 
\begin{align*}
    f(x) = \lfloor x \rfloor, \hspace{4pt} x \in [0, 2]
\end{align*}
With this example, we will show the right-sided limit $L^{+} = 1$.
\begin{proof}
Take $\epsilon = \delta$, where $n \in \mathbb{N}$ is arbitrary. We only need there to exist a $\delta > 0$ such that $x \in (1, 1+\delta)$, when $x$ is in the domain of $f$. For any $x$ in the domain of $f$ greater than and close to $1$, we have $x \in \Big(1, 1+\dfrac{1}{k}\Big)$. So, we choose $\delta = \dfrac{1}{k}$. Thus,
\begin{align*}
    \text{when} \hspace{4pt} x \in (1, 1 + \delta) \hspace{20pt} \text{we have} \hspace{20pt} (f(x) - 1) = (\lfloor x \rfloor - 1) = (1 - 1) = 0 < \dfrac{1}{k} = \delta = \epsilon
\end{align*}
\end{proof}

\resizebox{30em}{30em}{%
\begin{tikzpicture}[scale=\textwidth/4.2cm]
    % title and axes
    \node at (1.3, 1.5) {$f(x)=\lfloor x \rfloor \hspace{4pt} x \in [0, 2]$};
    \draw (0, 0) -- (2, 0)
        node[right] {$x$};
    \draw (0, 0) -- (0, 1.3)
        node[above] {$f(x)$};
    % ----------------------------------
    % range boundaries lower
    \draw[dotted] (0, 0.95) -- (1.05, 0.95);
    \node[] at (-0.2, 0.95) {$1-\epsilon$};
    \node [rotate=90] at (0, 0.95) {(};
    % range boundaries upper
    \draw[dotted] (0, 1.05) -- (1.05, 1.05);
    \node[] at (-0.2, 1.05) {$1+\epsilon$};
    \node [rotate=-90] at (0, 1.05) {(};
    % ----------------------------------
    % domain boundaries lower
    \draw[dotted] (0.95, 0) -- (0.95, 1.1);
    \node [rotate=45] at (0.80, -0.18) {$1-\delta$ $\longrightarrow$};
    \node [] at (0.95, 0) {(};
    % domain boundaries upper
    \draw[dotted] (1.05, 0) -- (1.05, 1.1);
    \node [rotate=-45] at (1.21, -0.18) {$\longleftarrow$ $1+\delta$};
    \node [] at (1.05, 0) {)};
    % ----------------------------------
    % graph
    \draw[blue, very thick] plot[smooth] file {limits_of_functions/python_generated_tables/floor_0_2_piece_0.table};
    \draw[blue, very thick] plot[smooth] file {limits_of_functions/python_generated_tables/floor_0_2_piece_1.table};
    \draw[blue, fill=white] (1,0) circle (.25mm);
    \draw[blue, fill=white] (2,1) circle (.25mm);
\end{tikzpicture}
}
\end{example}

\begin{exercise}
Find the following: 
\begin{align*}
    \lim_{x \longrightarrow 1^{-}} f(x) \hspace{20pt} \text{where} \hspace{20pt} f(x) = \lfloor x \rfloor,  \hspace{4pt} x \in [0, 2]
\end{align*}
\end{exercise}

\begin{exercise}
Prove the following:
\begin{align*}
    \lim_{x \longrightarrow n} \lfloor x \rfloor \hspace{8pt} \text{DNE for any integer} \hspace{8pt} n 
\end{align*}
\end{exercise}

\begin{exercise}
Prove the following:
\begin{align*}
    \lim_{x \longrightarrow 0} \dfrac{\lvert x \rvert}{x} \hspace{8pt} \text{DNE}
\end{align*}
\end{exercise}

\begin{exercise}
Determine if the following exists.
\begin{align*}
    \lim_{x \longrightarrow 4} f(x) \hspace{8pt} \text{given} \hspace{8pt} f(x) = 
    \begin{cases}
    \sqrt{x-4}, \hspace{4pt} &x > 4,\\[2ex]
    8-2x, \hspace{4pt} &x < 4
    \end{cases}
\end{align*}
\end{exercise}

\begin{exercise}
Determine if the following exists:
\begin{align*}
\lim_{x \longrightarrow -1} \dfrac{x^{2}-2x}{x^{2}-x-2}
\end{align*}
\end{exercise}

\begin{definition}
We say $f$ has an infinite limit 
\begin{align*}
    \lim_{x \longrightarrow c} f(x) = \infty
\end{align*}
if for all $\alpha \in \mathbb{R}$ we have some $\delta > 0$ such that
\begin{align*}
    \text{for all} \hspace{4pt} x \hspace{4pt} \text{satisfying} \hspace{4pt} \lvert x - c \rvert < \delta \hspace{4pt} \text{we have} \hspace{4pt} f(x) > \alpha
\end{align*}
\end{definition}

\begin{definition}
We say $f$ has an infinite limit 
\begin{align*}
    \lim_{x \longrightarrow c} f(x) = -\infty
\end{align*}
if for all $\alpha \in \mathbb{R}$ we have some $\delta > 0$ such that
\begin{align*}
    \text{for all} \hspace{4pt} x \hspace{4pt} \text{satisfying} \hspace{4pt} \lvert x - c \rvert < \delta \hspace{4pt} \text{we have} \hspace{4pt} f(x) < \alpha
\end{align*}
\end{definition}

\begin{recall}
Vertical asymptotes
\begin{align*}
    \lim_{x \longrightarrow c^{+}} f(x) = \pm\infty \hspace{20pt} \lim_{x \longrightarrow c^{-}} f(x) = \pm\infty 
\end{align*}
are one-sided infinite limits.
\end{recall}

\begin{definition}
We say $f$ has a limit $L$ at $\infty$ 
\begin{align*}
    \lim_{x \longrightarrow \infty} f(x) = L
\end{align*}
if for each $\alpha \in \mathbb{R}$ and all $\epsilon > 0$ 
\begin{align*}
    \text{there exists} \hspace{4pt} K \in \mathbb{N} \hspace{4pt} \text{such that} \hspace{4pt} K > \alpha \hspace{4pt} \text{and for any} \hspace{4pt} x > K \hspace{4pt} \text{we have} \hspace{4pt} \lvert f(x) - L \rvert < \epsilon
\end{align*}
\end{definition}

\begin{definition}
We say $f$ has a limit $L$ at $-\infty$ 
\begin{align*}
    \lim_{x \longrightarrow -\infty} f(x) = L
\end{align*}
if for each $\alpha \in \mathbb{R}$ and all $\epsilon > 0$ 
\begin{align*}
    \text{there exists} \hspace{4pt} K \in \mathbb{N} \hspace{4pt} \text{such that} \hspace{4pt} -K < \alpha \hspace{4pt} \text{and for any} \hspace{4pt} x < -K \hspace{4pt} \text{we have} \hspace{4pt} \lvert f(x) - L \rvert < \epsilon
\end{align*}
\end{definition}

\begin{exercise}
Find the following:
\begin{align*}
    \lim_{x \longrightarrow -3^{+}} \dfrac{x+2}{x+3}
\end{align*}
\end{exercise}

\begin{exercise}
Find the following:
\begin{align*}
    \lim_{x \longrightarrow -3^{-}} \dfrac{x+2}{x+3}
\end{align*}
\end{exercise}

\begin{exercise}
Find the following:
\begin{align*}
    \lim_{x \longrightarrow 1} \dfrac{2-x}{(x-1)^{2}}
\end{align*}
\end{exercise}

\begin{exercise}
Find the following:
\begin{align*}
    \lim_{x \longrightarrow 3^{+}} \ln(x^{2} - 9)
\end{align*}
\end{exercise}

\begin{exercise}
Is it possible to evaluate the following? Why or why not?
\begin{align*}
    \lim_{x \longrightarrow 3^{-}} \ln(x^{2} - 9)
\end{align*}
\end{exercise}

\begin{exercise}
Find the following:
\begin{align*}
    \lim_{x \longrightarrow 5^{-}} \dfrac{e^{x}}{(x-5)^{3}}
\end{align*}
\end{exercise}

\begin{exercise}
Find the following:
\begin{align*}
    \lim_{x \longrightarrow \pi^{-}} \cot x
\end{align*}
\end{exercise}

\begin{exercise}
Find the following:
\begin{align*}
    \lim_{x \longrightarrow 2\pi^{-}} x\csc x
\end{align*}
\end{exercise}

\begin{exercise}
Find the following:
\begin{align*}
    \lim_{x \longrightarrow 2^{-}} \dfrac{x^{2}-2x}{x^{2}-4x+4}
\end{align*}
\end{exercise}

\newpage
\section{More Theorems and Properties of Limits}

\begin{theorem}
We have the following equivalence:
\begin{align*}
    &\lim_{x \longrightarrow c} g(x) = L \hspace{20pt} \text{if and only if}\\[2ex]
    &\text{for all} \hspace{4pt} \{f(n)\}_{n=1}^{\infty} \hspace{4pt} \text{contained in Dom ($g$) such that} \hspace{4pt} \lim_{n \longrightarrow \infty} f(n)=c\\[2ex]
    &\text{we have} \hspace{20pt} \lim_{n \longrightarrow \infty} g(f(n)) = L
\end{align*}
\label{sequential_criterion_for_limits}
\end{theorem}

\begin{exercise}
Using Theorems \ref{limit_passes_under_square_root}, \ref{sequential_criterion_for_limits}, find the following:
\begin{align*}
    &\text{Suppose} \hspace{4pt} f(x) \geq 0 \hspace{4pt} \text{for all} \hspace{4pt} x \in \text{Dom($f$)}.\\[2ex]
    &\text{If} \hspace{4pt} \lim_{x \longrightarrow c} f(x) = L, \hspace{4pt} \text{then}\\[2ex]
    &\lim_{x \longrightarrow c} \sqrt{f(x)} = \hspace{4pt} ?
\end{align*}
\end{exercise}

\begin{exercise}
If $\lim_{x \longrightarrow c} f(x) = L$, prove that 
\begin{align*}
    \lim_{x \longrightarrow c} \lvert f(x) \rvert = \lvert L \rvert 
\end{align*}
\end{exercise}

\end{document}
