\section{Frobenius Norm}

\begin{tcolorbox}[sharp corners, colback=red!5!white, colframe=black!75!black, fontupper=\color{black}]
\begin{definition}
    For vector space, $V$, and for the inner product, $\langle \cdot, \cdot \rangle$, on vector space $V$, we have the norm of $\vec{v} \in V$ following the form:
    \begin{align*}
        \norm{\vec{v}} \hspace{2pt} = \hspace{2pt} \sqrt{\langle \vec{v}, \vec{v} \rangle}
    \end{align*}
\end{definition}
\end{tcolorbox}

For any vector space, $V$, fashioned with an inner product, we may refer to $V$ as an inner product space.

\begin{tcolorbox}[sharp corners, colback=red!5!white, colframe=black!75!black, fontupper=\color{black}]
\begin{definition}
    For vector space, $V$, and for $\vec{v}, \vec{u} \in V$, where 
    \begin{align*}
        \vec{v} \hspace{2pt} = \hspace{2pt} \begin{bmatrix}
            v_{1}\\ 
            v_{2}\\ 
            \vdots\\
            v_{n}
        \end{bmatrix}, \hspace{10pt}
        \vec{u} \hspace{2pt} = \hspace{2pt}
        \begin{bmatrix}
            u_{1}\\
            u_{2}\\
            \vdots\\
            u_{n}
        \end{bmatrix}
    \end{align*}
    we have the dot product on inner product space, $V$, assuming the following form:
    \begin{align*}
        \langle \vec{v}, \vec{u} \rangle \hspace{2pt} = \hspace{2pt} \vec{v} \cdot \vec{u} \hspace{2pt} = \hspace{2pt} v_{1} \cdot u_{1} + v_{2} \cdot u_{2} + \cdots + v_{n} \cdot u_{n}
    \end{align*}
\end{definition}
\end{tcolorbox}

\begin{tcolorbox}[sharp corners, colback=red!5!white, colframe=black!75!black, fontupper=\color{black}]
\begin{definition}
    The Frobenius Norm of matrix $A$ takes on the following form:
    \begin{align*}
        \norm{A}_{F} = \sqrt{\sum_{i}^{m}\sum_{j}^{n}(a_{i,j})^{2}}
    \end{align*}
\end{definition}
\end{tcolorbox}

How can we think about this norm? Take all rows in matrix $A$ strung together in a single vector. The Frobenius Norm is dot product of this vector with itself. 

\begin{example} \label{frobenius_example_1}
Take the following matrix:
    \begin{align*}
        A \hspace{2pt} = \hspace{2pt}
        \begin{bmatrix}
            1 & 2\\
            3 & 4
        \end{bmatrix} \hspace{2pt} = \hspace{2pt} 
        \begin{bmatrix}
            a_{1,1} & a_{1,2}\\
            a_{2,1} & a_{2,2}
        \end{bmatrix} \hspace{2pt} = \hspace{2pt} 
        \begin{bmatrix}
            \vec{a}_{1}\\
            \vec{a}_{2}
        \end{bmatrix}
    \end{align*}
Then the Frobenius Norm on $A$ is 
    \begin{align*}
        \norm{A}_{F} \hspace{2pt} = \hspace{2pt} \sqrt{\vec{a}_{1} \cdot \vec{a}_{1} + \vec{a}_{2} \cdot \vec{a}_{2}} \hspace{2pt} &= \hspace{2pt} \sqrt{(a_{1,1})^{2} + (a_{1,2})^{2} + (a_{2,1})^{2} + (a_{2,2})^{2}}\\[1ex]
        &= \hspace{2pt} \sqrt{(1)^{2} + (2)^{2} + (3)^{2} + (4)^{2}}
    \end{align*}

    \begin{tcolorbox}[sharp corners, colback=black!5!white, colframe=black!75!black, fontupper=\color{black}]
        \begin{lstlisting}[language=python, gobble = 12]
            import numpy as np

            A = np.array([[1,2],[3,4]])
            print("A:")
            print(A)
            
            A_frob = np.linalg.norm(A,'fro')
            print("")
            print("A_frob:")
            print(A_frob)
        \end{lstlisting}
    \end{tcolorbox}

    \begin{tcolorbox}[sharp corners, colback=green!10!white, colframe=black!75!black, fontupper=\color{black}]
        \begin{lstlisting}[language=python, gobble = 12]
            A:
            [[1 2]
            [3 4]]

            A_frob:
            5.477225575051661
        \end{lstlisting}
    \end{tcolorbox}
\end{example}

\begin{tcolorbox}[sharp corners, colback=yellow!5!white, colframe=black!75!black, fontupper=\color{black}]
\begin{theorem}
    The Cauchy-Schwarz Inequality states that, where $V$ is a vector space, 
    \begin{align*}
        \vec{v}, \vec{u} \in V \hspace{20pt} \Longrightarrow \hspace{20pt} \lvert \langle \vec{v} , \vec{u} \rangle \rvert \hspace{2pt} \leq \hspace{2pt} \norm{\vec{v}} \norm{\vec{u}}
    \end{align*}
    Equality holds whenever either $\vec{u}$ or $\vec{v}$ is a scalar multiple of the other.
\end{theorem}
\end{tcolorbox}

The Cauchy-Schwarz Inequality will be used to show the following inequality:
\begin{align*}
    \norm{A\vec{v}} \hspace{2pt} \leq \hspace{2pt} \norm{A}_{F} \norm{\vec{v}}
\end{align*}

\begin{proof}
    With our vector space fashioned with the dot product, we have that
    \begin{align*}
        \norm{A\vec{v}} \hspace{2pt} &= \hspace{2pt} \norm{
        \begin{bmatrix}
            \vec{a}_{1}\\
            \vec{a}_{2}\\
            \vdots\\
            \vec{a}_{m}
        \end{bmatrix}
        \vec{v} \hspace{2pt}
        } \hspace{2pt} = \hspace{2pt} \norm{
        \begin{bmatrix}
            \vec{a}_{1} \cdot \vec{v}\\
            \vec{a}_{2} \cdot \vec{v}\\
            \vdots\\
            \vec{a}_{m} \cdot \vec{v}
        \end{bmatrix}
        } \hspace{2pt} = \hspace{2pt} \norm{
        \begin{bmatrix}
            a_{1,1} \cdot v_{1} + a_{1,2} \cdot v_{2} + \cdots a_{1,n} \cdot v_{n}\\
            a_{2,1} \cdot v_{1} + a_{2,2} \cdot v_{2} + \cdots a_{2,n} \cdot v_{n}\\
            \vdots\\
            a_{m,1} \cdot v_{1} + a_{m,2} \cdot v_{2} + \cdots a_{m,n} \cdot v_{n}\\
        \end{bmatrix}
        }\\[2ex]
        &= \hspace{2pt} \sqrt{ 
        \begin{bmatrix}
            a_{1,1} \cdot v_{1} + a_{1,2} \cdot v_{2} + \cdots a_{1,n} \cdot v_{n}\\
            a_{2,1} \cdot v_{1} + a_{2,2} \cdot v_{2} + \cdots a_{2,n} \cdot v_{n}\\
            \vdots\\
            a_{m,1} \cdot v_{1} + a_{m,2} \cdot v_{2} + \cdots a_{m,n} \cdot v_{n}\\
        \end{bmatrix}
        \cdot
        \begin{bmatrix}
            a_{1,1} \cdot v_{1} + a_{1,2} \cdot v_{2} + \cdots a_{1,n} \cdot v_{n}\\
            a_{2,1} \cdot v_{1} + a_{2,2} \cdot v_{2} + \cdots a_{2,n} \cdot v_{n}\\
            \vdots\\
            a_{m,1} \cdot v_{1} + a_{m,2} \cdot v_{2} + \cdots a_{m,n} \cdot v_{n}\\
        \end{bmatrix}
        } \\[2ex]
        &= \hspace{2pt} \sqrt{(a_{1,1} \cdot v_{1} + \cdots a_{1,n} \cdot v_{n})^{2} + (a_{2,1} \cdot v_{1} + \cdots a_{2,n} \cdot v_{n})^{2} + \cdots + (a_{m,1} \cdot v_{1} + \cdots a_{m,n} \cdot v_{n})^{2}}
    \end{align*}
    By the Cauchy-Schwarz Inequality, and by the definition of the Frobenius Norm, we have
    \begin{align*}
        \norm{A\vec{v}}^{2} \hspace{2pt} &= \hspace{2pt} (a_{1,1} \cdot v_{1} + \cdots a_{1,n} \cdot v_{n})^{2} + (a_{2,1} \cdot v_{1} + \cdots a_{2,n} \cdot v_{n})^{2} + \cdots + (a_{m,1} \cdot v_{1} + \cdots a_{m,n} \cdot v_{n})^{2}\\[2ex]
        &= \hspace{2pt} \left( \begin{bmatrix}
            a_{1,1}\\
            \vdots\\
            a_{1,n}
        \end{bmatrix}
        \cdot
        \begin{bmatrix}
            v_{1}\\
            \vdots\\
            v_{n}
        \end{bmatrix} \right)^{2}
        + \cdots +
        \left( \begin{bmatrix}
            a_{m,1}\\
            \vdots\\
            a_{m,n}
        \end{bmatrix}
        \cdot
        \begin{bmatrix}
            v_{1}\\
            \vdots\\
            v_{n}
        \end{bmatrix} \right)^{2}\\[2ex]
        &= \hspace{2pt} \left( \begin{bmatrix}
            a_{1,1}\\
            \vdots\\
            a_{1,n}
        \end{bmatrix}
        \cdot
        \begin{bmatrix}
            v_{1}\\
            \vdots\\
            v_{n}
        \end{bmatrix} \right)
        \cdot
        \left( \begin{bmatrix}
            a_{1,1}\\
            \vdots\\
            a_{1,n}
        \end{bmatrix}
        \cdot
        \begin{bmatrix}
            v_{1}\\
            \vdots\\
            v_{n}
        \end{bmatrix} \right)
        + \cdots +
        \left( \begin{bmatrix}
            a_{m,1}\\
            \vdots\\
            a_{m,n}
        \end{bmatrix}
        \cdot
        \begin{bmatrix}
            v_{1}\\
            \vdots\\
            v_{n}
        \end{bmatrix} \right)
        \cdot
        \left( \begin{bmatrix}
            a_{m,1}\\
            \vdots\\
            a_{m,n}
        \end{bmatrix}
        \cdot
        \begin{bmatrix}
            v_{1}\\
            \vdots\\
            v_{n}
        \end{bmatrix} \right)\\[2ex]
        &= \hspace{2pt} \left(\sqrt{\left( \begin{bmatrix}
            a_{1,1}\\
            \vdots\\
            a_{1,n}
        \end{bmatrix}
        \cdot
        \begin{bmatrix}
            v_{1}\\
            \vdots\\
            v_{n}
        \end{bmatrix} \right)
        \cdot
        \left( \begin{bmatrix}
            a_{1,1}\\
            \vdots\\
            a_{1,n}
        \end{bmatrix}
        \cdot
        \begin{bmatrix}
            v_{1}\\
            \vdots\\
            v_{n}
        \end{bmatrix} \right)}\right)^{2}
        + \cdots +
        \left(\sqrt{\left( \begin{bmatrix}
            a_{m,1}\\
            \vdots\\
            a_{m,n}
        \end{bmatrix}
        \cdot
        \begin{bmatrix}
            v_{1}\\
            \vdots\\
            v_{n}
        \end{bmatrix} \right)
        \cdot
        \left( \begin{bmatrix}
            a_{m,1}\\
            \vdots\\
            a_{m,n}
        \end{bmatrix}
        \cdot
        \begin{bmatrix}
            v_{1}\\
            \vdots\\
            v_{n}
        \end{bmatrix} \right)}\right)^{2}\\[2ex]
        &= \hspace{2pt} \norm{
        \begin{bmatrix}
            a_{1,1}\\
            \vdots\\
            a_{1,n}
        \end{bmatrix}
        \cdot
        \begin{bmatrix}
            v_{1}\\
            \vdots\\
            v_{n}
        \end{bmatrix}
        }^{2} 
        + \cdots + 
        \norm{
        \begin{bmatrix}
            a_{m,1}\\
            \vdots\\
            a_{m,n}
        \end{bmatrix}
        \cdot
        \begin{bmatrix}
            v_{1}\\
            \vdots\\
            v_{n}
        \end{bmatrix}
        }^{2}\\[2ex]
        &= \hspace{2pt} \norm{\vec{a}_{1} \cdot \vec{v}}^{2} + \cdots + \norm{\vec{a}_{m} \cdot \vec{v}}^{2}\\[1ex]
        &\leq \hspace{2pt} \norm{\vec{a}_{1}}^{2} \norm{\vec{v}}^{2} + \cdots + \norm{\vec{a}_{m}}^{2} \norm{\vec{v}}^{2}\\[1ex]
        &= \hspace{2pt} (\norm{\vec{a}_{1}}^{2} + \cdots + \norm{\vec{a}_{m}}^{2}) \norm{\vec{v}}^{2}\\[1ex]
        &= \hspace{2pt} \left( \hspace{10pt}
        \left(
        \sqrt{
        \left( 
        \begin{bmatrix}
            a_{1,1}\\
            \vdots\\
            a_{1,n}
        \end{bmatrix}
        \cdot
        \begin{bmatrix}
            a_{1,1}\\
            \vdots\\
            a_{1,n}
        \end{bmatrix}
        \right)
        }
        \right)^{2}
        + \cdots +
        \left(
        \sqrt{
        \left( 
        \begin{bmatrix}
            a_{m,1}\\
            \vdots\\
            a_{m,n}
        \end{bmatrix}
        \cdot
        \begin{bmatrix}
            a_{m,1}\\
            \vdots\\
            a_{m,n}
        \end{bmatrix}
        \right)
        }
        \right)^{2} \hspace{10pt}
        \right) \norm{\vec{v}}^{2}\\[2ex]
        &= \hspace{2pt} \left( \hspace{10pt}
        \left(
        \begin{bmatrix}
            a_{1,1}\\
            \vdots\\
            a_{1,n}
        \end{bmatrix}
        \cdot
        \begin{bmatrix}
            a_{1,1}\\
            \vdots\\
            a_{1,n}
        \end{bmatrix}
        \right)
        + \cdots +
        \left( 
        \begin{bmatrix}
            a_{m,1}\\
            \vdots\\
            a_{m,n}
        \end{bmatrix}
        \cdot
        \begin{bmatrix}
            a_{m,1}\\
            \vdots\\
            a_{m,n}
        \end{bmatrix}
        \right) \hspace{10pt}
        \right) \norm{\vec{v}}^{2}\\[2ex]
        &= \hspace{2pt} ( \hspace{10pt} (a_{1,1})^{2} + \cdots + (a_{1,n})^{2} + \cdots \hspace{10pt} \cdots \hspace{10pt} \cdots \hspace{10pt} + (a_{m,1})^{2} + \cdots + (a_{m,n})^{2} \hspace{10pt} ) \norm{\vec{v}}^{2}\\[1ex]
        &= \hspace{2pt} \sum_{i}^{m}\sum_{j}^{n}(a_{i,j})^{2} \norm{\vec{v}}^{2}\\[1ex]
        &= \hspace{2pt} \norm{A}_{F}^{2} \norm{\vec{v}}^{2}
    \end{align*}
    Finally, we have
    \begin{align*}
        \norm{A\vec{v}}^{2} \hspace{2pt} \leq \hspace{2pt} \norm{A}_{F}^{2} \norm{\vec{v}}^{2}
    \end{align*}
    which gives us
    \begin{align*}
        \norm{A\vec{v}} \hspace{2pt} \leq \hspace{2pt} \norm{A}_{F} \norm{\vec{v}}
    \end{align*}
\end{proof}

\begin{example}
    Take matrix $A$ from Example \ref{frobenius_example_1}.
    \begin{align*}
        A \hspace{2pt} = \hspace{2pt}
        \begin{bmatrix}
            1 & 2\\
            3 & 4
        \end{bmatrix}
    \end{align*}
    Take the following vector
    \begin{align*}
        \vec{v} \hspace{2pt} = \hspace{2pt}
        \begin{bmatrix}
            5\\
            7
        \end{bmatrix}
    \end{align*}
    and observe the Frobenius inequality holds.
    \begin{align*}
        \norm{A\vec{v}} \hspace{2pt} &= \hspace{2pt} \norm{
        \begin{bmatrix}
            1 & 2\\
            3 & 4
        \end{bmatrix}
        \begin{bmatrix}
            5\\
            7
        \end{bmatrix}
        } \hspace{2pt} = \hspace{2pt} \norm{
        \begin{bmatrix}
            19\\
            43
        \end{bmatrix}
        } \hspace{2pt} = \hspace{2pt}
        \norm{
        \begin{bmatrix}
            1 \cdot 5 + 2 \cdot 7\\
            3 \cdot 5 + 4 \cdot 7
        \end{bmatrix}
        } \hspace{2pt} = \hspace{2pt}
        \sqrt{
        \begin{bmatrix}
            1 \cdot 5 + 2 \cdot 7\\
            3 \cdot 5 + 4 \cdot 7
        \end{bmatrix}
        \cdot
        \begin{bmatrix}
            1 \cdot 5 + 2 \cdot 7\\
            3 \cdot 5 + 4 \cdot 7
        \end{bmatrix}
        }\\[2ex]
        &= \hspace{2pt} \sqrt{(1 \cdot 5 + 2 \cdot 7)^{2} + (3 \cdot 5 + 4 \cdot 7)^{2}}
    \end{align*}
    So, following the proof of the Frobenius inequality,
    \begin{align*}
        \norm{A\vec{v}}^{2} \hspace{2pt} &= \hspace{2pt} \norm{
        \begin{bmatrix}
            1 & 2\\
            3 & 4
        \end{bmatrix}
        \begin{bmatrix}
            5\\
            7
        \end{bmatrix}
        }^{2} \\[2ex] 
        &= \hspace{2pt} (1 \cdot 5 + 2 \cdot 7)^{2} + (3 \cdot 5 + 4 \cdot 7)^{2} \hspace{2pt} = \hspace{2pt} \left(
        \begin{bmatrix}
            1\\
            2
        \end{bmatrix}
        \cdot
        \begin{bmatrix}
            5\\
            7
        \end{bmatrix}
        \right)
        \cdot
        \left(
        \begin{bmatrix}
            1\\
            2
        \end{bmatrix}
        \cdot
        \begin{bmatrix}
            5\\
            7
        \end{bmatrix}
        \right)
        +
        \left(
        \begin{bmatrix}
            3\\
            4
        \end{bmatrix}
        \cdot
        \begin{bmatrix}
            5\\
            7
        \end{bmatrix}
        \right)
        \cdot
        \left(
        \begin{bmatrix}
            3\\
            4
        \end{bmatrix}
        \cdot
        \begin{bmatrix}
            5\\
            7
        \end{bmatrix}
        \right)\\[2ex]
        &= \hspace{2pt} \left(
        \sqrt{
        \left(
        \begin{bmatrix}
            1\\
            2
        \end{bmatrix}
        \cdot
        \begin{bmatrix}
            5\\
            7
        \end{bmatrix}
        \right)
        \cdot
        \left(
        \begin{bmatrix}
            1\\
            2
        \end{bmatrix}
        \cdot
        \begin{bmatrix}
            5\\
            7
        \end{bmatrix}
        \right)
        }
        \right)^{2}
        +
        \left(
        \sqrt{
        \left(
        \begin{bmatrix}
            3\\
            4
        \end{bmatrix}
        \cdot
        \begin{bmatrix}
            5\\
            7
        \end{bmatrix}
        \right)
        \cdot
        \left(
        \begin{bmatrix}
            3\\
            4
        \end{bmatrix}
        \cdot
        \begin{bmatrix}
            5\\
            7
        \end{bmatrix}
        \right)
        }
        \right)^{2}\\[2ex]
        &= \hspace{2pt} \norm{
        \begin{bmatrix}
            1\\
            2
        \end{bmatrix}
        \cdot
        \begin{bmatrix}
            5\\
            7
        \end{bmatrix}
        }^{2} 
        +
        \norm{
        \begin{bmatrix}
            3\\
            4
        \end{bmatrix}
        \cdot
        \begin{bmatrix}
            5\\
            7
        \end{bmatrix}
        }^{2} \\[2ex]
        &\leq \hspace{2pt} \norm{
        \begin{bmatrix}
            1\\
            2
        \end{bmatrix}
        }^{2} \norm{
        \begin{bmatrix}
            5\\
            7
        \end{bmatrix}
        }^{2}
        +
        \norm{
        \begin{bmatrix}
            3\\
            4
        \end{bmatrix}
        }^{2} \norm{
        \begin{bmatrix}
            5\\
            7
        \end{bmatrix}
        }^{2} \\[2ex]
        &= \hspace{2pt} \left( \hspace{10pt}
        \left(
        \sqrt{
        \left(
        \begin{bmatrix}
            1\\
            2
        \end{bmatrix}
        \cdot
        \begin{bmatrix}
            1\\
            2
        \end{bmatrix}
        \right)
        }
        \right)^{2}
        + 
        \left(
        \sqrt{
        \left(
        \begin{bmatrix}
            3\\
            4
        \end{bmatrix}
        \cdot
        \begin{bmatrix}
            3\\
            4
        \end{bmatrix}
        \right)
        }
        \right)^{2}
        \hspace{10pt} \right) \norm{
        \begin{bmatrix}
            5\\
            7
        \end{bmatrix}
        }^{2} \\[2ex]
        &= \hspace{2pt} \left( \hspace{10pt}
        \left(
        \begin{bmatrix}
            1\\
            2
        \end{bmatrix}
        \cdot
        \begin{bmatrix}
            1\\
            2
        \end{bmatrix}
        \right)
        + 
        \left(
        \begin{bmatrix}
            3\\
            4
        \end{bmatrix}
        \cdot
        \begin{bmatrix}
            3\\
            4
        \end{bmatrix}
        \right) 
        \hspace{10pt} \right) \norm{
        \begin{bmatrix}
            5\\
            7
        \end{bmatrix}
        }^{2} \\[2ex]
        &= \hspace{2pt} (\hspace{10pt} (1)^{2} + (2)^{2} + (3)^{2} + (4)^{2} \hspace{10pt}) \norm{
        \begin{bmatrix}
            5\\
            7
        \end{bmatrix}
        }^{2} \hspace{2pt} = \hspace{2pt} \norm{
        \begin{bmatrix}
            1 & 2\\
            3 & 4
        \end{bmatrix}
        }_{F}^{2} \norm{
        \begin{bmatrix}
            5\\
            7
        \end{bmatrix}
        }^{2} \hspace{2pt} = \hspace{2pt} \norm{A}_{F}^{2} \norm{\vec{v}}^{2}
    \end{align*}
    So, we have
    \begin{align*}
        \norm{A\vec{v}} \hspace{2pt} &= \hspace{2pt} \norm{
        \begin{bmatrix}
            1 & 2\\
            3 & 4
        \end{bmatrix}
        \begin{bmatrix}
            5\\
            7
        \end{bmatrix}
        } \hspace{2pt} \leq \hspace{2pt} \norm{
        \begin{bmatrix}
            1 & 2\\
            3 & 4
        \end{bmatrix}
        }_{F} \norm{
        \begin{bmatrix}
            5\\
            7
        \end{bmatrix}
        } \hspace{2pt} = \hspace{2pt} \norm{A}_{F} \norm{\vec{v}}
    \end{align*}

    \begin{tcolorbox}[sharp corners, colback=black!5!white, colframe=black!75!black, fontupper=\color{black}]
        \begin{lstlisting}[language=python, gobble = 12]
            import numpy as np

            A = np.array([[1,2],[3,4]])
            v = np.array([5,7])
            A_row1 = np.array([1,2])
            A_row1_v = np.dot(A_row1,v)
            A_row1_v_norm = np.linalg.norm(A_row1_v)
            A_row1_norm = np.linalg.norm(A_row1)
            A_row2 = np.array([3,4])
            A_row2_v = np.dot(A_row2,v)
            A_row2_v_norm = np.linalg.norm(A_row2_v)
            A_row2_norm = np.linalg.norm(A_row2)
            v_norm = np.linalg.norm(v)
            Av = A@v
            Av_norm = np.linalg.norm(Av)
            A_frob_norm = np.linalg.norm(A,'fro')

            print("")
            print("A:")
            print(A)
            print("")
            print("v:")
            print(v)
            print("")
            print("A_row1_v_norm:")
            print(A_row1_v_norm)
            print("")
            print("A_row2_v_norm:")
            print(A_row2_v_norm) 
            print("")
            print("A_row1_norm * v_norm:")
            print(A_row1_norm * v_norm)
            print("")
            print("A_row2_norm * v_norm:")
            print(A_row2_norm * v_norm)
            print("")
            print("Av:")
            print(Av)
            print("")
            print("Av_norm:")
            print(Av_norm)
            print("")
            print("A_frob_norm * v_norm:")
            print(A_frob_norm * v_norm)
        \end{lstlisting}
    \end{tcolorbox}

    \begin{tcolorbox}[sharp corners, colback=green!10!white, colframe=black!75!black, fontupper=\color{black}]
        \begin{lstlisting}[language=python, gobble = 12]
            A:
            [[1 2]
             [3 4]]

            v:
            [5 7]

            A_row1_v_norm:
            19.0

            A_row2_v_norm:
            43.0

            A_row1_norm * v_norm:
            19.235384061671347

            A_row2_norm * v_norm:
            43.01162633521314

            Av:
            [19 43]

            Av_norm:
            47.01063709417264

            A_frob_norm * v_norm:
            47.11687595755899
        \end{lstlisting}
    \end{tcolorbox}
\end{example}

